% https://latex.ihep.ac.cn/project/63d5f0c2e31bc50081b059d4
%
\documentclass[aps,preprint,showkeys,superscriptaddress]{revtex4}
%%\documentclass[aps,showkeys,superscriptaddress,12pt,tightenlines]{revtex4}
%\documentclass[12pt]{article}
\usepackage{times}
\usepackage{amsmath}
\usepackage{epsfig}
\usepackage{dcolumn}
\usepackage{bm}
\usepackage[utf8]{inputenc}
\usepackage{graphicx}
\usepackage{float} 
\usepackage{subfigure}
%\usepackage{subcaption}

% use Chinese
\usepackage{CJKutf8}
\newcommand{\chinese}[1]{\begin{CJK}{UTF8}{gbsn}#1\end{CJK}}

\usepackage{color}
\newcommand{\red}[1]{\textcolor{red}{#1}}

% definitions of particles
\newcommand{\pp}{$p$}
\newcommand{\pim}{$\pi^-$}
\newcommand{\nbar}{$\bar{n}$}
\newcommand{\jpsi}{$J/\psi$}

% definitions of resolutions
\newcommand{\pres}{$\sigma_p$}
\newcommand{\tres}{$\sigma_\theta$}
\newcommand{\phires}{$\sigma_\phi$}

\newcommand{\tbd}{\red{[tbd]}\ }

\parskip=5pt plus 1pt minus 1pt

\begin{document}

\title{\boldmath An approach to study interactions of antineutrons with CsI \\
at a \jpsi~factory}
\author{Si-Cheng Yuan}\email{yuansicheng@ihep.ac.cn} 
\author{Liang-Liang Wang}\email{llwang@ihep.ac.cn} 
\author{Wei-Dong Li}\email{liwd@ihep.ac.cn} 

\affiliation{Institute of High Energy Physics, Chinese Academy of Sciences, Beijing 100049, China}
\affiliation{University of Chinese Academy of Sciences, Beijing 100049, China}

\date{\today}

\begin{abstract}

Cesium Iodide (CsI) crystals are widely used in high-energy physics 
for their scintillation properties that enable detection of charged and neutral 
particles via direct and indirect ionization and form the basis of electromagnetic
calorimeters. However, knowledge of antineutron interactions with CsI 
is limited due to the difficulty of obtaining sources of antineutron
of sufficient intensity and energy definition. 
As antineutron are abundantly produced by many processes it would be 
particularly useful to improve understanding of the interactions of antineutrons
with CsI crystals. 

We propose to use the decay $J/\psi\to p\pi^-\bar{n}$ at the BEPCII $J/\psi$ factory
as a source of antineutrons using the BESIII detector with a CsI target added between 
the beam pipe and the detector. The BESIII Monte Carlo simulation with varying thicknesses of CsI target 
is used to validate the approach and optimize the target thickness. 
Selecting $p\pi^-$ charged particle tracks from the Monte Carlo we obtain clean antineutron samples 
with well defined momentum and direction. 
The selection efficiency, momentum and angular resolutions, as well as the interaction probability 
between antineutron and the CsI target are estimated. 

As the CsI thickness is increased more antineutron CsI interactions are obtained,
however the quality of the $p\pi^-$ selection is degraded. The Monte Carlo study yields
an optimum thickness that balances these effects. 
This approach can be applied to similar experiments with other types of target materials to measure baryons 
such as liquid hydrogen/deuterium and $\Lambda/\Xi$ hyperons.
\end{abstract}

\keywords{Antineutron, Cesium Iodide, Calorimeter, $J/\psi$ decays}

\maketitle

\section{Introduction}

Many decays of $J/\psi$, $\psi(2S)$, $\Lambda_c$, and $B$~\cite{pdg} have final
states that include the antineutron, but the detection of antineutron is
limited as it is a neutral particle that does not directly cause ionisation.
In some cases the antineutron is inferred from the recoil of the other final
state particles~\cite{BESIII:2012imn,CLEO:2008aum}.  Alternatively it is
identified by examining the shower produced by its annihilation interaction 
with a calorimeter~\cite{BESIII:2021tbq}.  Detection with a calorimeter requires
reliable Monte Carlo (MC) simulation of the interaction of the antineutron the
calorimeter material to provide the selection efficiency and effective
background suppression. 

Cesium Iodide (CsI) has high luminous intensity, high luminescence, small
radiation length and is chemically stable. It is widely used as the
interaction matter of electromagnetic calorimeters for particle and nuclear
physics experiments, such as BESIII~\cite{bes3}, CLEO-c~\cite{cleoc}, and
Belle~II~\cite{belle2}. In these experiments, the interaction of antineutron
and CsI is simulated with {\sc geant4}~\cite{geant4}, however the agreement
between data and MC simulation is usually poor and data driven methods are used
to estimate the efficiency but the precision remains
limited and the full event shape information is not fully utilised~\cite{Liu:2021rrx}.  

To achieve better precision for the antineutron involved measurements, more
reliable MC simulation is needed, and knowledge on the antineutron and the
calorimeter material interaction is necessary while so far it is not available
due to extremely rare experimental data on antineutron interaction with any
material. The main reason of the lack of data is due to the fact that
antineutron beams with suitable intensity and momentum control are typically
difficult to be obtained. 

The best antineutron sources obtained so far are from
BNL E-767~\cite{E767} and the CERN OBELIX~\cite{OBELIX} experiments with
limited statistics and momentum range ($<500$~MeV/$c$), and the experimental
measurements are more related to the interaction with light nuclei~\cite{nbarPhysics}.

Because of the high cross section of $e^+e^-\to J/\psi$ and large decay
branching fractions of $J/\psi$ to antineutrons like $J/\psi\to p\pi^-\bar{n}$,
super \jpsi~factories are proposed to provide antineutron sources by tagging
the accompanying particles like $p$ and $\pi^-$ in $J/\psi\to p\pi^-\bar{n}$
and to perform rich experiments related to nuclear and particle physics by
placing specific custom-made targets just outside of the beam
pipe~\cite{hypronProjectileFromJpsi}. 

Such antineutron sources have the advantage of high statistics 
and also the maximum momentum of antineutrons from $J/\psi\to p\pi^-\bar{n}$ 
can reach $1174$~MeV/$c$ uniquely enabling studies beyond 500~MeV/$c$.

The BESIII experiment~\cite{bes3} is an $e^+e^-$ annihilation experiment which
can provide a copious source of antineutrons enabling novel measurements.
We propose a measurement of antineutron interactions with CsI 
using antineutron produced in $e^+e^-\to J/\psi\to p\pi^-\bar{n}$ by 
using a CsI target placed between the beam pipe and the BESIII detector. 
%This will allow a measurement of many final states of the interactions and the data will be
%essential for the experiments, whether in operation or being proposed, which
%use CsI as the electromagnetic calorimeter. 

In this article, we propose to study the interaction between
antineutrons and CsI, using antineutron source from $J/\psi$ decays and a CsI
target. By taking BESIII~\cite{bes3} as a demonstration, we perform
quantitative validations of the proposal based on full MC simulation. Adding a
CsI target in the full simulation of BESIII is described in
Sec.~\ref{sec:target}, the simulation results of single $p$, single $\pi^+$,
single $\bar{n}$, and $J/\psi\to p\pi^-$\nbar\ events are shown in
Sec.~\ref{sec:results}, and the conclusion and perspectives are discussed in
Sec.~\ref{sec:conclusion}. 

\section{\boldmath The ${\rm CsI}$ target and antineutron source at BESIII}
\label{sec:target}
%    
The BESIII~\cite{bes3} detector at the BEPCII~\cite{bepc2} $e^+e^-$ collider
runs in the tau-charm energy region, enabling a rich physics program 
including light hadron spectroscopy, charmonium and charmoniumlike exotic states, 
charm physics, QCD studies with light meson decays, tau physics 
and more~\cite{bes3physics,bes3physicsFuture}.
Due to the large production rate of $e^+e^-\to J/\psi$, it is also a $J/\psi$
factory. The BESIII detector, as described in detail in Ref.~\cite{bes3},
consists of a multi-layer drift chamber (MDC) filled with helium-based gas
outside of a beryllium beam pipe, time-of-flight (TOF) counters made of plastic
scintillators, an electromagnetic calorimeter made of CsI(Tl) crystals, a
superconducting magnet providing a field of 1~Tesla, and a muon system made of
resistive plate chambers.
    
The outer radius and the length of the BESIII beam pipe are 33.7~mm and 296~mm,
respectively. The inner radius of MDC is 59.2~mm. The gap between the beam pipe
and MDC is filled with air. Adding some target material within this gap does
not affect the beam, although some degradation in detector performance is expected.  
%
The proposed target material, CsI (we use pure CsI in the simulation, 
neglecting the Thallium), has a one to one ratio of Cesium and Iodide atoms 
with a density of $4.51$~g/cm$^3$. The target is formed into a tube shape 
of the same length as the outer beryllium tube of the beam pipe which it is attached to. 
The thickness of additional CsI is the most critical parameter, greater thickness 
of CsI yields more antineutron interactions, however it also reduces the efficiency 
and resolution of tagging the proton and $\pi^-$ hence reducing efficiency
and angular and energy resolution of the tagged antineutrons.
For these reasons, the optimum thickness is a balance between the
quantity of interactions and the quality with which they are measured as
detailed below.   


The BESIII experiment~\cite{bes3} collects 10~billion \jpsi\ events each year.
The branching fraction for $\jpsi \to p\pi^-\bar{n}$ is $(2.12\pm 0.09)\times 10^{-3}$~\cite{pdg}
giving this approach strong potential to provide a copious source of antineutrons. 
Identification of the proton and the $\pi^-$ allows a recoil mass of the $p\pi^-$ system
to be determined. A clear antineutron signal at recoil mass of 0.94~GeV/$c^2$ 
over a very low background is obtained, providing a sample of antineutrons with more 
than 99\% purity. Selecting events with antineutron interactions within the additional CsI target
is similar to the standard selection with an additional requirement that the
charged tracks excluding \pp\ and \pim\ originate from a secondary vertex
located within the CsI tube. 

% THIS PARAGRAPH SAYS NOTHING NEW : SO I REMOVE IT 
%The goal of our designed experiment is to study the antineutron and CsI
%interactions, and the quantity and quality of the tagged antineutrons and the
%quantity of the antineutron and CsI interactions. 
%As mentioned in the previous section, quantity refers to how many
%antineutrons can be produced and the efficiency of the tagging process, while
%quality refers to the resolutions, including momentum resolution and angular
%resolution of the predicted antineutrons, both of which are important for the
%following physics analysis. 

In Geant4, the simulation of \pp\ and \pim\ is
regarded as trustworthy, so the effect of the added additional CsI on them can be simulated
reasonably well. This contrasts with the simulation of the interaction between 
antineutron and the CsI target which only provides a rough indication. 

% THE FOLLOWING PARAGRAPH PROVIDES NO USEFUL INFORMATION 
% EITHER ADD LOTS OF DETAIL TO EXPLAIN YOUR VAGUE STATEMENTS 
% OR REMOVE AS I HAVE DONE
%
%the incorrect EMC simulation results indicate that the
%simulation is not very reliable, nonetheless, we can get a rough feeling about
%the reliability by comparing the interactions between antineutron with beam
%pipe and antineutron with additional CsI. Although not accurate enough, it
%still provide useful information.
    
The official BESIII software {\sc boss} is used for simulation and 
reconstruction~\cite{boss,bes3physics,bes3hough} with modifications
for the additional CsI target. The simulation is modified by adding the CsI tube 
of the target into the detector structure definition. 
The reconstruction is modified to account for effects of additional energy loss 
and multiple scattering on the track fitting which uses a Kalman filter~\cite{kalFit}.
A global service is implemented to ensure consistency of the material properties and geometry  
used by the simulation and reconstruction. 

Single proton, $\pi^-$, and antineutron particle samples, as well as $J/\psi\to p \pi^- \bar{n}$ 
samples are generated for each of the configured thicknesses of the CsI tube. 
These simulated samples allow studies of efficiencies and resolutions and a determination 
of the optimum thickness of the CsI tube. 

%%%%%%%%% HERE

\section{Study of simulated samples}
\label{sec:results}

Firstly, we study how the additional CsI layer affects track reconstruction 
for protons, pions, and antineutrons. Subsequenly we examine how the degraded track 
reconstruction combines for the selection of $J/\psi\to p \pi^- \bar{n}$. 
For simplicity, background particles introduced by the accelerator beam are not taken 
into account in the simulation.

\subsection{Study of single proton and pion samples}

The tagged particles in our study are \pp\ and \pim. Within the same material 
\pp\ looses more energy than \pim due to its large mass, thus at low
momentum the efficiency of \pp\ is much lower than that of \pim. For this
reason, different momentum ranges are set in \pp\ and \pim simulations. The
minimum momentum is 0.3~GeV/$c$ for \pp\ and 0.1~GeV/$c$ for \pim; meanwhile,
the maximum momentum is 1.2~GeV/$c$ for both \pp\ and \pim. As for the polar
angle $\theta$, the range of $\cos\theta$ is set to $[0,~0.93]$, half of the
detector coverage. The particles are distributed uniformly in both momentum and
$\cos\theta$. In total, $10^6$ events of each \pp\ and \pim\ sample are
simulated.
        
For the tagged particles, we are mainly concerned with the efficiency of
reconstruction and the resolutions of momentum and direction. A Kalman filter is
applied to the reconstructed tracks and the efficiency is defined as the number
of selected tracks divided by the number of generated ones. After selection,
the tracks are divided into momentum bins of width 0.05~GeV/$c$ 
and polar angle bins of width 0.05~radian.
The difference between the reconstructed momentum and polar angle and the MC truth 
for each track is used to estimate the resolution, these are combined for all 
tracks with a Gaussian fit to yield the resolutions of momentum and polar angle.
        
Figure~\ref{tracking} shows the comparison of the proton and $\pi^-$ tracking
efficiency loss ($1-\varepsilon$) without and with 10~mm CsI as a function of
momentum and $\cos\theta$. As expected the added CsI material 
causes a greater reduction if the reconstruction efficiency at lower momentum 
and larger angles. Also the impact with \pp\ is far greater than with \pim\ 
due to the large $dE/dx$ for low momentum protons.
        
        \begin{figure}[htbp]
        	\centering  %图片全局居中
        	\subfigbottomskip=2pt %两行子图之间的行间距
        	\subfigcapskip=-5pt %设置子图与子标题之间的距离
        	\subfigure[]{
        		\includegraphics[width=0.4\linewidth]{figs/single_ppi/single_p+_1-efficiency_0.eps}}
        	\subfigure[]{
        		\includegraphics[width=0.4\linewidth]{figs/single_ppi/single_p+_1-efficiency_10.eps}}
        	\subfigure[]{
        		\includegraphics[width=0.4\linewidth]{figs/single_ppi/single_pi-_1-efficiency_0.eps}}
        	\subfigure[]{
        		\includegraphics[width=0.4\linewidth]{figs/single_ppi/single_pi-_1-efficiency_10.eps}}
        	\caption{ Loss of tracking efficiency as a function of momentum and $\cos\theta$ for: 
(a) protons without a CsI layer, 
(b) protons with additional 10 mm thick CsI layer, 
(c) pions without a CsI layer, 
(d) pions with additional 10 mm thick CsI layer.  }
        	\label{tracking}
        \end{figure}
        
The momentum resolution~(\pres), the polar angle resolution~(\tres), and the
azimuthal angle resolution~(\phires) are also checked with respect to the track
momentum for different thickness of the added CsI tube.
Figure~\ref{fig:p_resolution} shows \pres\ of \pp\ and \pim. We can see that
additional CsI mainly affects the \pp\ with momentum lower than 0.8~GeV/$c$,
but for \pim, 20~mm thick CsI worsens the \pres\ by less than 1~MeV/$c$ in the
full momentum range. The CsI effect on \tres\ and \phires\ for \pp\ and \pim\
is shown in Fig.~\ref{fig:t_resolution} and Fig.~\ref{fig:phi_resolution}.
Without any additional CsI, the \tres\ and \phires\ is below 10~milliradians for
most of the momentum range.  Additional CsI material has a greater impact for
the low momentum tracks. With the increase of CsI thickness, the impact of
adding the same additional thickness of CsI decreases gradually.  With the
thickness of additional CsI increasing, the tracking efficiency of low momentum
particles is low, thus it is more difficult to obtain accurate \pres, \tres\
and \phires\ (missing points in
Figs.~\ref{fig:p_resolution},~\ref{fig:t_resolution} and
\ref{fig:phi_resolution}).
        
        \begin{figure}[htbp]
        	\centering  %图片全局居中
        	\subfigbottomskip=2pt %两行子图之间的行间距
        	\subfigcapskip=-5pt %设置子图与子标题之间的距离
        	\subfigure[]{
        		\includegraphics[width=0.45\linewidth]{figs/single_ppi/single_p+_p_resolution.eps}}
        	\subfigure[]{
        		\includegraphics[width=0.45\linewidth]{figs/single_ppi/single_pi-_p_resolution.eps}}
        	\caption{Momentum resolution \pres~ as a function of momentum for: (a) protons, (b) pions generated within the range of $0.1 < \cos\theta < 0.2$ for adding a CsI layer with the thickness of 0, 5, 10, 15, and 20 mm.}
        	\label{fig:p_resolution}
        \end{figure}

        \begin{figure}[htbp]
        	\centering  %图片全局居中
        	\subfigbottomskip=2pt %两行子图之间的行间距
        	\subfigcapskip=-5pt %设置子图与子标题之间的距离
        	\subfigure[]{
        		\includegraphics[width=0.45\linewidth]{figs/single_ppi/single_p+_theta_resolution.eps}}
        	\subfigure[]{
        		\includegraphics[width=0.45\linewidth]{figs/single_ppi/single_pi-_theta_resolution.eps}}
        	\caption{Polar angle resolution, \tres, as a function of of momentum for: (a) protons, (b) pions generated within the range of $0.1 < \cos\theta < 0.2$ for adding a CsI layer with the thickness of 0, 5, 10, 15, and 20 mm. }
        	\label{fig:t_resolution}
        \end{figure}
        
        \begin{figure}[htbp]
        	\centering  %图片全局居中
        	\subfigbottomskip=2pt %两行子图之间的行间距
        	\subfigcapskip=-5pt %设置子图与子标题之间的距离
        	\subfigure[]{
        		\includegraphics[width=0.45\linewidth]{figs/single_ppi/single_p+_phi_resolution.eps}}
        	\subfigure[]{
        		\includegraphics[width=0.45\linewidth]{figs/single_ppi/single_pi-_phi_resolution.eps}}
        	\caption{Azimuth angle resolution, \phires, as a function of of momentum for: (a) protons, (b) pions generated within the range of $0.1 < \cos\theta < 0.2$ for adding a CsI layer with the thickness of 0, 5, 10, 15, and 20 mm. }
        	\label{fig:phi_resolution}
        \end{figure}
        
\subsection{Study of single antineutron sample}
        
Although not very reliable, we may still get a rough feeling about the
probability of the antineutron and CsI interactions by simulating a single
antineutron and adding CsI tube with different thickness in the detector. The
range of antineutron momentum is set to $[0,~1.2]$~GeV/$c$ while the polar
angle is the same as that of \pp\ and \pim. In total, $10^6$ events of \nbar\,
sample are simulated. We can get the fractions of \nbar\ annihilation and
\nbar-nuclei elastic scattering in the additional CsI. For \nbar\ annihilation
events the final position of the antineutron indicates where the annihilation
happened, it can be obtained from the MC truth information; for \nbar-nuclei
elastic scattering events, the difference in the momentum between entering and leaving 
the additional CsI indicates whether elastic scattering occurred. 
The threshold of momentum change is set to 0.01~GeV/$c$ although
the momentum change is strictly zero when no scattering occured. 
        
The above method of identifying the annihilation and elastic scattering events
is validated with the simulation of antineutron and beam pipe interactions. The
simulation shows that 4.7\% of events include an antineutron interaction in the CsI 
or beam pipe interaction. The 4.7\% total includes 3.7\% of annihilation 
and 1.0\% of elastic scattering. If the threshold of momentum is increased 
to 0.3~GeV/$c$, the results become 2.6\%, 1.8\%, and 0.8\%, respectively, 
which are consistent with the estimation of 1--2\% in Ref.~\cite{hypronProjectileFromJpsi}. 

Figure~\ref{fig:fraction_nbar} shows the fractions of \nbar\ annihilation and
\nbar-nuclei elastic scattering in the additional CsI tube with thickness of 0,
5, 10, 15, and 20~mm. We find that the fractions increase almost linearly as a
function of the thickness of the additional CsI tube, each additional 1~mm of
CsI results in 2.5\% of antineutron interaction, of which 1.2\% is
annihilation and 1.3\% is elastic scattering. The ratio of elastic scattering
and annihilation is different from that in the beam pipe. 
        
        \begin{figure}[htbp]
        	\centering  %图片全局居中
        	\subfigbottomskip=2pt %两行子图之间的行间距
        	\subfigcapskip=-5pt %设置子图与子标题之间的距离
        	\subfigure[]{
        		\includegraphics[width=0.45\linewidth]{figs/single_nbar/annihilate_ratio.eps}}
        	\subfigure[]{
        		\includegraphics[width=0.45\linewidth]{figs/single_nbar/scatter_ratio.eps}}
        	\caption{
Interaction probabilities for (a) annihilation, (b) elastic scattering between
anti-neutron and CsI as a function of momentum with the addition of a CsI layer with the
thickness of 0, 5, 10, 15, and 20 mm. }

        	\label{fig:fraction_nbar}
        \end{figure}

With a 10~mm thick CsI tube added, we find that about 24\% of the antineutrons
interact with it, of which 11\% are annihilation and 13\% are elastic
scattering. The chance for multiple interactions will be increased if more CsI
material is added and this may make the study of the antineutron interaction
more complicated. This should be considered in the design of the experiment. 

\subsection{\boldmath Study of the $J/\psi\to $\pp\pim\nbar~sample}

MC samples of $J/\psi\to $\pp\pim\nbar\ are simulated with additional CsI tube
of thickness 0, 5, 10, 15, and 20~mm and subject to event selection to tag the
antineutrons. Two charged tracks with opposite charge originating from the
interaction point are required in each event, after particle identification,
they are marked as \pp\ and \pim. The recoil mass of \pp\ and \pim\ is required
to be between 0.90 and 0.98~GeV/$c^2$.:
In total, $10^6$ events of $J/\psi\to
$\pp\pim\nbar\ sample are simulated. 
%
%After the above selection, the efficiency and the resolutions of recoiled antineutron are calculated and shown in Figs.~\ref{fig:ppinbar}(b) and~(d), respectively. 

The additional CsI has an obvious influence on the resolutions of the recoiling
antineutrons. After thicker CsI tube is added, the distribution of the
recoiling mass is significantly widened, thus the resolution of the recoiling
mass is reduced as shown in Fig.~\ref{fig:nbar_resolutions}(a).
Figure~\ref{fig:nbar_resolutions}(b, c, d) show \pres, \tres\ and \phires\ of
the tagged antineutrons without or with different thicknesses of additional CsI tube
added, they decrease with the increase of momentum.

\begin{figure}[htbp]
	\centering  %图片全局居中
	\subfigbottomskip=2pt %两行子图之间的行间距
	\subfigcapskip=-5pt %设置子图与子标题之间的距离
	\subfigure[]{
	    \includegraphics[width=0.45\linewidth]{figs/ppinbar/recoil_mass.eps}}
	\subfigure[]{
		\includegraphics[width=0.45\linewidth]{figs/ppinbar/p_resolution.eps}}
	\subfigure[]{
		\includegraphics[width=0.45\linewidth]{figs/ppinbar/theta_resolution.eps}}
	\subfigure[]{
		\includegraphics[width=0.45\linewidth]{figs/ppinbar/phi_resolution.eps}}
	\caption{
Differences between the reconstructed and true values (a) shows the resolution
of recoil mass of \pp and \pim as a function of the thickness of additional CsI
layer. (b) (c) (d) show respectively \pres, $\sigma_\theta$ and $\sigma_\phi$
(d) of the recoiling antineutron as a function of momentum for adding a CsI
layer with the thickness of 0, 5, 10, 15, and 20 ~mm.
}
	\label{fig:nbar_resolutions}
\end{figure}

With additional CsI of different thickness, the selection efficiencies of the
recoiling antineutron in different momentum ranges are shown in
Fig.~\ref{fig:ppinbar}(a). The efficiency decreases almost uniformly as the CsI
thickness increases and the low valley in the range of 0.8 to 1.0~GeV/$c$ 
arises due to the kinematics causing the corresponding tagged proton to have a 
low momentum resulting in a low efficiency.
When a 10~mm thick CsI tube is added, the total efficiency is 67.5\%, which is
reduced by 10.8\% compared to 78.3\% without additional CsI; when a 20~mm thick
CsI tube is added, the total efficiency and the reduction become 57.3\% and
21.0\%.
        
With the recoiling antineutron efficiency and the fraction of interactions
between antineutron and additional CsI, the fraction of the antineutrons that
can be observed is obtained by simply multiplying them. Although the efficiency
decreases, the antineutron interactions that can be selected still increases
almost monotonically, as shown in Fig.~\ref{fig:ppinbar}(b). 
However, the growth rate slows down as the CsI thickness increases.
        
Figure~\ref{fig:ppinbar}(c) shows the total efficiency of antineutron and CsI interactions 
and the resolutions at each of the various thicknesses of CsI tube.
We also studied the changes in total efficiency and resolution when using stricter
selection criteria. We have added the following conditions: the distance
between the vertex and the xy-plane is less than 0.5cm, while the distance from
the z-axis is less than 5cm, the fitted chi2 of the vertex is less than 30, and
the recoil mass is between 0.92 and 0.96~Gev/c. It can be seen that when the
thickness of CsI is greater than 15~mm, the total efficiency almost does not
increase, while the resolution still decreases linearly.

\begin{figure}[htbp]
	\centering  %图片全局居中
	\subfigbottomskip=2pt %两行子图之间的行间距
	\subfigcapskip=-5pt %设置子图与子标题之间的距离
	\subfigure[]{
		\includegraphics[width=0.45\linewidth]{figs/ppinbar/efficiency.eps}}
	\subfigure[]{
		\includegraphics[width=0.45\linewidth]{figs/ppinbar/total_efficiency.eps}}
	\subfigure[]{
		\includegraphics[width=0.45\linewidth]{figs/ppinbar/efficiency_and_resolutions.eps}}
		\subfigure[]{
		\includegraphics[width=0.45\linewidth]{figs/ppinbar/efficiency_and_resolutions_new.eps}}
	\caption{(a) (b) respectively show antineutron selection efficiencies and yields in different momentum ranges for adding a CsI layer with the thickness of 0, 5, 10, 15, and 20 ~mm. (c) The variation trends for antineutron yield, \pres\, \tres\ and \phires\ (d) as a function of the thickness of the additional CsI layer. For the left y-axis, the unit is percent. For the right y-axis, the unit for \pres\ is MeV/$c$ and the units for \tres\ and \phires\ are mrad; (d) shows the similar results to (c) after applying stricter select criteria. }
	\label{fig:ppinbar}
\end{figure}
        
        
\section{Conclusion and discussion}
\label{sec:conclusion}

From a MC simulation study of single proton, \pim\, and \nbar, as well as
$J/\psi\to p \pi^- \bar{n}$ samples, without and with an additional CsI tube of
various thickness, we find that the thicker CsI tube yields more antineutron-CsI 
interactions. However, the \pres, \tres\ and \phires\ of the antineutrons 
decrease and make the measurement of the interactions less precise as the 
thickness is increased. Optimization of the thickness of the
additional CsI tube is a balance between event rate and the resolutions of
the antineutrons as shown by Fig.~\ref{fig:ppinbar}(c). Therefore, we can draw the
following conclusion. Firstly, based on the specific research content of the
experiment, a reasonable thickness range can be obtained by referring to the
Figure~\ref{fig:ppinbar}. In order to ensure the accuracy of the
antineutron tagging, the thickness of CsI should not exceed 15~mm. Secondly,
the efficiency and resolution of tagged antineutrons vary significantly in
different momentum ranges. When studying antineutrons with high momentum (above
0.8~GeV/c), thicker CsI needs to be added to ensure the quantity of tagged
antineutrons. Although thicker CsI will results in more interactions recorded,
the events may suffer from multiple interactions which will complicate the analysis. 
The reconstruction and identification of the \nbar\ and CsI interaction events 
should be investigated further and it is out of the scope of
this work. Meanwhile, when selecting the thickness of CsI, the duration of the
experiment should be taken into account. Prolonged time can result in some
uncertainty, as the state of the detector changes over time.

In the BESIII experiment, with the ability of generating 10 billion \jpsi\ events
per year, we estimate that when the thickness of the additional CsI tube is
10~mm, 2.6 million events of interaction between antineutron and CsI tube can
be collected per year according to the total efficiency shown in Fig.~\ref{fig:ppinbar}(d). 
The most interesting prospect for this approach is the presence of significant numbers  
of antineutron above 500~MeV, which is unique in the world~\cite{hypronProjectileFromJpsi}, 
which may enable novel measurements.
% This unprecedented data sample will benefit the antineutron simulation very significantly.
    
% \chinese{[对比其他实验]}\red{There was no data on \nbar+CsI, just estimate the number of events at BESIII in one-year running (about 10 billion \jpsi~events) here?}
% For comparison, while adding \tbd mm thick CsI, the same measurement accuracy as the experiment \tbd can be obtained at \tbd GeV. At the same time, more antineutron events than this experiment can be selected by collecting data for \tbd years in BESIII.
    
There is room for improvement in the experimental design. With the increase of
$\cos\theta$, the efficiency and resolution of \pp, $\pi^-$, and antineutron
decrease significantly, because the particles need to pass through thicker CsI
material before entering the detector. To ensure that the particle at different
polar angle passes through the same amount of material, the thickness of
additional CsI tube can be different at different polar angle, for example, to
be $t_0\sin\theta$, where $t_0$ is the thickness at $\theta=90^\circ$.
    
The study presented in this article could be extended to other materials,
especially other kinds of crystals, such as NaI, PbWO$_4$, and BGO which are used 
as the materials of electromagnetic calorimeters or even thin layers of iron 
and lead which are used as the materials of hadronic calorimeters. 

%    \chinese{\red{[类似的方法可以研究在束流管外加入液氢和液氘靶后对于超子与核子相互作用测量的影响,我们尝试了在现在有限的空间内加入2cm靶后的情形。]}}
    
A similar method can be used to study the interaction between hyperons and
nucleons. Many hyperons can be produced and tagged from \jpsi\ decays as
proposed in Ref.~\cite{hypronProjectileFromJpsi}. In this case, the target must
be changed into liquid hydrogen or liquid deuterium. We also simulated the
efficiency and resolutions of \pp\ and \pim\ by adding a layer of liquid
hydrogen and liquid deuterium contained between two solid tubes. Limited by the
space inside the BESIII detector~\cite{bes3}, only about 20~mm of matter can be
added at most, which is far from enough for the study of hyperon and nucleon
interactions. Future experiments planning to perform light target material 
measurements would need a design with greater space to allow sufficient interactions.  
    
\section{Acknowledgments}

Thank Weimin Song and Liang Liu for their help with the software.
This work is supported in part by National Key Research and Development Program of China under Contract No.~2020YFA0406300, National Natural Science Foundation of China (NSFC) under contract No.~12275297.    
    
\begin{thebibliography}{40}

\bibitem{pdg}
R.~L.~Workman \textit{et al.} (Particle Data Group),
%``Review of Particle Physics,''
PTEP \textbf{2022}, 083C01 (2022)
doi:10.1093/ptep/ptac097
%712 citations counted in INSPIRE as of 01 Mar 2023

%\cite{CLEO:2008aum}
\bibitem{CLEO:2008aum}
S.~B.~Athar \textit{et al.} (CLEO Collaboration),
%``First Observation of the Decay D(s)+ ---\ensuremath{>} p anti-n,''
Phys. Rev. Lett. \textbf{100}, 181802 (2008).

%\cite{BESIII:2012imn}
\bibitem{BESIII:2012imn}
M.~Ablikim \textit{et al.} (BESIII Collaboration),
%``Measurement of $\chi_{cJ}$ decaying into $p\bar{n}\pi^{-}$ and $p\bar{n}\pi^{-}\pi^{0}$,''
Phys. Rev. D \textbf{86}, 052011 (2012).

%\cite{BESIII:2021tbq}
\bibitem{BESIII:2021tbq}
M.~Ablikim \textit{et al.} (BESIII Collaboration),
%``Oscillating features in the electromagnetic structure of the neutron,''
Nature Phys. \textbf{17}, no.11, 1200-1204 (2021).

\bibitem{bes3} M.~Ablikim {\it et al.} (BESIII Collaboration), Nucl.\ Instrum.\ Methods Phys. Res., Rect.\ A {\bf 614}, 245 (2010).

\bibitem{cleoc}
R.~A.~Briere \textit{et al.} (CLEO Collaboration),
%``CLEO-c and CESR-c: A New frontier of weak and strong interactions,''
CLNS-01-1742.
%59 citations counted in INSPIRE as of 03 Mar 2023

%\cite{Belle-II:2018jsg}
\bibitem{belle2}
E.~Kou \textit{et al.} (Belle-II Collaboration),
%``The Belle II Physics Book,''
PTEP \textbf{2019}, no.12, 123C01 (2019)
[erratum: PTEP \textbf{2020}, no.2, 029201 (2020)].

\bibitem{geant4} S.~Agostinelli {\it et al.}, (GEANT4 Collaboration), Nucl.\ Instrum.\ Meth.\ A {\bf 506}, 250 (2003). %``GEANT4: A Simulation toolkit,''

%\cite{Liu:2021rrx}
\bibitem{Liu:2021rrx}
L.~Liu, X.~Zhou and H.~Peng,
%``Development of a data-driven method to simulate the detector response of anti-neutron at BESIII,''
Nucl. Instrum. Meth. A \textbf{1033}, 166672 (2022).

\bibitem{E767} %\cite{BROOKHAVEN-HOUSTON-PENNSYLVANIASTATE-RICE:1987vhf}
T.~Armstrong \textit{et al.} [BROOKHAVEN-HOUSTON-PENNSYLVANIA STATE-RICE],
%``Measurement of Anti-neutron Proton Total and Annihilation Cross-sections From 100-{MeV}/c to 500-{MeV}/c,''
Phys. Rev. D \textbf{36}, 659-673 (1987).

\bibitem{OBELIX} 
M.~Agnello %, E.~Botta, T.~Bressani, D.~Calvo, A.~Feliciello, A.~Filippi, P.~Gianotti, F.~Iazzi, S.~Marcello and B.~Minetti, 
\textit{et al.},
%``The antineutron beam at OBELIX,''
Nucl. Instrum. Meth. A \textbf{399}, 11-26 (1997).

\bibitem{nbarPhysics} %For a review, see 
T.~Bressani and A.~Filippi,
%``Antineutron physics,''
Phys. Rept. \textbf{383}, 213-297 (2003).

\bibitem{hypronProjectileFromJpsi} 
C.~Z.~Yuan and M.~Karliner,
%``Cornucopia of Antineutrons and Hyperons from a Super J/\ensuremath{\psi} Factory for Next-Generation Nuclear and Particle Physics High-Precision Experiments,''
Phys. Rev. Lett. \textbf{127}, no.1, 012003 (2021).

\bibitem{bepc2} Q. Qin, L. Ma, J. Wang, and C. Zhang,  Conf. Proc. C {\bf 100523}, 2359 (2010), IPAC-2010-WEXMH01, http://accelconf.web.cern.ch/AccelConf/IPAC10/papers/wexmh01.pdf.

\bibitem{bes3physics} 
D.~M.~Asner {\it et al.}, Int.\ J.\ Mod.\ Phys.\ A {\bf 24}, S1 (2009).

\bibitem{bes3physicsFuture} 
M.~Ablikim \textit{et al.} (BESIII Collaboration),
%``Future Physics Programme of BESIII,''
Chin. Phys. C \textbf{44}, no.4, 040001 (2020).

\bibitem{boss}
 W.~Li {\it et al.}, Proc. Int. Conf. Comput. High Energy and Nucl. Phys. 225 (2006).

\bibitem{bes3hough}
J.~Zhang  {\it et al.}, Radiat. Detect. Technol. Methods {\bf 2}, 20 (2018).
 
\bibitem{kalFit}
 J. Wang {\it et al.}, Chin. Phys. C {\bf 33}, 870-879 (2009)

\end{thebibliography}

\end{document}

