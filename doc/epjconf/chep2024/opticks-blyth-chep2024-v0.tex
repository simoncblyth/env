%2024v0
\documentclass{webofc}
\usepackage[varg]{txfonts}
\newcommand{\comment}[1]{}
\usepackage{graphicx}
\usepackage{array}
\comment{

1. integrated analytic and triangulated
2. out-of-core multi-launch, billion photon test
3. interactive ray traced viz
4. jump to Philox 
5. 1st vs 3rd gen performance

}
%
\begin{document}
\title{Opticks : GPU ray traced optical photon simulation}
\author{\firstname{Simon} C. \lastname{Blyth}\inst{1}\fnsep\thanks{Corresponding author \email{simon.c.blyth@gmail.com}.}}
\institute{Institute of High Energy Physics, CAS, Beijing, China.}
\abstract{Opticks is an open source project that accelerates optical photon simulation 
by integrating NVIDIA GPU ray tracing, accessed via the NVIDIA OptiX API, with
Geant4 toolkit based simulations.
Optical photon simulation times of 10 seconds per 100 million photons
with the full JUNO geometry and optical model have been measured 
with a 3rd generation RTX GPU. The JUNO optical model incorporates 
optical processes inside PMTs including thin layer interference effects. 
The GPU geometry is auto-translated from the Geant4 geometry 
Optical physics processes of scattering, absorption, scintillator reemission 
and boundary processes are implemented in CUDA based on Geant4.  
Wavelength-dependent material and surface
properties as well as inverse cumulative distribution functions for reemission
are interleaved into GPU textures providing fast interpolated property lookup
or wavelength generation. 
In this work we describe performance measurements and new features developed 
to facilitate and optimize production usage of Opticks within the JUNO simulation framework. 
The features include geometry generalization enabling use of triangulated solids 
together with analytic geometry, out-of-core processing enabling simulation of 
events with more photons than can fit within available VRAM and also the adoption
of the Philox counter-based random number generator optimizing random number generation.   
}
\maketitle
%
\section{Introduction}%
\label{intro}
%
Opticks[1-8] enables Geant4[9-11] based simulations 
to benefit from high performance GPU ray tracing made accessible 
by NVIDIA\textregistered\ OptiX\texttrademark[12-16].
%
The Jiangmen Underground Neutrino Observatory (JUNO)\cite{juno} 
located in southern China is the world's largest liquid scintillator detector, 
with a 20 kton spherical volume of 35 m diameter. The large size and high photon yield, illustrated in Figure~\ref{problem}, 
makes optical photon simulation extremely computationally challenging for both processing time and memory resources. 
Opticks eliminates these bottlenecks by offloading optical simulation to the GPU. 
%
%
\begin{figure}
\centering
\includegraphics[width=\textwidth,clip]{env/graphics/ggeoview/jpmt-before-contact_half.png}
\caption{Cutaway OpenGL rendering of millions of simulated optical photons from a 200 GeV muon crossing the JUNO liquid scintillator. 
Each line corresponds to a single photon with line colors representing the polarization direction. 
Primary particles are simulated by Geant4, "gensteps" are uploaded to the GPU and photons are generated, propagated
and visualized all on the GPU. 
%Representations of some of the many thousands of photomultiplier tubes that instrument the liquid scintillator are visible.
%The acrylic vessel that contains the liquid scintillator is not shown.
}
\label{problem}
\vspace{-5mm}
\end{figure}%
%
Sequential simulation of large numbers of 
optical photons has extreme computational and memory costs. 
Opticks enables drastically improved optical photon simulation performance.
%
Although developed for simulation of the JUNO detector, Opticks
supports use with other detector geometries. 
Any optical photon limited simulation can benefit from Opticks.

Opticks was presented to the five prior CHEP conferences, with each contribution
covering different aspects of its development. The 2023 contribution\cite{chep2023} 
covered the almost complete re-implementation of Opticks required to adopt the entirely new NVIDIA OptiX 7+ API
with particular emphasis on the various geometry models and conversions between them.  
%
The 2021 contribution\cite{chep2021} covered initial stages of the transition to the NVIDIA OptiX 7+ API and 
the integration of Opticks with detector simulation frameworks. 
The 2019 plenary presentation and proceedings\cite{chep2019} focused on RTX\cite{rtx} performance measurements.
The earlier contributions[7,8] covered the first 
implementations of geometry translation and the CUDA port of photon generation and optical physics.

These proceedings describe new features developed to facilitate and optimize production usage of Opticks 
within the JUNO simulation framework. The features include generalizations enabling 
integrated use of analytic and triangulated geometry representations and the automated handling of 
events with more photons than can fit within VRAM, interactive ray traced visualization and also the adoption
of the Philox counter-based random number generator optimizing random number generation.
In addition a simulation performance comparison between GPUs from the 1st and 3rd RTX generations
is provided. 
%
%
\subsection{Importance of optical photon simulation}%
%
%
Suppression of cosmic muon induced backgrounds with veto selections are crucial for neutrino
detectors such as JUNO\cite{juno}, necessitating production of large simulated samples of muon events. 
However, a muon of typical energy 200 GeV crossing the JUNO scintillator can yield tens of millions of 
optical photons, which are found with Geant4 simulations to consume more than 99\% of CPU time
and impose severe memory constraints.
%
As optical photons in neutrino detectors can be considered to be produced
only by scintillation and Cherenkov processes and yield only hits
on photomultiplier tubes, it is straightforward to combine an external optical photon simulation 
with a Geant4 simulation of all other particles.
%
\subsection{GPU ray tracing}%
%
GPUs evolved to perform rasterized rendering, optimizing throughput\cite{throughput} rather than minimizing latency.
GPUs are suited to problems with millions of independent low resource parallel tasks allowing thousands of threads 
to be in flight simultaneously.
Optical simulation is well matched to these requirements with abundant parallelism 
from huge numbers of photons and low register usage from simplicity of the physics.

The most computationally demanding aspect of photon propagation
is the calculation of intersection positions of rays representing photons with the detector geometry.
This ray tracing limitation of simulation is shared with the synthesis of realistic images in computer graphics. 
NVIDIA RTX\cite{rtx} GPUs include hardware RT cores dedicated to ray geometry intersection. 
The first generation of RTX GPUs was introduced in 2018, each of the subsequent RTX generations have 
approximately doubled ray tracing performance.  
%
%
\subsection{NVIDIA\textregistered\ OptiX\texttrademark\ ray tracing engine}
%
OptiX makes GPU ray tracing accessible with a single ray programming model.
Ray tracing pipelines are constructed combining code for acceleration structure traversal, 
with user code for ray generation, intersection and closest hit handling.
Spatial index acceleration structures (AS) provide accelerated ray geometry intersection. 
%
OptiX allows user-defined primitives bounded by specified axis-aligned bounding
to be implemented with CUDA intersection functions. Alternatively geometry can be 
defined with a set of triangles that use built-in triangle intersection.  
%
%optix7
In August 2019 NVIDIA introduced the OptiX 7 API\cite{optix7},
that together with the Vulkan and DirectX ray tracing extensions provides access 
to the same NVIDIA ray tracing technology including AS construction and RTX hardware access. 
The latest OptiX 8.1 release in October 2024 has a very similar API to OptiX 7.   
%
%
\begin{figure}[t]
\centering
%                                    left lower right upper
\includegraphics[width=\textwidth,trim={0 4cm 0 4cm},clip]{env/Documents/Geant4OpticksWorkflow7/Geant4OpticksWorkflow7_005.png}
\caption{Hybrid Geant4 + Opticks workflow : {\tt G4CXOpticks} translates Geant4 geometry to GPU appropriate form. 
{\tt U4} collects "gensteps" enabling GPU generation of scintillation and Cerenkov photons.
}
% including the number of photons to generate, the line segment along 
%which to generate them and all other parameters used by the photon generatation loop.
%}
\label{workflow} 
\vspace{-5mm}
\end{figure}
%
\section{Hybrid simulation workflow}%
\label{secworkflow}%
%
Figure~\ref{workflow} summarizes the hybrid workflow. 
At initialization the Geant4 top volume is passed to Opticks
which translates the geometry and uploads it to the GPU as described in section \ref{secgeom}.
%
Geant4 models scintillation and Cerenkov processes with the classes {\tt G4Scintillation} and {\tt G4Cerenkov}. 
At each simulated step of applicable particles the classes calculate a number of optical photons 
to generate depending on particle and material properties, followed by a loop that generates the optical photons. 
With the hybrid workflow these classes are modified, replacing the generation loop with the collection of 
generation parameters termed "gensteps" that include the number of photons and the line segment along which to generate them and all
other parameters needed to reproduce photon generation on the GPU. Relocating photon generation to the
GPU avoids CPU memory allocation. Only non-culled photon hits needed for the next stage electronics 
simulation require CPU memory allocation. 

GPU optical photon generation and propagation are implemented in simple headers that are included
into the OptiX ray tracing pipeline that runs in parallel. 
For each step of the propagation, rays representing photons are intersected
with the geometry using simple header intersect functions that are also included into the ray tracing pipeline.
The intersected boundary together with the photon wavelength are used to do interpolated texture lookups of
material properties such as absorption and scattering lengths.
Converting these lengths to distances using pseudorandom numbers and 
the known exponential distributions allows a comparison of absorption and scattering distances 
with geometrical boundary distance to assign photon histories. 
%Earlier proceedings\cite{chep2016} detail efficient use of the cuRAND\cite{curandURL} pseudorandom generator.
%
%
\begin{figure}
\centering
\includegraphics[width=\textwidth,clip]{env/presentation/GEOM/V1J009/CSGOptiXRdrTest/cxr_min__eye_-10,0,0__zoom_0p5__tmin_0p1__sChimneyAcrylic_increased_TMAX.jpg}
\caption{Render of the PMTs of the JUNO detector comprising 1920x1080 ray traced pixels created with a CUDA launch under 16 ms 
using a single NVIDIA TITAN RTX GPU and NVIDIA OptiX 7.
\label{j1808}}
\vspace{-5mm}
\end{figure}
%
\section{Detector geometry} 
\label{secgeom}%
% FIRST : HIGH LEVEL DESCRIPTION OF THE VARIOUS GEOMETRY MODELS AND TRANSLATIONS BETWEEN THEM
% AND THEN GET INTO THE DETAILS IN THE SUBSECTIONS
At initialization Opticks translates the geometry through the below sequence of models:
%
\begin{enumerate}
\item Geant4 : deep hierarchy of structural volumes and trees of {\tt G4VSolid} CSG nodes
\item Opticks {\tt stree} : intermediate n-ary trees of volumes and CSG nodes
\item Opticks {\tt CSGFoundry} : GPU model with {\tt CSGSolid}, {\tt CSGPrim} and {\tt CSGNode} 
\item NVIDIA OptiX 7+ : Instance and Geometry Acceleration Structures (IAS, GAS)
\end{enumerate}  
%
The {\tt stree} intermediate model is created from the Geant4 model by {\tt U4Tree.h},
which traverses the Geant4 volume tree converting materials, surfaces, solids, volumes and sensors.
Triangle and face data for all solids are collected by {\tt U4Mesh.h} using 
the Geant4 tesselation accessed from {\tt G4Polyhedron}. 
Subsequently the {\tt CSGFoundry} model is created from the intermediate model and uploaded to GPU.   
Most of the geometry information in the intermediate model is managed by a single header only struct, {\tt NPFold.h}, 
which provides an in memory directory tree of arrays using a recursive folders of folders of arrays data structure.
The complete structural n-ary tree of volumes and n-ary CSG trees for each {\tt G4VSolid} 
are serialized into arrays using first child and sibling references.

Both the {\tt stree} and {\tt CSGFoundry} geometry models are independent of Geant4 and can be persisted into directories of NumPy\cite{numpy} binary files. 
Fast binary file loading and uploading to GPU allows optical simulation and visualization
executables to initialize full detector geometries in less than a second.  
The prior proceedings\cite{chep2023} provide a detailed description of the geometry models and conversion between them,
including the crucial geometry factorization into repeated groups of volumes and a remainder of 
other insufficiently repeated volumes that is done within the intermediate model.
Also the representation of solid shapes using n-ary trees of {\tt sn.h} CSG nodes within  
the {\tt stree} intermediate model are detailed in the prior proceedings. 
%
%
\comment{
\subsection{Structural volumes and geometry factorization}%
%
The Geant4 model of the JUNO geometry contains almost 400,000 structural volumes organized in a deep containment tree
heirarchy of volumes with associated transforms.
The Opticks approach to modelling these volumes is based upon the observation that many of the
volumes are repeated in groups, corresponding for example to the small number of volumes that represent each type of PMT.
Hence an efficient representation must make full use of geometry instancing to avoid duplication of information 
on the GPU by storing repeated elements only once together with 4x4 transforms
that specify the locations and orientations of each instance.

The factorization of the volumes into repeated groups of volumes and a remainder of 
other insufficiently repeated volumes is done within the intermediate model by the 
{\tt stree::factorize} method which uses sub-tree digests that represent the geometry and transforms 
beneath every node of the geometry tree. 
The outcome of the factorization is a repeat index on every structural node allowing 
the instance transforms of each repeat to be collected. 

Each of the factors from the intermediate model become 
compound {\tt CSGSolid} within the {\tt CSGFoundry} model.  
For the JUNO geometry the factorization yields ten compound solids including four 
different types of PMT assemblies, some of which are repeated many thousands of times. 
The instanced {\tt CSGSolid} typically contain a few {\tt CSGPrim}, for example PMT masks and Pyrex and Vacuum volumes of the PMT,
and the remainder solid contains several thousand {\tt CSGPrim}. 
The {\tt CSGPrim} refer to sequences of {\tt CSGNode} which are one-to-one related to the {\tt sn.h} CSG nodes
of the intermediate model. These CSG nodes typically correspond to constituent {\tt G4VSolid} such as {\tt G4Ellipsoid} and {\tt G4Tubs}. 
}%endcomment
%
\comment{
\subsection{CSGFoundry geometry model}
%
Table~\ref{tabcsgfoundry} summarizes the role and associations of the vector members 
of the {\tt CSGFoundry} struct which are serialized and uploaded to the GPU 
forming the inputs to the creation of acceleration structures and also the CSG node 
parameters used by the CSG intersection functions. 
%
The {\tt CSGFoundry} model is designed to facilitate creation 
of OptiX acceleration structures by the {\tt CSGOptiX} package. 
Geometry acceleration structures (GAS) are created from each of the 
compound {\tt CSGSolid} objects and a single instance acceleration structure (IAS)
for the entire geometry is created from the instance transform vector. 
%
%
%
\begin{center}
\begin{table}
\begin{tabular}{ |m{22mm}|m{80mm}|m{16mm}| } 
 \hline
 struct, member        & Associations and role                                                          & Geant4 Equivalent   \\
\hline\hline 
 {\tt qat4} inst       & instance transform, references {\tt CSGSolid}                                  &  None               \\
\hline 
 {\tt qat4} tran       & CSG transform, referenced from {\tt CSGSolid}                                  &  None               \\
\hline 
 {\tt CSGSolid} solid  & references sequence of {\tt CSGPrim}                                           & Group of volumes      \\ 
\hline 
 {\tt CSGPrim} prim    & references sequence of {\tt CSGNode}, bounding box                             & root {\tt G4VSolid} \\
\hline 
 {\tt CSGNode} node    & references CSG transform, node parameters, typecode                            & constituent {\tt G4VSolid} \\
 \hline
\end{tabular}
\caption{\label{tabcsgfoundry}Principal std::vector members of {\tt CSGFoundry} struct with Geant4 equivalent. }
\end{table}
\vspace{-8mm}
\end{center}%
}%endcomment
%
\comment{
\subsection{Solid shapes}%
%
The {\tt stree} intermediate model carries solid shape information
within an n-ary tree of {\tt sn.h} CSG nodes. 
%
%primitives
The Opticks CSG package implements ray primitive shape intersection in simple headers that 
are CUDA compatible but can also be used and debugged on the CPU.
These functions use implicit equations for the primitives together with 
the parametric ray equation, to yield a polynomial in $t$, the distance along the ray from its origin position. 
Roots and derivatives yield intersections and surface normals.
%
% combinations
Arbitrarily complex solids are described using constructive solid geometry (CSG) modelling, 
which builds shapes from the combination of primitive constituents by boolean set operations
and is represented with a binary tree data structure.
%
% serialization 
Each primitive or operator node is serialized into an array of up to 16 four byte elements. 
These elements include float parameters of the primitives and integer index references 
into a separate transform array. 
A complete binary tree serialization with array indices matching level order tree indices
and zeros at missing nodes is used for the serialization of the CSG trees. This simple 
serialization allows tree navigation directly from bitwise manipulations of the serialized array index.
Complete binary tree serialization is simple and effective for small trees but very inefficient 
with the large trees that result from complex shapes with many constituent primitives. 

The Opticks CSG intersection implementation arranged into three levels : 
tree, node and leaf. This allows the node level to support compound 
multi-leaf shapes without resorting to recursion in the intersect function, which is disallowed in OptiX pipelines. 
As multi-leaf nodes can be used within binary CSG trees it becomes possible for complex solids to 
be represented by binary trees of greatly reduced depth avoiding deep tree performance issues.
The multi-leaf nodes are similar to {\tt G4MultiUnion}, but are restricted to leaf constituents.   
%Various types multi-leaf nodes are under development including dis-contiguous, contiguous and overlap nodes.
Communicating the intent of the multi-leaf nodes allows use of better suited intersect algorithms.
For example, a common cause of deep CSG trees is the subtraction of a union of many "holes" 
from a base shape. Typically the holes do not overlap making it possible to 
use a simple low resource dis-contiguous intersect function providing more efficient 
intersection thanks to both reduced tree height and the better communication of the 
dis-contiguous nature of the holes. 
}%endcomment
%
%
\subsection{Integrated Analytic and Triangulated geometry}
% why use an approx geom
Use of analytic CSG geometry typically allows float precision intersection 
positions to very closely match the double precision intersection positions 
that Geant4 provides. 
Intersection of a ray with a torus requires solution of a quartic equation
with coefficients of greatly varying magnitude that result in unacceptably 
poor numerical precision for some rays when using float precision. 
Replacing the analytic torus geometry with an approximate
tesselated geometry is found to yield more robust intersect positions
and avoids performance reductions from resorting to double precision.  

% how used at high level
During geometry factorization the {\tt stree::collectGlobalNodes} method
assigns non-instanced structural nodes as analytic or
triangulated according to a user provided list of solid names to use 
a tesselated representation. This assignment is communicated via the {\tt CSGSolid}     
to the geometry acceleration structure creation of {\tt SBT::createGAS}. 
%
% stree::findForceTriangulateLVID
% stree::collectGlobalNodes into rem and tri vectors of snode
% CSGSolid.h intent R:rem F:fac T:tri 
% SBT::createGAS 
%
%
%
\section{Visualization}%
%
Opticks provides several OpenGL based visualization executables 
implemented using the lightweight Graphics Library Framework GLFW,
which manages OpenGL windows as well as keyboard and mouse inputs. 
Interoperation between OpenGL and CUDA used by {\tt SGLFW\_CUDA.h}
allows CUDA/OptiX ray trace render kernels to effectively write to OpenGL 
pixel buffer objects on the GPU which are subsequently accessed from the 
rasterization pipeline via texture samplers. This approach together with
high performance OptiX ray tracing enables smoothly interactive rendering of 
exactly the same detector geometry as that used by the simulation. 
Navigation in 3D via keys and mouse together 
with viewpoint bookmarking have proved useful for finding geometry issues.      
Initialization time of visualization of the full JUNO geometry 
is minimized when using the CSGFoundry geometry as the binary NumPy arrays defining 
the geometry can be loaded and uploaded within a few seconds.  

The Opticks render kernel computes {\tt uchar4} values for each pixel in the image 
plane with RGB color values based on the surface normal direction at geometry intersects. 
The fourth component of the pixel values is set to the z-depth in the eye frame 
mapped appropriately for the effective perspective or orthographic projection matrix. 
This depth component is used from the OpenGL shader rasterization pipeline to set the 
so called fragment depth of the ray traced pixel. This allows compositing of OptiX ray 
trace rendered pixels together with OpenGL rasterized fragments for example 
allowing representations of photon positions or gensteps to be 
drawn together with the ray traced geometry. 
 
% simtrace 

\section{Optical physics}%
%
Opticks optical photon simulation is implemented in the QUDARap package, 
that depends only on the SysRap base package and CUDA, with no dependency on OptiX. 
The prior proceedings\cite{chep2023} provide further details. 
%
\subsection{Optimizing random number generation}
%
The curand library provides several random number generators
including the default {\tt XORWOW} generator which has been used within Opticks.  
The curand XORWOW implementation requires resource intensive state initialization.
This requirement motivated use of install time creation of {\tt curandStateXORWOW} 
struct for all photon "slots" which are persisted to binary files. 
These state files are loaded and uploaded to device at initialization 
of the simulation. This allows every simulation kernel thread 
to use the XORWOW generator with its prepared state without having the expense of 
initializing the state. However this approach has the inconvenience of 
limiting the maximum number of photons that can be simulated to the number
of persisted states and also the initialization time to load and upload the 
states becomes significant when supporting simulation of hundreds of millions 
of photons. 

The {\tt Philox4\_32\_10} random number generator is also provided by the curand library. 
Philox is a counter based random number generator which may be implemented to 
use only integer counters for internal state. 
The statistical quality of Philox generated uniform random numbers are comparable to those from XORWOW\cite{curandRNGTest} 
Adopting the Philox random number generator within Opticks allows direct initialization within
the simulation kernels avoiding the use of state files and hence removing the 
limitation on the number of photons that can be simulated and greatly reducing 
initialization time. 
%
\subsection{Out-of-core optical photon simulation}
%
Simulation of more photons than can fit within VRAM is implemented
using multiple kernel launches invoked from {\tt QSim::simulate}. 
Following genstep collection index range genstep slices are chosen 
such that the number of photons within each slice is less than 
the configured maximum number of CUDA thread "slots" within a single launch.
The default maximum number of slots is determined based upon the total
global memory of the GPU. Result arrays such as photons and hits are gathered
from device to host into separate {\tt NPFold} instances for each kernel launch 
which are subsequently concatenated. To check this functionality 
one billion photon events were simulated with an NVIDIA RTX 5000 Ada 
generation GPU with VRAM of 32GB in four kernel launches 
in a total time under 100 seconds. 
% SEvt::gather_components  
% QSim::simulate
%
\section{Simulation performance}

Using artificial "torch" gensteps with increasing numbers of photons in a sequence of 
purely optical fabricated events provides a convenient way to scan performance as 
a function of photon count. The purely optical events 
avoids dependency on Geant4 allowing initialization times of a few seconds only rather than the several minutes for Geant4 
voxelization and physics setup of the JUNO geometry.  

\begin{figure}
\centering
%                                    left lower right upper
\includegraphics[width=\textwidth,trim={0 0 0 4cm},clip]{env/presentation/GEOM/J_2024aug27/CSGOptiXSMTest/ALL1_sreport/figs/sreport_ab/mpcap/AB_Substamp_ALL_Etime_vs_Photon_rtx_gen1_gen3.png}
\caption{%
Optical photon simulation times for a sequence of purely optical events with numbers of photons 
in the range from one million to one hundred million. The blue and orange points are measurements from 
two Dell workstations with NVIDIA Titan RTX (1st gen.) and NVIDIA RTX 5000 Ada (3rd gen.) respectively.
For both sets of measurements the simulation time is found to scale linearly with photon counts. The 
performance of the 3rd generation RTX GPU is found to be approximately a factor of four.   
}
\label{scan}
\vspace{-5mm}
\end{figure}%
%
%
\section{Summary}
%
Opticks enables Geant4-based optical photon simulations to benefit from 
state-of-the-art NVIDIA GPU ray tracing, made accessible via the NVIDIA OptiX 7+ API,
allowing memory and time processing bottlenecks to be eliminated. 
A full re-implementation of Opticks for the OptiX 7+ API is complete. 
Opticks now features many shared CPU/GPU headers and a minimal intermediate geometry model 
that has enabled drastic code reduction and simplification. 
The many small headers design together with mocking of a CUDA texture and 
random number functions enables fine grained testing of almost all functionality 
on both GPU and CPU.

Several groups from various experiments and the Geant4 Collaboration are evaluating Opticks
and an example of Opticks usage is now included within the Geant4 distribution. 
%
%
\newpage
\section*{Acknowledgements}
%
The JUNO collaboration is acknowledged for the use of detector 
geometries and simulation software. Dr. Tao Lin is acknowledged 
for his JUNO software assistance over many years. NVIDIA is acknowledged for extensive 
support for the OptiX 7+ API transition.
%
This work is supported by National Natural Science Foundation of China (NSFC)
under grant No. 12275293.
%
\begin{thebibliography}{}
%
%1
\bibitem{opticksURL}
Opticks Repository, {\tt https://bitbucket.org/simoncblyth/opticks/}
%2
\bibitem{opticksRefs}
Opticks References, {\tt https://simoncblyth.bitbucket.io}
%3
\bibitem{opticksGroup}
Opticks Group, {\tt https://groups.io/g/opticks}
%4 
\bibitem{chep2023}
S. Blyth, EPJ Web Conf. {\bf 295}, 11014 (2024) \\
{\tt https://doi.org/10.1051/epjconf/202429511014}
%5
\bibitem{chep2021}
S. Blyth, EPJ Web Conf. {\bf 251}, 03009 (2021) \\
{\tt https://doi.org/10.1051/epjconf/202125103009}
%6
\bibitem{chep2019}
S. Blyth, EPJ Web Conf. {\bf 245}, 11003 (2020) \\
{\tt https://doi.org/10.1051/epjconf/202024511003}
%7
\bibitem{chep2018}
S. Blyth, EPJ Web Conf. {\bf 214}, 02027 (2019) \\
{\tt https://doi.org/10.1051/epjconf/201921402027}
%8 
\bibitem{chep2016}
Blyth Simon C 2017 J. Phys.: Conf. Ser. {\bf 898} 042001 \\
{\tt https://doi.org/10.1088/1742-6596/898/4/042001}
%
%9
\bibitem{g4A}
S. Agostinelli, J. Allison, K. Amako, J. Apostolakis, H. Araujo, P. Arce et al., Nucl. Instrum. Methods. Phys. Res. A {\bf 506}, 250 (2003)
%10
\bibitem{g4B}
J. Allison, K. Amako, J. Apostolakis, H. Araujo, P. Dubois, M. Asai et al., IEEE Trans Nucl Sci, {\bf 53}, 270 (2006)
%11
\bibitem{g4C}
J. Allison, K. Amako, J. Apostolakis, P. Arce, M. Asai, T. Aso et al., Nucl. Instrum. Methods. Phys. Res. A {\bf 835}, 186 (2016)
%
%
%12
\bibitem{optixPaper}
OptiX: a general purpose ray tracing engine \\
S. Parker, J. Bigler, A. Dietrich, H. Friedrich, J. Hoberock et al., ACM Trans. Graph.: Conf. Series {\bf 29}, 66 (2010)
%13
\bibitem{optixSite}
OptiX introduction, {\tt https://developer.nvidia.com/optix}
%14
\bibitem{optixDocs}
OptiX API, {\tt https://raytracing-docs.nvidia.com/optix8/index.html}
%15
\bibitem{optix7}
OptiX 7, {\tt https://developer.nvidia.com/blog/how-to-get-started-with-optix-7/}
%16
\bibitem{rtx}
NVIDIA RTX\texttrademark\, Platform, {\tt https://developer.nvidia.com/rtx}
%17
\bibitem{juno}
Neutrino physics with JUNO \\
F. An et al., J. Phys. G. {\bf 43}, 030401 (2016) 
%
\bibitem{jpom}
A new optical model for photomultiplier tube\\
Y. Wang, G. Cao, L. Wen, Y. Wang, Eur. Phys. J. C 82(4), 329 (2022).\\
{\tt https://doi.org/10.1140/epjc/s10052-022-10288-y}
%
\bibitem{custom4}
Custom4 repository, {\tt https://github.com/simoncblyth/customgeant4}
%
\bibitem{throughput}
Understanding Throughput Oriented Architectures \\
M. Garland, D.B. Kirk, Commun. ACM {\bf 53}(11), 58 (2010) 
%18
\bibitem{curandURL}
cuRAND, {\tt http://docs.nvidia.com/cuda/curand/index.html}
%
\bibitem{curandRNGTest}
cuRAND generator tests, {\tt https://docs.nvidia.com/cuda/curand/testing.html}
%19
\bibitem{numpy}
The NumPy array: a structure for efficient numerical computation \\
S. Van der Walt, S. Colbert, G. Varoquaux, Comput. Sci. Eng. {\bf 13}, 22 (2011)
%20
%\bibitem{thrust}
%Chapter 26 - Thrust: A Productivity-Oriented Library for CUDA \\
%N. Bell, J. Hoberock, GPU Computing Gems Jade Edition, (2012), pp 359-371
%
\end{thebibliography}
%
\end{document}
