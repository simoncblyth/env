%2024v1
\documentclass{webofc}
\usepackage[varg]{txfonts}
\newcommand{\comment}[1]{}
\usepackage{graphicx}
\usepackage{array}
\comment{


1. integrated analytic and triangulated "tri-ana generalization"
2. out-of-core multi-launch, billion photon test, 
   automated handling of events with more photons than can fit within VRAM
3. interactive ray traced visualization
4. adoption of the Philox counter-based random number generator 
5. [fig] simulation performance comparison between GPUs from the 1st and 3rd RTX generations
 


    1. Introduction
       1.1 Importance of optical photon simulation 
       1.2 GPU ray tracing 
       1.3 NVIDIA OptiX ray tracing engine
       
    2. Hybrid simulation workflow

    3. Detector geometry

       3.1 Integrated Analytic and Triangulated geometry

    4. Visualization

    5. Optical physics

       5.1 Optimizing random number generation
       5.2 Out-of-core optical photon simulation

    6. Simulation performance

    7. Summary

}


%
\begin{document}
\title{Opticks : GPU ray traced optical photon simulation}
\author{\firstname{Simon} C. \lastname{Blyth}\inst{1}\fnsep\thanks{Corresponding author \email{simon.c.blyth@gmail.com}.}}
\institute{Institute of High Energy Physics, CAS, Beijing, China.}
\abstract{\input{opticks-blyth-chep2024-v1-abstract.tex}}
\maketitle
%
\section{Introduction}%
\label{intro}
%
Opticks[1-6] enables Geant4~\cite{g4} based simulations 
to benefit from high performance GPU ray tracing made accessible 
by NVIDIA\textregistered\ OptiX\texttrademark~\cite{optix}.
%
The Jiangmen Underground Neutrino Observatory (JUNO)~\cite{juno} 
located in southern China is the world's largest liquid scintillator detector, 
with a 20 kton spherical volume of 35 m diameter. The large size and high photon yield, illustrated in Figure~\ref{problem}, 
makes optical photon simulation extremely computationally challenging in processing time and memory resources. 
Opticks eliminates these bottlenecks by offloading optical simulation to the GPU which
enables drastically improved optical photon simulation performance.
%
\begin{figure}
\centering
\includegraphics[width=\textwidth,clip]{env/graphics/ggeoview/jpmt-before-contact_half.png}
\caption{
This figure is reproduced from~\cite{chep2019}. Cutaway OpenGL rendering of millions of simulated optical photons from a 200 GeV muon crossing the JUNO liquid scintillator. 
Each line corresponds to a single photon with line colors representing the polarization direction. 
Primary particles are simulated by Geant4, "gensteps" are uploaded to the GPU and photons are generated, propagated
and visualized all on the GPU. 
}
\label{problem}
\vspace{-5mm}
\end{figure}%
%
%Sequential simulation of large numbers of 
%optical photons has extreme computational and memory costs. 
%
Although developed for simulation of the JUNO detector, Opticks
supports use with other detector geometries. 
Any optical photon limited simulation can benefit from Opticks.

Opticks was presented to the five prior CHEP conferences, with each contribution
covering different aspects of its development. 
The first 2016 contribution~\cite{chep2016} details the initial approach of using a mostly tesselated 
geometry with manual analytic PMTs, low resource usage of the curand random number 
generator within OptiX kernels as well as the CUDA port of photon generation and optical physics. 
The subsequent 2018 contribution~\cite{chep2018} highlighted development of a low resource CSG intersection algorithm. 
The 2019 plenary presentation and proceedings~\cite{chep2019} focused on RTX~\cite{rtx} performance measurements, 
and the development of automated geometry translation.  
The 2021 contribution~\cite{chep2021} covered initial stages of the transition to the NVIDIA OptiX 7+ API 
and the integration of Opticks with detector simulation frameworks. The 2023 contribution~\cite{chep2023}
covered the almost complete re-implementation of Opticks required to adopt the entirely new NVIDIA OptiX 7+ API
with particular emphasis on the various geometry models and conversions between them.  
%

These proceedings describe new features developed to facilitate and optimize production usage of Opticks 
within the JUNO simulation framework. The features include generalizations enabling 
integrated use of analytic and triangulated geometry representations and the automated handling of 
events with more photons than can fit within VRAM, interactive ray traced visualization and also the adoption
of the Philox counter-based random number generator optimizing random number generation.
In addition a simulation performance comparison between GPUs from the 1st and 3rd RTX generations
is provided. 
%
%
\comment{
\subsection{Importance of optical photon simulation}%
%
%
Suppression of cosmic muon induced backgrounds with veto selections are crucial for neutrino
detectors such as JUNO~\cite{juno}, necessitating production of large simulated samples of muon events. 
However, a muon of typical energy 200 GeV crossing the JUNO scintillator can yield tens of millions of 
optical photons, which are found with Geant4 simulations to consume more than 99\% of CPU time
and impose severe memory constraints.
%
As optical photons in neutrino detectors can be considered to be produced
only by scintillation and Cherenkov processes and yield only hits
on photomultiplier tubes, it is straightforward to combine an external optical photon simulation 
with a Geant4 simulation of all other particles.
}%endcomment
%
\subsection{GPU ray tracing}%
%
GPUs evolved to perform rasterized rendering, optimizing throughput~\cite{throughput} rather than minimizing latency.
GPUs are suited to problems with millions of independent low resource parallel tasks allowing thousands of threads 
to be in flight simultaneously.
Optical simulation is well matched to these requirements with abundant parallelism 
from huge numbers of photons and low register usage from simplicity of the physics.

The most computationally demanding aspect of photon propagation
is the calculation of intersection positions of rays representing photons with the detector geometry.
This ray tracing limitation of simulation is shared with the synthesis of realistic images in computer graphics. 
NVIDIA RTX~\cite{rtx} GPUs include hardware RT cores dedicated to ray geometry intersection. 
The first generation of RTX GPUs was introduced in 2018, each of the subsequent RTX generations have 
approximately doubled ray tracing performance with the 4th generation introduced in 2025 providing approximately 8 times faster ray tracing. 
%
%
\subsection{NVIDIA\textregistered\ OptiX\texttrademark\ ray tracing engine}
%
OptiX~\cite{optix} makes GPU ray tracing accessible with a single ray programming model.
Ray tracing pipelines are constructed combining code for acceleration structure traversal, 
with user code for ray generation, intersection and closest hit handling.
Spatial index acceleration structures (AS) provide accelerated ray geometry intersection. 
%
OptiX allows user-defined primitives bounded by specified axis-aligned bounding
to be implemented with CUDA intersection functions. Alternatively geometry can be 
defined with a set of triangles that use built-in triangle intersection.  
%
%optix7
In August 2019 NVIDIA introduced the OptiX 7 API,
that together with the Vulkan and DirectX ray tracing extensions provides access 
to the same NVIDIA ray tracing technology including AS construction and RTX hardware access. 
The latest OptiX 9.0.0 release from January 2025 has a very similar API to OptiX 7.   
%
%
\begin{figure}[t]
\centering
%                                    left lower right upper
\includegraphics[width=\textwidth,trim={0 4cm 0 4cm},clip]{env/Documents/Geant4OpticksWorkflow7/Geant4OpticksWorkflow7_005.png}
\caption{This figure is reproduced from~\cite{chep2023}, which provides further details. Illustration of the
hybrid Geant4 + Opticks workflow : {\tt G4CXOpticks} translates Geant4 geometry to GPU appropriate form. 
{\tt U4} collects "gensteps" enabling GPU generation of scintillation and Cerenkov photons. 
}
\label{workflow} 
\vspace{-5mm}
\end{figure}
%
\section{Hybrid simulation workflow}%
\label{secworkflow}%
%
Figure~\ref{workflow} summarizes the hybrid workflow. 
At initialization the Geant4 top volume is passed to Opticks
which translates the geometry and uploads it to the GPU as described in section \ref{secgeom}.
%
At each Geant4 simulated step of applicable particles the {\tt G4Scintillation} and {\tt G4Cerenkov} classes 
calculate a number of optical photons to generate depending on particle and material properties, 
followed by a loop that generates the optical photons. 
With the hybrid workflow these classes are modified, replacing the generation loop with the collection of 
generation parameters termed "gensteps" that include the number of photons and the line segment along which to generate them and all
other parameters needed to reproduce photon generation on the GPU. Relocating photon generation to the
GPU avoids CPU memory allocation. Only non-culled photon hits needed for the next stage electronics 
simulation require CPU memory allocation. Details on the GPU implementation of the simulation 
can be found in prior proceedings~\cite{chep2023}. 
%
%GPU optical photon generation and propagation are implemented in simple headers that are included
%into the OptiX ray tracing pipeline that runs in parallel. 
%For each step of the propagation, rays representing photons are intersected
%with the geometry using simple header intersect functions that are also included into the ray tracing pipeline.
%The intersected boundary together with the photon wavelength are used to do interpolated texture lookups of
%material properties such as absorption and scattering lengths.
%Converting these lengths to distances using pseudorandom numbers and 
%the known exponential distributions allows a comparison of absorption and scattering distances 
%with geometrical boundary distance to assign photon histories. 
%
%
\begin{figure}
\centering
\includegraphics[width=\textwidth,clip]{/env/presentation/GEOM/J_2024aug27/CSGOptiXRenderInteractiveTest/cxr_min__eye_1,0,0__zoom_1__tmin_0p5__sSurftube_0V1_0:0:-100000.jpg}
\caption{Screenshot from a smoothly interactive ray trace rendering application of the JUNO detector geometry comprising 2560x1440 ray traced pixels created with a CUDA launch of 6 ms
using a single NVIDIA RTX 5000 Ada Generation GPU (3rd generation) with NVIDIA OptiX 8.0 and CUDA 12.4.
The geometry is mostly analytic CSG with a few user selected solids, such as the guide tube torus, using a tesselated representation.  
}    
\label{interactive}
\vspace{-5mm}
\end{figure}
%
%
\section{Detector geometry} 
\label{secgeom}%
% FIRST : HIGH LEVEL DESCRIPTION OF THE VARIOUS GEOMETRY MODELS AND TRANSLATIONS BETWEEN THEM
% AND THEN GET INTO THE DETAILS IN THE SUBSECTIONS
At initialization Opticks translates the geometry through the below sequence of models:
%
\begin{enumerate}
\item Geant4 : deep hierarchy of structural volumes and trees of {\tt G4VSolid} CSG nodes
\item Opticks {\tt stree} : intermediate n-ary trees of volumes and CSG nodes
\item Opticks {\tt CSGFoundry} : CPU/GPU model with {\tt CSGSolid}, {\tt CSGPrim} and {\tt CSGNode} 
\item NVIDIA OptiX 7+ : Instance and Geometry Acceleration Structures (IAS, GAS)
\end{enumerate}  
%
%The {\tt stree} intermediate model is created from the Geant4 model by {\tt U4Tree.h},
%which traverses the Geant4 volume tree converting materials, surfaces, solids, volumes and sensors.
%Triangle and face data for all solids are collected by {\tt U4Mesh.h} using 
%the Geant4 tesselation accessed from {\tt G4Polyhedron}. 
%Subsequently the {\tt CSGFoundry} model is created from the intermediate model and uploaded to GPU.   
%Most of the geometry information in the intermediate model is managed by a single header only struct, {\tt NPFold.h}, 
%which provides an in memory directory tree of arrays using a recursive folders of folders of arrays data structure.
%The complete structural n-ary tree of volumes and n-ary CSG trees for each {\tt G4VSolid} 
%are serialized into arrays using first child and sibling references.
%
Both the {\tt stree} and {\tt CSGFoundry} geometry models are independent of Geant4 and can be persisted into directories of NumPy~\cite{numpy} binary files. 
%Fast binary file loading and uploading to GPU allows optical simulation and visualization
%executables to initialize full detector geometries in a few seconds.  
The prior proceedings~\cite{chep2023} provide a detailed description of the geometry models and conversion between them,
including the crucial geometry factorization into repeated groups of volumes and a remainder of 
other insufficiently repeated volumes that is done within the intermediate {\tt stree} model.
Also the representation of solid shapes using n-ary trees of {\tt sn.h} CSG nodes within  
the {\tt stree} intermediate model are detailed. 
%
%
%
\subsection{Integrated Analytic and Triangulated geometry}
% why use an approx geom
Use of analytic CSG geometry typically allows float precision intersection 
positions to very closely match the double precision intersection positions 
that Geant4 provides. 
Intersection of a ray with a torus requires solution of a quartic equation
with coefficients of greatly varying magnitude that result in unacceptably 
poor numerical precision for some rays when using float precision. 
Replacing the analytic torus geometry with an approximate
tesselated geometry is found to yield more robust intersect positions
and avoids performance reductions from resorting to double precision.  

% how used at high level
During geometry factorization the {\tt stree::collectGlobalNodes} method
assigns non-instanced structural nodes as analytic or
triangulated according to a user provided list of solid names to use 
a tesselated representation. This assignment is communicated via the {\tt CSGSolid}     
to the geometry acceleration structure creation of {\tt SBT::createGAS}. 
%
% stree::findForceTriangulateLVID
% stree::collectGlobalNodes into rem and tri vectors of snode
% CSGSolid.h intent R:rem F:fac T:tri 
% SBT::createGAS 
%
%
%
\section{Visualization}%
%
Opticks provides OpenGL based visualization using GLFW, a lightweight 
Graphics Library Framework, which manages OpenGL windows as well as 
keyboard and mouse inputs. 
Interoperation between OpenGL and CUDA provided by the CUDA runtime API 
allows CUDA/OptiX ray trace render kernels to write to OpenGL 
pixel buffer objects on the GPU which are subsequently accessed from the 
OpenGL rasterization pipeline via texture samplers. This 
GPU resource sharing enables smoothly interactive rendering of 
exactly the same detector geometry as that used by the simulation, as
illustrated in Figure~\ref{interactive}.  
Navigation in 3D via keys and mouse together with viewpoint bookmarking have enabled 
geometry overlap issues to be found by visual inspection.       
Initialization time for visualization of the full JUNO geometry 
is about two seconds to load and upload the binary NumPy arrays of the 
persisted CSGFoundry geometry.   

The Opticks render kernel computes {\tt uchar4} values for each pixel in the image 
plane with RGB color values based on the surface normal direction at geometry intersects. 
The fourth component of the pixel values is set to the z-depth in the eye frame 
mapped appropriately for the effective perspective or orthographic projection matrix, 
as shown in Figure~\ref{depth}.  
This depth component is used from the OpenGL shader rasterization pipeline to set the 
so called fragment depth of the ray traced pixel. This allows compositing of OptiX ray 
trace rendered pixels together with OpenGL rasterized fragments such that 
ray traced geometry can be drawn together with representations of photon positions or gensteps.
%
%
\begin{figure}
\centering
\includegraphics[width=\textwidth,clip]{/env/presentation/GEOM/J_2025jan08/CSGOptiXRenderInteractiveTest/cxr_min__eye_1,0,0__zoom_1__tmin_0p5__NNVT:0:000000.jpg}
\caption{Render of eye frame z-depth from within JUNO detector geometry, used to enable compositing of ray traced geometry and rasterized event representations.}  
\label{depth}
\vspace{-5mm}
\end{figure}%endfig
%
% 
\section{Optical physics}%
%
Opticks optical photon simulation is implemented in the QUDARap package, 
depending only on the SysRap base package and CUDA, with no dependency on OptiX,
as detailed in the prior proceedings~\cite{chep2023}.
%
\subsection{Optimizing random number generation}
\label{optrng} 
%
The curand library~\cite{curand} is used for concurrent generation 
of pseudorandom numbers with separate curandState within each thread
enabling reproducible simulation.
As detailed in prior proceedings~\cite{chep2016}, with the old OptiX 5 API
it was found that a large stack size was required to successfully initialize curand
with the default XORWOW generator. This large stack size resulted in poor ray tracing 
performance, presumably due to larger thread resource usage limiting parallelism.  
This issue motivated install time creation of {\tt curandStateXORWOW} struct 
for all photon "slots" which are persisted to binary files.  
These state files are loaded and uploaded to device at initialization,  
allowing each simulation kernel thread to use the XORWOW generator with its 
prepared state without the expense of initializing the state. 
This approach limits the maximum number of photons 
that can be simulated to the number of persisted states and also the initialization time 
to load and upload the states and required global memory becomes significant when simulating 
hundreds of millions of photons. 
The workaround of dividing curand initialization and generation into 
separate kernel launches is described with example code in the performance notes 
section of the curand manual~\cite{curand}. 

In addition to the default {\tt XORWOW} generator, curand provides several 
pseudorandom number generators including the {\tt Philox4\_32\_10} generator.
Philox is a counter based random number generator which may be implemented to 
use only integer counters for internal state. 
The statistical quality of Philox generated uniform random numbers are comparable 
to those from XORWOW~\cite{curand}.
Adopting the Philox random number generator within Opticks allows direct initialization within
the simulation kernels avoiding the use of state files and hence removing the 
limitation on the number of photons that can be simulated and greatly reducing 
initialization time. 
%
\subsection{Out-of-core optical photon simulation}
%
Simulation of more photons than can fit within VRAM is implemented
using multiple kernel launches invoked from {\tt QSim::simulate}. 
Following genstep collection index range slices into the genstep array 
are chosen such that the number of photons within each slice is less than 
the configured maximum number of CUDA thread "slots" within a single launch.
The default maximum number of slots is determined based upon the total
global memory of the GPU. Result arrays such as photons and hits are gathered
from device to host into separate {\tt NPFold} instances for each kernel launch 
which are subsequently concatenated. To check this functionality 
an event with one billion photons was simulated with an NVIDIA RTX 5000 Ada 
generation GPU with VRAM of 32GB in four kernel launches 
in a total time under 100 seconds. 
% SEvt::gather_components  
% QSim::simulate
%
\section{Simulation performance and validation techniques}
%
Using artificial point sources of photons known as "torch" gensteps with increasing numbers of photons in a sequence of 
purely optical fabricated events provides a convenient way to scan performance as 
a function of photon count. Measurements of Opticks optical simulation times with 1st and 3rd generation 
RTX GPUs are shown in Figure~\ref{rtxgenscan}.
For both sets of measurements the simulation time is found to scale linearly with photon counts
above 20M photons. The performance of the 3rd generation RTX GPU is found to be approximately a factor of four faster 
than the 1st generation GPU. For these measurements the default curand XORWOW generator 
was used with state loading as described in section~\ref{optrng}.  
%
It is notable that the measured times at 1M and 10M photons are almost the same at 0.45(0.14) 
seconds with 1st(3rd) gen. RTX GPUs. Potentially a fixed overhead is dominating processing time. 
That overhead may be the copying of a large fixed number of curandState 
from global to local memory of all threads. This copying is done in order to avoid 
initialization of the XORWOW generator, as described in section~\ref{optrng}.
Further scans comparing performance with the XORWOW and Philox generators
with and without curandState loading are needed to test this idea.  

The Opticks optical simulation is validated by comparisons to Geant4 with several approaches.
One approach uses common input photons and aligns the consumption of randoms by the two optical only simulations 
providing direct comparison unclouded by statistics, as described in~\cite{chep2018}.   
In addition a statistical approach is used which contrasts the histories of photons between the 
two simulations using a Chi-squared comparison of the frequencies of different histories. 
Each step of every photon is characterized with a 4 bit integer enumeration representing reflection, 
transmission, bulk absorption, surface absorption, surface detection, scattering, re-emission etc..
Histories of up to 32 steps of the photon are collected into 128 bit integers for each photon.
Arrays of these photon histories are recorded for both optical simulations. CUDA Thrust is
used to sort the arrays of photon history integers and total the frequencies of each.  
These history frequencies are then compared using a Chi-squared. Issues with the 
simulations such as geometry overlaps, coincidences or close coincidences lead to
large Chi-squared deviations in certain photon histories which typically require
changes to the geometry or the modelling of the geometry to resolve.   
%
\begin{figure}
\centering
%                                    left lower right upper
\includegraphics[width=\textwidth,trim={0 0 0 4cm},clip]{env/presentation/GEOM/J_2024aug27/CSGOptiXSMTest/ALL1_sreport/figs/sreport_ab/mpcap/AB_Substamp_ALL_Etime_vs_Photon_rtx_gen1_gen3.png}
\caption{%
Optical photon simulation times for a sequence of purely optical events with numbers of photons 
in the range from one million to one hundred million. The blue and orange points are measurements from 
two Dell workstations with NVIDIA Titan RTX (1st gen.) and NVIDIA RTX 5000 Ada (3rd gen.) respectively.
}
\label{rtxgenscan}
\vspace{-5mm}
\end{figure}%
%
% Reproduced the v0 plot and Philox equialent on laptop::
%     
%      A=N7 B=A7 SLI=3:12 PLOT=AB_Substamp_ALL_Etime_vs_Photon ~/o/sreport_ab.sh
%      A=N8 B=A8 SLI=2:12 PLOT=AB_Substamp_ALL_Etime_vs_Photon ~/o/sreport_ab.sh
%
%      A=N7 B=A7 A_SLI=3:12 B_SLI=3:12 C=N8 D=A8 C_SLI=2:12 D_SLI=2:12 PLOT=ABCD_Substamp_ALL_Etime_vs_Photon ~/o/sreport_abcd.sh
%
%
\section{Summary}
%
Opticks enables Geant4-based optical photon simulations to benefit from 
state-of-the-art NVIDIA GPU ray tracing, made accessible via the NVIDIA OptiX 7+ API,
allowing memory and time processing bottlenecks to be eliminated. 
Recent feature additions such as out-of-core optical photon simulation and the 
adoption of the curand Philox random number generator remove limits on the number of 
photons that can be simulated and make Opticks easier to use. 
Comparisons between different GPUs shows optical simulation performance 
to closely follow ray tracing performance with a factor of four improvement between 
GPUs from the first and third RTX generations. 
%
%
%\newpage
\section*{Acknowledgements}
%
The JUNO collaboration is acknowledged for the use of detector 
geometries and simulation software. Dr. Tao Lin is acknowledged 
for his JUNO software assistance over many years. NVIDIA is acknowledged for extensive 
support for the OptiX 7+ API transition.
%
This work is supported by National Natural Science Foundation of China (NSFC)
under grant No. 12275293.
%
\begin{thebibliography}{}
%
%1 
\bibitem{opticks}
Opticks Repository, {\tt https://bitbucket.org/simoncblyth/opticks/}\\
Opticks References, {\tt https://simoncblyth.bitbucket.io}\\
Opticks Group, {\tt https://groups.io/g/opticks}
%
%2
\bibitem{chep2023}
S. Blyth, EPJ Web Conf. {\bf 295}, 11014 (2024) \\
{\tt https://doi.org/10.1051/epjconf/202429511014}
%3
\bibitem{chep2021}
S. Blyth, EPJ Web Conf. {\bf 251}, 03009 (2021) \\
{\tt https://doi.org/10.1051/epjconf/202125103009}
%4
\bibitem{chep2019}
S. Blyth, EPJ Web Conf. {\bf 245}, 11003 (2020) \\
{\tt https://doi.org/10.1051/epjconf/202024511003}
%5
\bibitem{chep2018}
S. Blyth, EPJ Web Conf. {\bf 214}, 02027 (2019) \\
{\tt https://doi.org/10.1051/epjconf/201921402027}
%6 
\bibitem{chep2016}
Blyth Simon C 2017 J. Phys.: Conf. Ser. {\bf 898} 042001 \\
{\tt https://doi.org/10.1088/1742-6596/898/4/042001}
%
%7
\bibitem{g4}
S. Agostinelli et al., Nucl. Instrum. Methods. Phys. Res. A {\bf 506}, 250 (2003)\\
\hspace*{0.0em}J. Allison et al., IEEE Trans Nucl Sci, {\bf 53}, 270 (2006)\\
\hspace*{0.0em}J. Allison et al., Nucl. Instrum. Methods. Phys. Res. A {\bf 835}, 186 (2016)
%
%   Recent Developments in Geant4, J. Allison et al., Nucl. Instrum. Meth. A 835 (2016) 186-225
%   Geant4 Developments and Applications, J. Allison et al., IEEE Trans. Nucl. Sci. 53 (2006) 270-278
%   Geant4 - A Simulation Toolkit, S. Agostinelli et al., Nucl. Instrum. Meth. A 506 (2003) 250-303
%
%8
\bibitem{optix}
S. Parker et al., ACM Trans. Graph.: Conf. Series {\bf 29}, 66 (2010)\\
\hspace*{0.0em}OptiX introduction, {\tt https://developer.nvidia.com/optix}\\
\hspace*{0.0em}OptiX API, {\tt https://raytracing-docs.nvidia.com/optix8/index.html}
%
%9
\bibitem{rtx}
NVIDIA RTX\texttrademark\, Platform, {\tt https://developer.nvidia.com/rtx}
%10
\bibitem{juno}
Neutrino physics with JUNO \\
F. An et al., J. Phys. G. {\bf 43}, 030401 (2016) 
%11
\bibitem{throughput}
Understanding Throughput Oriented Architectures \\
M. Garland, D.B. Kirk, Commun. ACM {\bf 53}(11), 58 (2010) 
%12
\bibitem{numpy}
The NumPy array: a structure for efficient numerical computation \\
S. Van der Walt, S. Colbert, G. Varoquaux, Comput. Sci. Eng. {\bf 13}, 22 (2011)
%
%13
\bibitem{curand}
cuRAND, {\tt http://docs.nvidia.com/cuda/curand/index.html}\\
\hspace*{0.0em}Generator tests, {\tt https://docs.nvidia.com/cuda/curand/testing.html}\\
\hspace*{0.0em}{\tt https://docs.nvidia.com/cuda/curand/device-api-overview.html\#performance-notes}
%
%
\end{thebibliography}
%
\end{document}
