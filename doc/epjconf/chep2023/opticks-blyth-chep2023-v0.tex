%2023v0
\documentclass{webofc}
\usepackage[varg]{txfonts}
\newcommand{\comment}[1]{}
\usepackage{graphicx}

\comment{

TODO:

* NEED TO SLIM DOWN FROM 10 TO 8 PAGES
* ACTUALLY NEED TO GET DOWN TO ~6.5 TO ADD NEW SECTIONS:
  
  * new code organization
  * list-nodes
  * TMM

grep section\{ *.tex


* Introduction

  * Importance of optical photon simulation
  * GPU ray tracing
  * NVIDIA\textregistered\ OptiX\texttrademark\ ray tracing engine

* Hybrid simulation workflow

* Detector geometry

  * Material and surface properties
  * Solid shapes
  * Structural volumes
  * Translation from intermediate SSim/stree to CSGFoundry GPU geometry

* Optical physics

* Summary




}

\begin{document}
\title{Opticks : GPU Optical Photon Simulation via NVIDIA OptiX}
\author{\firstname{Simon} \lastname{Blyth}\inst{1}\fnsep\thanks{Corresponding author \email{simon.c.blyth@gmail.com}.}}
\institute{Institute of High Energy Physics, CAS, Beijing, China.}

\abstract{%
Opticks is an open source project that accelerates optical photon simulation  
by integrating GPU ray tracing, accessed via the NVIDIA OptiX 7 API, with 
Geant4 based simulations. 
A single NVIDIA Turing GPU, first RTX generation, has been measured to provide optical 
photon simulation speedup factors exceeding 1500 times single threaded Geant4 
with a full JUNO analytic GPU geometry automatically translated from the Geant4 geometry.
%
Optical physics processes of scattering, absorption, scintillator reemission and 
boundary processes are implemented in CUDA programs based on Geant4.
Wavelength-dependent material and surface properties as well as  
inverse cumulative distribution functions for reemission are interleaved into 
GPU textures providing fast interpolated property lookup or wavelength generation.
%

In this work we describe the near complete re-implementation required to adopt 
the entirely new NVIDIA OptiX 7 API, with the implementation now directly CUDA 
based with OptiX usage restricted to providing intersects.  
The new Opticks features a modular many small header
design that provides fine grained testing both on GPU and CPU 
together with large code reductions from CPU/GPU sharing.  
Enhanced modularity has enabled CSG tree generalization to support "list-nodes", 
similar to G4MultiUnion, that improve performance for complex CSG solids.  
In addition support for interference effects on boundaries with multiple thin layers,
such as anti-reflection coatings and photocathodes, using CUDA compatible
transfer matrix method (TMM) calculations of reflectance, transmittance and
absorptance are reported.

}
%
\maketitle
%
%
\section{Introduction}%
\label{intro}
%
Opticks[1-6] enables Geant4[7-9] based optical photon simulations 
to benefit from high performance GPU ray tracing made accessible 
by NVIDIA\textregistered\ OptiX\texttrademark[10-12].
%
The Jiangmen Underground Neutrino Observatory (JUNO)\cite{juno} 
under construction in southern China will be the world's largest liquid scintillator detector, 
with a 20 kton spherical volume of 35 m diameter. The large size and high photon yield, illustrated in Figure~\ref{problem}, 
makes optical photon simulation extremely computationally challenging for both processing time and memory resources. 
Opticks eliminates these bottlenecks by offloading optical simulation to the GPU. 
%
%
\begin{figure}
\centering
\includegraphics[width=\textwidth,clip]{env/graphics/ggeoview/jpmt-before-contact_half.png}
\caption{Cutaway OpenGL rendering of millions of simulated optical photons from a 200 GeV muon crossing the JUNO liquid scintillator. 
Each line corresponds to a single photon with line colors representing the polarization direction. 
Primary particles are simulated by Geant4, "gensteps" are uploaded to the GPU and photons are generated, propagated
and visualized all on the GPU. 
%Representations of some of the many thousands of photomultiplier tubes that instrument the liquid scintillator are visible.
%The acrylic vessel that contains the liquid scintillator is not shown.
}
\label{problem}
\end{figure}%
%
%Monte Carlo simulation is the primary technique used to design, optimize 
%and analyse diverse detection systems. However, 
Sequential simulation of large numbers of 
optical photons has extreme computational and memory costs. 
Opticks enables drastically improved optical photon simulation performance.
%
Although developed for simulation of the JUNO detector, Opticks
supports use with other detector geometries. 
Any optical photon limited simulation can benefit from Opticks.

Opticks was presented to the four prior CHEP conferences, with each contribution
covering different aspects of its development. The 2021\cite{chep2021} proceedings 
covered initial stages of the transition to the NVIDIA OptiX 7 API and 
the changes to facilitate integration of Opticks with detector simulation frameworks. 
The 2019 plenary presentation and proceedings\cite{chep2019} focused on RTX performance measurements.
2018\cite{chep2018} described the first implementation geometry translation and 
2016\cite{chep2016} detailed the initial CUDA port of photon generation and optical physics.
%
%
These proceedings describe the almost completely re-implemented Opticks,
as required to adopt the entirely new NVIDIA OptiX 7 API. 
Ongoing developments to improve performance with complex solid shapes with "list-nodes" are described
and modelling of multiple thin layer interference effects are described. 
%
%
\subsection{Importance of optical photon simulation}%
%
%
Vetos that suppress of cosmic muon induced backgrounds are crucial for neutrino
detectors such as JUNO\cite{juno}, necessitating production of large simulated samples of muon events. 
However, a muon of typical energy 200 GeV crossing the JUNO scintillator can yield tens of millions of 
optical photons, which are found with Geant4 simulations to consume more than 99\% of CPU time
and impose severe memory constraints.
%
As optical photons in neutrino detectors can be considered to be produced
only by scintillation and Cherenkov processes and yield only hits
on photomultiplier tubes, it is straightforward to combine an external optical photon simulation 
with a Geant4 simulation of all other particles.
%
\subsection{GPU ray tracing}%
%
GPUs evolved to perform rasterized rendering, optimizing throughput\cite{throughput} rather than minimizing latency.
GPUs are suited to problems with millions of independent low resource parallel tasks allowing thousands of threads 
to be in flight simultaneously.
Optical simulation is well matched to these requirements with abundant parallelism 
from huge numbers of photons and low register usage from simplicity of the physics.

The most computationally demanding aspect of photon propagation
is the calculation of intersection positions of rays representing photons with the detector geometry.
This ray tracing limitation of simulation is shared with the synthesis of realistic images in computer graphics. 
Many recent NVIDIA GPUs include 3rd generation RTX hardware dedicated to ray geometry intersection. 
NVIDIA GPU ray tracing performance continues to improve rapidly with performance factors of 2 
from each RTX generation. 
%
%
\subsection{NVIDIA\textregistered\ OptiX\texttrademark\ ray tracing engine}
%
OptiX makes GPU ray tracing accessible with a single ray programming model.
Ray tracing pipelines are constructed combining code for acceleration structure traversal, 
with user code for ray generation, intersection and closest hit handling.
Spatial index acceleration structures (AS) provide accelerated ray geometry intersection. 
OptiX provides only acceleration, not the intersection itself, thus allowing any form of geometry 
to be implemented. 
%
%optix7
In August 2019 NVIDIA introduced the OptiX 7 API\cite{optix7},
that together with the Vulkan and DirectX ray tracing extensions provides access 
to the same NVIDIA ray tracing technology including AS construction and RTX hardware access. 
%
%
%
%
\begin{figure}[t]
\centering
%                                    left lower right upper
\includegraphics[width=\textwidth,trim={0 4cm 0 4cm},clip]{env/Documents/Geant4OpticksWorkflow7/Geant4OpticksWorkflow7_005.png}
\caption{Hybrid Geant4 + Opticks workflow : {\tt G4CXOpticks} translates Geant4 geometry to GPU appropriate form. 
{\tt U4} collects "gensteps" enabling GPU generation of scintillation and Cerenkov photons.
}
% including the number of photons to generate, the line segment along 
%which to generate them and all other parameters used by the photon generatation loop.
%}
\label{workflow} 
\end{figure}
%
\section{Hybrid simulation workflow}%
\label{secworkflow}%
%
Figure~\ref{workflow} summarizes the hybrid workflow. 
At initialization the Geant4 top volume is passed to Opticks
which translates the geometry and uploads it to the GPU as described in section \ref{secgeom}.
%
At each step of applicable particles the {\tt G4Scintillation} and {\tt G4Cerenkov} 
processes calculate a number of optical photons to generate depending on particle and material properties 
followed by a loop that generates the optical photons. With the hybrid workflow 
these classes are modified, replacing the generation loop with the collection of generation parameters termed "gensteps" 
that include the number of photons and the line segment along which to generate them and all
other parameters needed to reproduce photon generation on the GPU. Relocating photon generation to the
GPU avoids CPU memory allocation. Only non-culled photon hits needed for the next stage electronics 
simulation require CPU memory allocation. 

GPU optical photon generation and propagation are implemented in simple headers that are included
into the OptiX ray tracing pipeline that runs in parallel. 
For each step of the propagation, rays representing photons are intersected
with the geometry using simple header intersect functions that are included into the ray tracing pipline.
The intersected boundary index together with the photon wavelength are used to do texture lookups of
material properties such as absorption and scattering lengths.
Converting these lengths to distances using pseudorandom numbers and 
the known exponential distributions allows a comparison of absorption and scattering distances 
with geometrical boundary distance to assign photon histories. 
Earlier proceedings\cite{chep2016} detail efficient use of the cuRAND\cite{curandURL} pseudorandom generator.
%
\section{Detector geometry} 
\label{secgeom}%
%
Opticks provides automated translation between geometry models in the below sequence. 
%
\begin{enumerate}
\item Geant4 : deep hierarchy of structural volumes and trees of {\tt G4VSolid} CSG nodes
\item Opticks {\tt SSim/stree} : CPU intermediate model, with n-ary tree of CSG nodes
\item Opticks {\tt CSGFoundry} : GPU model with {\tt CSGSolid}, {\tt CSGPrim} and {\tt CSGNode} 
\item NVIDIA OptiX 7 : Instance and Geometry Acceleration Structures (IAS, GAS)
\end{enumerate}  
%
The {\tt SSim/stree} intermediate model is created from the Geant4 model by {\tt U4Tree}, 
which traverses the Geant4 volume tree converting materials, surfaces, solids, volumes and sensors.
The {\tt CSGFoundry} model is uploaded to GPU and used from CUDA OptiX 
intersection programs. 
%
Both the Opticks geometry models do not depend on Geant4 and can be serialized to NumPy\cite{numpy} binary files. 
Fast binary file loading and uploading to GPU allows optical simulation and visualization
to initialize full detector geometries in under a second.  
%
The {\tt SSim/stree} model is based upon the {\tt NPFold.h} recursive folders of folders of
{\tt NP.hh} arrays data structure which enables a very minimal approach. This is very different 
from the former heavyweight {\tt GGeo} model that it replaces. 
%
%
%3.1
\subsection{Material and surface properties} 
%
Material and surface properties as a function of wavelength are interpolated onto a common wavelength domain.
The properties include refractive indices, absorption lengths, scattering lengths, reemission probabilities, 
as well as surface reflectivities, detection efficiencies and absorption fractions. 
% and stored into Opticks instances {\tt GMaterialLib} and {\tt GSurfaceLib}.
Each volume of the geometry is assigned a boundary index uniquely identifying the combination of four indices representing 
outer and inner materials and outer and inner surfaces. Outer/inner surfaces handle inwards/outwards going photons, which allows the Geant4 border and skin
surface functionality to be translated. Surfaces with a non-zero efficiency property are used to identify sensor volumes. 

GPUs contain hardware dedicated to fast texture lookup and interpolation.
This is exploited by using a single 2D {\tt float4} texture, named the boundary texture,
that contains interleaved material and surface properties as a function of wavelength for all
unique boundaries.
The boundary index returned from a ray traced primitive intersection
%,together with an orientation offset identified from the angle between the geometric normal and ray direction,
enables four wavelength interpolated material or surface properties to be
obtained from a single hardware optimized texture lookup.
%
% 3.2
\subsection{Solid shapes}%
%
The {\tt SSim/stree} intermediate model carries solid shape information
within an n-ary tree of {\tt sn.h} CSG nodes. Although most of the geometry 
can be treated as a binary tree certain Geant4 solids such as {\tt G4MultiUnion}
are better handled with an n-ary tree.   
%
%primitives
The Opticks CSG package implements ray primitive shape intersection in simple headers that 
are CUDA compatible but can also be used and debugged on the CPU.
These functions use implicit equations for the primitives together with 
the parametric ray equation, to yield a polynomial in $t$, the distance along the ray from its origin position. 
Roots and derivatives yield intersections and surface normals.
%
% combinations
Arbitrarily complex solids are described using constructive solid geometry (CSG) modelling, 
which builds shapes from the combination of primitive constituents by boolean set operations
and is represented with a binary tree data structure.
%
% serialization 
Each primitive or operator node is serialized into an array of up to 16 elements. 
These elements include float parameters of the primitives and integer index references 
into a separate transform array. 
A complete binary tree serialization with array indices matching level order tree indices
and zeros at missing nodes is used for the serialization of the CSG trees. This simple 
serialization allows tree navigation directly from bitwise manipulations of the serialized array index.
%
% balancing
Complete binary tree serialization is simple and effective for small trees but very inefficient 
with the large trees that result from complex shapes with many constituent primitives. 
Prior proceedings\cite{chep2018} describe the use of tree balancing to reduce the  
tree height and avoid poor performance with complex solids. However studies have revealed that 
tree balancing is not compatible with the CSG intersection algorithm currently used by Opticks. 
% TODO: add list-nodes section and refer to it from here
%
%
\subsection{Structural volumes}%
%
The Geant4 structural geometry model comprises a containment tree hierarchy of volumes with associated transforms.
The Opticks geometry model is based upon the observation that many of these volumes are 
repeated in groups, corresponding for example to the small number of volumes that represent each type of PMT, 
and thus an efficient representation must make full use of geometry instancing. 
Geometry instancing is a technique used in graphics libraries including OpenGL and NVIDIA OptiX that 
avoids duplication of information on the GPU by storing repeated elements only once together 
with 4x4 transform matrices that specify the locations and orientations of each instance.
%
The translation between the Geant4 and Opticks geometry models by {\tt U4Tree.h} 
starts by creating a minimal n-ary tree of Opticks {\tt snode.h} 
with {\tt num\_child, first\_child, next\_sibling} fields that allow simple serialization. 

The Geant4 model of JUNO geometry contains almost 400,000 structural volumes.
These volumes are factorized into repeated groups of volumes and a remainder of 
other insufficiently repeated volumes using subtree digests that represent the geometry and transforms 
beneath every node of the tree. This factorization is now implemented in a minimal way 
within {\tt stree::factorize}, replacing the former heavyweight implementation spread across many classes. 
The outcome of the factorization is a repeat index on every structural node allowing 
the instance transforms of each repeat to be collected. 

%
\subsection{Translation from intermediate SSim/stree to CSGFoundry GPU geometry}%
%
The small number of geometry factors correspond to repeated assemblies of groups of volumes, 
that each become compound {\tt CSGSolid} within the {\tt CSGFoundry} model.  
With the JUNO geometry the {\tt CSGSolid} corresponds for example to the different types of PMT.
Each {\tt CSGSolid} typically contains several {\tt CSGPrim}, for example the Pyrex and Vacuum volumes of the PMT, 
which usually correspond to root {\tt G4VSolid} in the Geant4 model. 
The {\tt CSGPrim} in turn contain CSG node trees of {\tt CSGNode} that are one to one related with the {\tt sn.h} CSG nodes. 
These correspond to constituent {\tt G4VSolid} such as {\tt G4Ellipsoid} and {\tt G4Tubs}. 
Figure~\ref{j1808} presents a ray traced rendering of the analytic representation of the JUNO detector geometry. 

%
%
\begin{figure}
\centering
%\includegraphics[width=\textwidth,clip]{env/presentation/j1808/j1808_top_rtx}
%\caption{Ray traced rendering of the chimney region of JUNO detector showing photomultiplier tubes, 
%acrylic sphere, supports and the calibration guide tube torus. 
%The rendering uses exactly the same analytic geometry as the simulation.
%The geometry was directly converted from Geant4 into an Opticks geometry including analytic CSG 
%and persisted into a geometry cache of NumPy\cite{numpy} binary files. 
%}
\includegraphics[width=\textwidth,clip]{env/presentation/CSGOptiXRender/CSG_GGeo/cvd1/70000/cxr_overview/cam_0_tmin_0.4/cxr_overview_emm_1,2,3,4_moi_-1.jpg}
\caption{Render of the PMTs of the JUNO detector comprising 1920x1080 ray traced pixels created with a CUDA launch of under a hundredth of a second  
using a single NVIDIA TITAN RTX GPU and NVIDIA OptiX 7.
The geometry is modelled using a shared CPU/GPU geometry model\cite{csgfoundry} 
designed to work with the NVIDIA OptiX 7 API, converted with CSG\_GGeo\cite{csgggeo} and rendered using CSGOptiX\cite{csgoptix}.
\label{j1808}}
\end{figure}
%
%
\section{Optical physics}%
%
Optical physics processes of scattering, absorption, scintillator reemission and 
boundary processes are implemented in CUDA functions based on the Geant4
implementations. 
%
The single ray programming model of NVIDIA OptiX enables 
direct ports of the corresponding Geant4 implementations adapted
to use GPU textures for property access. 
%
On the CPU, it is convenient to implement scintillator reemission using Geant4 secondary 
tracks. A different approach is adopted on the GPU where a fraction of absorbed photons
are reborn with modified direction and wavelength within the same CUDA thread. 
A reemission texture that encapsulates an inverse cumulative distribution function 
is used to generate wavelengths that follow the desired distribution on lookup of pseudorandom numbers. 
%
Further details on the CUDA ports of Geant4 optical physics and use of GPU textures 
are given in the 2016 proceedings\cite{chep2016}.
%
%
%
%
\section{Summary}
%
Opticks enables Geant4-based optical photon simulations to benefit from 
state-of-the-art NVIDIA GPU ray tracing, made accessible via NVIDIA OptiX,
that allows memory and time processing bottlenecks to be eliminated.
Recent developments enable Opticks to greatly reduce the CPU memory for hits 
by moving collection efficiency hit culling to the GPU. 
Opticks meets the challenge of optical photon simulation in JUNO, 
the world's largest scintillator detector, and can benefit any simulation limited by optical photons. 

Several groups from various experiments and the Geant4 Collaboration are evaluating Opticks.
Physicists from the LZ dark matter experiment and LBNL suggested and 
organized a series of meetings with NVIDIA engineers that have assisted with the 
migration of Opticks to the all new NVIDIA OptiX 7 API. Figure.\ref{j1808} 
is one of the first renders of the full JUNO geometry with OptiX 7.
%The initial steps in evaluating Opticks with your experiment geometry are straightforward,
%{\tt\bf https://simoncblyth.bitbucket.io/opticks/docs/opticks.html}.  
%
%
\newpage
\section*{Acknowledgements}
%
The JUNO collaboration is acknowledged for the use of detector 
geometries and simulation software. Dr. Tao Lin is acknowledged for his assistance with 
the JUNO offline software. 
This work is funded by Chinese Academy of Sciences President’s International Fellowship Initiative,
Grant No. 2018VMB0002.
%
%
%
\begin{thebibliography}{}
%
%1
\bibitem{opticksURL}
Opticks Repository, {\tt https://bitbucket.org/simoncblyth/opticks/}
%2
\bibitem{opticksRefs}
Opticks References, {\tt https://simoncblyth.bitbucket.io}
%3
\bibitem{opticksGroup}
Opticks Group, {\tt https://groups.io/g/opticks}
%4 
\bibitem{chep2019}
S. Blyth, EPJ Web Conf. {\bf 245}, 11003 (2020) \\
{\tt https://doi.org/10.1051/epjconf/202024511003}
%5
\bibitem{chep2018}
S. Blyth, EPJ Web Conf. {\bf 214}, 02027 (2019) \\
{\tt https://doi.org/10.1051/epjconf/201921402027}
%6 
\bibitem{chep2016}
Blyth Simon C 2017 J. Phys.: Conf. Ser. {\bf 898} 042001 \\
{\tt https://doi.org/10.1088/1742-6596/898/4/042001}
%
%
%
%
%
%
%
%7
\bibitem{g4A}
S. Agostinelli, J. Allison, K. Amako, J. Apostolakis, H. Araujo, P. Arce et al., Nucl. Instrum. Methods. Phys. Res. A {\bf 506}, 250 (2003)
%8
\bibitem{g4B}
J. Allison, K. Amako, J. Apostolakis, H. Araujo, P. Dubois, M. Asai et al., IEEE Trans Nucl Sci, {\bf 53}, 270 (2006)
%9
\bibitem{g4C}
J. Allison, K. Amako, J. Apostolakis, P. Arce, M. Asai, T. Aso et al., Nucl. Instrum. Methods. Phys. Res. A {\bf 835}, 186 (2016)
%
%
%
%
%
%10
\bibitem{optixPaper}
OptiX: a general purpose ray tracing engine \\
S. Parker, J. Bigler, A. Dietrich, H. Friedrich, J. Hoberock et al., ACM Trans. Graph.: Conf. Series {\bf 29}, 66 (2010)
%11
\bibitem{optixSite}
OptiX introduction, {\tt https://developer.nvidia.com/optix}
%12
\bibitem{optixDocs}
OptiX API, {\tt http://raytracing-docs.nvidia.com/optix/index.html}
%
\bibitem{optix7}
OptiX 7 {\tt https://developer.nvidia.com/blog/how-to-get-started-with-optix-7/}
%
\bibitem{junosim}
The Application of SNiPER to the JUNO Simulation,\\
T. Lin et al., J.Phys.Conf.Ser. {\bf 898} 042029 (2017)\\
{\tt https://doi.org/10.1088/1742-6596/898/4/042029}
%
\bibitem{juno}
Neutrino physics with JUNO \\
F. An et al., J. Phys. G. {\bf 43}, 030401 (2016) 
%
\bibitem{rtx}
NVIDIA RTX, {\tt https://developer.nvidia.com/rtx}
%
%
%
\bibitem{throughput}
Understanding Throughput Oriented Architectures \\
M. Garland, D.B. Kirk, Commun. ACM {\bf 53}(11), 58 (2010) 
%
\bibitem{curandURL}
cuRAND, {\tt http://docs.nvidia.com/cuda/curand/index.html}
%
\bibitem{numpy}
The NumPy array: a structure for efficient numerical computation \\
S. Van der Walt, S. Colbert, G. Varoquaux, Comput. Sci. Eng. {\bf 13}, 22 (2011)
%
\bibitem{thrust}
Chapter 26 - Thrust: A Productivity-Oriented Library for CUDA \\
N. Bell, J. Hoberock, GPU Computing Gems Jade Edition, (2012), pp 359-371
%
\bibitem{csgfoundry}
Shared GPU/CPU "CSGFoundry" geometry model,\\ {\tt https://github.com/simoncblyth/CSG}
%
\bibitem{csgggeo}
Converter of Opticks/GGeo geometry to "CSGFoundry" model,\\ {\tt https://github.com/simoncblyth/CSG\_GGeo}
%
\bibitem{csgoptix}
NVIDIA OptiX 7 and pre-7 renderer of "CSGFoundry" geometry,\\ {\tt https://github.com/simoncblyth/CSGOptiX}
%
\end{thebibliography}
%
\end{document}
