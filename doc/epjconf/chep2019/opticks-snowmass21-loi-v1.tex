%2019v0
\documentclass{webofc}
\usepackage[varg]{txfonts}
\newcommand{\comment}[1]{}
\usepackage{graphicx}

\begin{document}
%
\title{Opticks: GPU photon simulation via NVIDIA\textregistered\ OptiX\texttrademark}

\author{\firstname{Simon} C. \lastname{Blyth}\inst{1}\fnsep\thanks{\email{simon.c.blyth@gmail.com}.}}

\institute{Institute of High Energy Physics, CAS, Beijing, China.}

\abstract{%
Opticks is an open source project that accelerates optical photon simulation by 
integrating NVIDIA GPU ray tracing, accessed via NVIDIA OptiX, with 
Geant4 toolkit based simulations. 
A single NVIDIA Turing architecture GPU has been measured to provide optical 
photon simulation speedup factors exceeding 1500 times single threaded Geant4 
with a full JUNO analytic geometry automatically translated from the Geant4 geometry.
%
Optical physics processes of scattering, absorption, scintillator reemission and 
boundary processes are implemented within CUDA OptiX programs based on the Geant4
implementations. Wavelength-dependent material and surface properties as well as  
inverse cumulative distribution functions for reemission are interleaved into 
GPU textures giving fast interpolated lookup or wavelength generation.
%
}
%
\maketitle
%
%
%\section{Introduction}%
\label{intro}
%
\comment{
The introduction should provide background that puts the manuscript into
context and allows readers outside the field to understand the purpose and
significance of the study. It should define the problem addressed and explain
why it is important.
}
%
Opticks[1-5] enables Geant4[6-8]-based optical photon simulations 
to benefit from high performance GPU ray tracing made accessible 
by NVIDIA\textregistered\ OptiX\texttrademark[9-11].
%
The Jiangmen Underground Neutrino Observatory (JUNO)\cite{juno} 
under construction in southeast China features the world's largest liquid scintillator detector, 
with a 20 kton spherical volume of 35 m diameter. The large size and high photon yield
of the scintillator, makes the JUNO optical photon simulation extremely computationally 
challenging with regard to both processing time and memory resources. Opticks eliminates both these 
bottlenecks by offloading the optical photon simulation to the GPU. 

%Although Opticks was developed for the simulation of the JUNO detector, it 
%is structured to enable use with other detector geometries. 
Opticks auto-translates Geant4 detector geometries to GPU optimized forms without
approximation. This translation was developed in the context of the JUNO detector. 
%
Any detector simulation limited by optical photons 
can benefit from Opticks.
Drastically improved optical photon simulation performance can be transformative 
to the design, operation and understanding of diverse optical systems.
%
Several groups from various neutrino experiments and dark matter search experiments are 
evaluating Opticks.
%
Recent Opticks developments allow the optical photon simulation performance 
to benefit from ray trace dedicated processors, called RT cores\cite{rtx}, 
available in NVIDIA Turing architecture GPUs.
%
%
\begin{figure}
\centering
% left lower right upper
\includegraphics[width=\textwidth,trim={0 5cm 0 5cm},clip]{env/Documents/Geant4OpticksWorkflow/Geant4OpticksWorkflow1_001.png}
\caption{Comparison of the standard workflow of Geant4 optical photon simulation (left) with the hybrid Geant4 + Opticks workflow (right)
using a single {\tt G4Opticks} interface class.
%A single Opticks class 
% acts to interface Geant4 user code with the Opticks GPU propagation. 
%Hybrid simulation requires modification of the classes representing scintillation and Cherenkov processes
%to collect "genstep" data structures.
}
%}
\label{workflow} 
\end{figure}
%
%
%\subsection{GPU ray tracing}%
%

The most computationally demanding aspect of optical photon simulation 
is the calculation, at each step of the propagation, 
of intersection positions of rays representing photons with the geometry of the system.
This ray tracing limitation of optical photon simulation is shared 
with the synthesis of realistic images in computer graphics. 
%Due to the many applications
%of ray tracing in the advertising, design, games and film industries, 
The computer graphics community has continuously improved ray tracing techniques. The Turing GPU architecture 
introduced by NVIDIA in 2018 is marketed as the world's first Ray-tracing GPU, with   
hardware "RT Cores" 
%in every streaming multiprocessor (SM) 
dedicated to the 
acceleration of ray geometry intersection.
NVIDIA claims performance of more than 10 billion ray intersections 
per second, which is a factor 10 more than possible with earlier GPUs
which perform the intersection acceleration in software. 
%
%
%\section{Hybrid simulation workflow}
\label{secworkflow}
%

Implementing an efficient GPU optical photon simulation equivalent to the Geant4 simulation 
requires that all aspects of the Geant4 context relevant to optical photon generation and 
propagation are translated into an appropriate form and uploaded to the GPU. 
%The primary aspects are the detector geometry including material/surface properties, optical physics and optical photons.
%
Figure~\ref{workflow} illustrates the hybrid simulation workflow. 
At initialization the Geant4 top volume pointer is
passed to Opticks which translates the geometry and constructs 
the OptiX GPU context including intersection, bounding box 
and closest hit CUDA programs and buffers that these programs access.
GPUs contain hardware dedicated to fast texture lookup and interpolation, that 
is exploited via a single 2D {\tt float4} "boundary" texture containing interleaved material 
and surface properties as a function of wavelength for all unique boundaries.
The boundary index returned from a ray traced primitive intersection
%together with an orientation offset 
%identified from the angle between the geometric normal and ray direction,
enables four wavelength interpolated material or surface properties to be
obtained from a single hardware optimized texture lookup.

%\subsection{Solid shapes}%
%
% primitives
\comment{
Opticks provides CUDA functions that return ray intersections for ten primitive shapes including sphere, hyperboloid and torus.
These functions use implicit equations for the primitives together with 
the parametric ray equation, to yield a polynomial in $t$, the distance along the ray from its origin position. 
Roots of the polynomials provide intersections, and surface normals at intersects are obtained
using the derivative of the implicit equation.
}
%
% combinations
\comment{
Arbitrarily complex solids are described using constructive solid geometry (CSG) modelling, 
which builds shapes from the combination of primitive constituents by boolean set operations: union, intersection and difference.
A binary tree data structure with primitives at the leaves of the tree and operators at the internal nodes is used
to represent the solids. Any node can have an associated local transform, represented by a 4x4 transformation matrix, which 
is combined with other local transforms to yield global transforms in the frame of the root node of the tree.
%
% serialization 
Each primitive or operator node is serialized into an array of up to 16 elements. 
These elements include float parameters of the primitives and integer index references 
into a separate global transform buffer. 
%For the convex polyhedron primitive which is defined by a list of surface planes, 
%the primitive contains an integer index referencing into a separate plane buffer together with the number of planes. 
A complete binary tree serialization with array indices matching level order tree indices
and zeros at missing nodes is used for the serialization of the CSG trees. This simple 
serialization allows tree navigation directly from bitwise manipulations of the serialized array index.
%
% balancing
Complete binary tree serialization is simple and effective for small trees but very inefficient 
for unbalanced trees, necessitating tree balancing for shapes with many constituent primitives 
to reduce the tree height. 
The prior proceedings\cite{chep2018} provide further details of the constructive solid geometry modelling, 
tree balancing and translation between Geant4 and Opticks solids.
}
%\subsection{Structural volumes}%
%

The Opticks geometry model is based upon the observation that many elements of a detector 
geometry are repeated demanding the use of instancing for efficient representation, see Figure~\ref{problem}. 
Geometry instancing is a technique used in computer graphics libraries including OpenGL and NVIDIA OptiX that avoids 
duplication of information on the GPU by storing repeated elements only once together with 4x4 transform matrices 
that specify the locations and orientations of each instance.
%The Geant4 geometry model comprises a hierarchy of volumes with associated transforms. 
A digest string for every structure node is formed from the transforms and shape indices of it's progeny nodes. 
Subsequently repeated sub-trees and their placement transforms are identified using the digests, after disqualifying repeated sub-trees
that are contained within other repeats. 
All structure nodes passing instancing criteria are assigned an instance index
with the remainder forming the global non-instanced group. 
These instanced sub-trees are used for the creation of the NVIDIA OptiX analytic geometry instances, and OpenGL mesh geometry instances.  
%Figure~\ref{j1808} presents a ray traced rendering of the analytic representation of the JUNO detector geometry. 
%
%
%
%
%
%
%\section{Optical photons and "gensteps"}%
%

Photons are generated on the GPU via NVIDIA OptiX ray generation programs, 
using CUDA ports of Geant4 photon generation loops and "genstep" buffers  
collected within modified scintillation and Cerenkov processes. 
%
Instead of generating photon secondary tracks, "genstep" parameters such as   
the process type code, the number of photons to generate and the line segment along which to generate
them are collected. 
Collecting and copying gensteps to the GPU rather than photons avoids
allocation of CPU memory for the photons, only collected photon hits 
require CPU memory allocation. 
%
%

Opticks aims to provide GPU accelerated optical photon simulation for any detector.
Achieving this requires physicists from many experiments to use and improve Opticks.
Snowmass can assist by introducing Opticks to a wider community.
%
%
%
%Opticks aspires to become the standard way to perform optical photon simulation. In order to achieve this
%it is necessary to attract many more users and developers. It is hoped that exposure and analysis from 
%the Snowmass process will assist Opticks and the community.
%
%
\newpage
%
%
\begin{figure}
\centering
\includegraphics[width=\textwidth,clip]{env/graphics/ggeoview/jpmt-before-contact_half.png}
\caption{Cutaway OpenGL rendering of millions of simulated optical photons from a 200 GeV muon crossing the JUNO liquid scintillator. 
Each line corresponds to a single photon with line colors representing the polarization direction. Primary particles
are simulated by Geant4, scintillation and Cerenkov "gensteps" are uploaded to the GPU and photons are generated, propagated
and visualized all on the GPU. Representations of some of the many thousands of photomultiplier tubes that instrument the liquid scintillator are visible.
The acrylic vessel that contains the liquid scintillator is not shown.
}
\label{problem}
\end{figure}%
%
\begin{thebibliography}{}
%
%1
\bibitem{opticksURL}
Opticks Repository, {\tt https://bitbucket.org/simoncblyth/opticks/}
%2
\bibitem{opticksRefs}
Opticks References, {\tt https://simoncblyth.bitbucket.io}
%3
\bibitem{opticksGroup}
Opticks Group, {\tt https://groups.io/g/opticks}
%4 prior proceedings
\bibitem{chep2018}
S. Blyth, EPJ Web Conf. {\bf 214}, 02027 (2019) \\
{\tt https://doi.org/10.1051/epjconf/201921402027}
%5 earlier proceedings
\bibitem{chep2016}
Blyth Simon C 2017 J. Phys.: Conf. Ser. {\bf 898} 042001 \\
{\tt https://doi.org/10.1088/1742-6596/898/4/042001}
%
%
%
%6
\bibitem{g4A}
S. Agostinelli, J. Allison, K. Amako, J. Apostolakis, H. Araujo, P. Arce et al., Nucl. Instrum. Methods. Phys. Res. A {\bf 506}, 250 (2003)
%7
\bibitem{g4B}
J. Allison, K. Amako, J. Apostolakis, H. Araujo, P. Dubois, M. Asai et al., IEEE Trans Nucl Sci, {\bf 53}, 270 (2006)
%8
\bibitem{g4C}
J. Allison, K. Amako, J. Apostolakis, P. Arce, M. Asai, T. Aso et al., Nucl. Instrum. Methods. Phys. Res. A {\bf 835}, 186 (2016)
%
%
%
%9
\bibitem{optixPaper}
OptiX: a general purpose ray tracing engine \\
S. Parker, J. Bigler, A. Dietrich, H. Friedrich, J. Hoberock et al., ACM Trans. Graph.: Conf. Series {\bf 29}, 66 (2010)
%10
\bibitem{optixSite}
OptiX introduction, {\tt https://developer.nvidia.com/optix}
%11
\bibitem{optixDocs}
OptiX API, {\tt http://raytracing-docs.nvidia.com/optix/index.html}
%12
\bibitem{juno}
Neutrino physics with JUNO 
F. An et al., J. Phys. G. {\bf 43}, 030401 (2016) 
%13
\bibitem{rtx}
NVIDIA RTX, {\tt https://developer.nvidia.com/rtx}
%
%

\end{thebibliography}





%
\end{document}
