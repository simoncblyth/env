\documentclass{webofc}
\usepackage[varg]{txfonts}
\newcommand{\comment}[1]{}

\begin{document}
%
\title{Opticks : GPU Optical Photon Simulation for\\ Particle Physics using NVIDIA\textregistered\ OptiX\texttrademark}

\author{\firstname{Simon} \lastname{Blyth}\inst{1}\fnsep\thanks{\email{simon.c.blyth@gmail.com}} }

\institute{Institute of High Energy Physics, CAS, Beijing, China.}

\abstract{%
Opticks is an open source project that integrates the NVIDIA OptiX 
GPU ray tracing engine with Geant4 toolkit based simulations.
Massive parallelism brings drastic performance improvements with  
optical photon simulation speedup expected to exceed 1000 times Geant4 
with workstation GPUs. 

Optical physics processes of scattering, absorption, scintillator reemission and 
boundary processes are implemented as CUDA OptiX programs based on the Geant4
implementations. Wavelength dependent material and surface properties as well as  
inverse cumulative distribution functions for reemission are interleaved into 
GPU textures providing fast interpolated property lookup or wavelength generation.
OptiX handles the creation and application of a choice of acceleration structures
such as boundary volume heirarchies and the transparent use of multiple GPUs. 

A major recent advance is the implementation of GPU ray tracing of  
complex constructive solid geometry shapes, enabling
automated translation of Geant4 geometries to the GPU without approximation.
Using common initial photons and random number sequences allows 
the Opticks and Geant4 simulations to be run point-by-point aligned.
Aligned running has reached near perfect equivalence with test geometries.
}
%
\maketitle
%
%
\section{Introduction}
\label{intro}
%
\comment{
The introduction should provide background that puts the manuscript into
context and allows readers outside the field to understand the purpose and
significance of the study. It should define the problem addressed and explain
why it is important.
}
%
Opticks\cite{opticksURL} enables Geant4\cite{g4A}\cite{g4B}\cite{g4C}
based optical photon simulations to benefit from high performance massively parallel GPU ray tracing 
provided by the NVIDIA\textregistered\ OptiX\texttrademark\cite{optixPaper}\cite{optixSite} package.
With the recently introduced NVIDIA Turing architecture GPUs OptiX performance is further enhanced 
by dedicated ray-tracing processors called RT cores\cite{rtx}.
Recent Opticks developments enable automated direct translation of Geant4 geometries into a GPU 
optimized constructive solid geometry (CSG) form without approximation, enabling a precise matching 
between the the CPU and GPU geometry to be achieved. 

Monte Carlo simulations of optical photon propagation 
are used to develop models of diverse photon transport systems 
ranging from human tissue within medical imaging scanners 
to large volumes of scintillators instrumented with photon detectors in neutrino 
and dark matter search experiments. 
Such simulations are the primary technique used to design, optimize 
and analyse complex detection systems. However the versatility 
of the Monte Carlo technique comes with computational and memory costs 
that become extreme when propagating large numbers of photons
using traditional sequential processing. Opticks aims to provide a 
parallel solution to this optical photon simulation problem.  Drastically 
improved optical photon simulation performance can be transformative 
to the design, operation and understanding of diverse optical systems.

These proceedings focus on recent major changes to the geometry workflow of Opticks 
and to the validation approach. The prior proceedings\cite{chep2016} provide further
details on NVIDIA OptiX, use of GPU textures and the CUDA port of Geant4 photon generation 
and optical physics.
%
\subsection{Importance of Cosmic Muon background to Neutrino Detectors}
%
Cosmic muon induced processes are crucial backgrounds for neutrino
detectors such as Daya Bay~\cite{dyb} and JUNO\cite{juno},
necessitating underground sites, water shields and muon veto systems.
Minimizing the dead time and dead volume that results from applying
a veto requires an understanding of the detector response to a muon.
Large simulated samples of muon events are crucial in order to
develop such an understanding.
%
The number of optical photons estimated to be produced by a muon of
typical energy 100 GeV crossing the JUNO scintillator is at the level of tens of millions.
Profiling the Geant4 toolkit based simulation shows that the optical photon propagation 
consumes more than 99\% of CPU time, and imposes severe memory constraints that have forced
the use of event splitting.  
%
As optical photons in neutrino detectors can be considered to be produced
by only the scintillation and Cerenkov processes and yield only hits
on photomultiplier tubes it is straightforward to integrate an
external optical photon simulation with a Geant4 simulation of all other particles.
%
\subsection{Throughput Oriented Processing}
%
Graphics Processing Units (GPUs) originally designed for rendering 
are now increasingly used for computing tasks with parallel workloads.
Efficient use of GPUs requires problems with can be divided into 
many thousands of mostly independent parallel tasks, which are 
each executed within separate threads. To maximize the number of simultaneous 
running threads the use of shared resources such as registers and shared 
memory and synchronization between threads must be minimized.

The characteristics of the simulation of optical photons are well matched to 
the requirements for efficient GPU usage, with sufficient parallelism from 
the large numbers of photons, low register usage from the simplicity of 
the optical physics, and decoupled nature of the photons avoiding synchronization.

The most computationally demanding aspect of optical photon simulation 
arises from the calculation, at each step of the propagation, 
of intersection positions of rays representing photons with the geometry of the system.
This ray tracing limitation of optical photon simulation is shared 
with the synthesis of realistic images in computer graphics. Due to the many applications
of ray tracing in the advertising, design, games and film industries the computer graphics
community has continuously improved ray tracing techniques. The Turing GPU architecture 
introduced by NVIDIA in 2018 is marketed as the worlds first Ray-tracing GPU, with   
hardware "RT Cores" in every streaming multiprocessor (SM) dedicated to the 
acceleration of ray geometry intersection testing.
NVIDIA claims performance of more than 10 billion ray geometry intersections 
per second, which is a factor 10 more than possible with its Pascal architecture GPUs
which performs the intersection acceleration in software. 

GPUs are throughput-oriented\cite{throughput}, optimizing the total amount of work completed per unit time,
by using many simple processing units. This contrasts with CPUs which are latency-oriented, 
minimizing the time elapsed between initiation and completion of a single task by the use of complex
caching systems to avoid the latency of memory access and other complicated techniques such as branch prediction 
and speculative execution. GPUs generally do not use the complicated techniques common in CPUs, 
allowing more chip area to be dedicated to parallel compute, which results in greater computational throughput across 
all threads at the expense of slower single-thread execution.
Threads blocked while waiting to access memory are tolerated by using 
hardware multithreading to resume other unblocked threads, allowing latencies to be hidden assuming 
there are sufficient parallel threads in flight. 
GPUs evolved to meet the needs of real-time computer graphics rendering images of millions of pixels from geometries
composed of millions of triangles, they are designed to execute literally billions of small "shader" 
programs per second. When porting CPU code to run on the GPU a total reorganization of 
data and computation is required in order to make effective use of the totally different processor 
architecture.
%
\subsection{NVIDIA\textregistered\ OptiX\texttrademark\ Ray Tracing Engine}
%
OptiX\cite{optixPaper}\cite{optixSite} is a general-purpose ray tracing engine
designed for NVIDIA GPUs that exposes an accessible single ray programming model.
The core of OptiX is a domain-specific just-in-time compiler that constructs 
ray tracing pipelines combining code for acceleration structure creation and traversal 
together with user provided CUDA code for ray generation, object intersection and closest hit handling.
Spatial index data structures, such as the boundary volume hierarchy (BVH), are the 
principal technique for accelerating ray geometry intersection.  The OptiX API\cite{optixDocs} 
provides only the acceleration of ray geometry intersection, not the intersection inself 
thus affording users full flexibility to implement intersections with any form of geometry. In the recently introduced Turing 
architecture GPUs some of the BVH traversal workload has been moved into hardware "RT cores".
The OptiX acceleration structure system supports instancing, allowing the structures to be shared 
between multiple instances of the same geometry such as the photomultipler tubes in the JUNO geometry.  
%
\section{Geometry Translation and Upload to GPU}
\label{geometry}
%
Implementing an efficient GPU optical photon simulation equivalent to the Geant4 simulation 
requires that all aspects of the Geant4 context relevant to optical photon propagation are 
translated into an appropriate form and serialized for upload to the GPU.
%
\subsection{Modelling Complex Shapes using Constructive Solid Geometry}
%
% primitives
Opticks provides ray intersection for ten primitive shapes including sphere, hyperboloid and torus.
Ray primitive intersection uses the parametric ray equation together with implicit equations 
of the primitives to yield a polynomial in $t$, the distance along the ray from its origin position. 
Intersections are found from the roots of the polynomial with $t > t_{min}$ and surface normals are obtained at intersects
using the derivative of the implicit equation or by inspection.
%
% combinations
Arbitrarily complex solids are described using constructive solid geometry (CSG) modelling, 
which builds shapes from the combination of primitive constituent shapes by boolean set operations: union, intersection and difference.
A binary tree data structure with primitives at the leaves of the tree and operators at the internal nodes is used
to represent the solids. Any node can have an associated local transform, represented by 4x4 transformation matrix corresponding 
to a rotation followed by a translation. The local transforms are combined to yield global transforms in the frame of the root node of the tree
for every node of the tree. CSG requires all constituents to be closed solids, necessitating 
endcap implementations for all primitives that allow truncation. 
%
\subsection{Ray Intersection with CSG Shapes}
\label{intersection}
%
% intersection
Intersecting rays with general CSG shapes requires the appropriate primitive intersect to be 
selected depending on the origin of the ray and the current $t_{min}$. Traditional implementations of CSG intersection first calculate 
ray intervals with each primitive and then combine these intervals using the boolean operators to determine intersects.  
Efficient use of GPUs requires many thousands of simultaneously operational threads which disfavors the traditional approach due to 
the requirement to store intervals for all constituent primitives.  A quite different approach
described by Andrew Kensler\cite{kensler} avoids interval storage by instead selecting between 
candidate intersects A and B at each level of the binary tree, which allows a recursive algorithm 
to be developed. The two candidate intersects at each level are classified as "Enter", "Exit" or "Miss" 
using the angle between the ray direction and surface normal. Six decision tables corresponding to which side is closer and 
to the three boolean operators are used to determine an action from the classifications such as returning an intersect 
or advancing $t_{min}$ and intersecting again. 
% recursive to iterative
Recursive function calls are a natural way to process self similar structures such as CSG trees,  
however recursion is a memory expensive technique which makes it inappropriate for GPU usage.
Although NVIDIA OptiX supports recursive ray tracing in does not support recursion within intersect programs. 
The Opticks "evaluative" CSG implementation was inspired by the realization that CSG node tree intersection 
directly parallels binary expression tree evaluation and that techniques to simplify expression 
tree evaluation such as using postorder traversals could be applied. 
Binary expression trees are used to represent and evaluate mathematical expressions. 
A postorder traversal of a node tree visits every node in sequence such that
child nodes are visited before their parents.
Factoring out the postorder sequence allowed an iterative solution to be developed 
for a recursive problem.  

% coincidence issue
The CSG implementation relies on selection of the closer of two intersects at each level 
of the node tree, when faces of the constituent shapes coincide the ambiguity regarding which is closer
can cause spurious intersects. Modifing some constituents to prevent coincident faces avoids 
the issue without changing the intended geometry. As coincidences are rather common Opticks includes 
detection and automated fixing for some common situations.        
%
\subsection{Solid Serialization and Translation from Geant4 Solids}
\label{translation}
%
% serialization
Each primitive or operator node is serialized into an array of 16 32-bit elements. 
These elements include float parameters of the primitives and integer index references 
into a separate global transforms buffer. For the convex polyhedron primitive which is defined by a list of surface planes, 
the primitive contains an integer index referencing into a separate plane buffer together with the number of planes. 
A complete binary tree serialization, illustrated in Figure~\ref{tree},  with array indices matching level order tree indices
and zeros at missing nodes is used for the serialization of the CSG trees. This structure   
allows tree navigation directly from bitwise manipulations of the serialized array index. 
%
\begin{figure}[h]
\begin{verbatim}
                                                   depth     elevation

                     1                               0           3   

          10                   11                    1           2   

     100       101        110        111             2           1   
                    
 1000 1001  1010 1011  1100 1101  1110  1111         3           0   

parent(i) = i >> 1                 leftmost(height) = 1 << height 
leftchild(i) = i << 1               
rightchild(i) = (i << 1) + 1  
postorder(i,elevation) = i & 1 ? i >> 1 : (i << elevation) + (1 << elevation) 
\end{verbatim}

\caption{Height 3 complete binary tree, with nodes labelled with 1-based level order indices $i$ in binary. Postorder traversal of a binary tree
visits every node in sequence such that child nodes are visited before their parents. The simplicity of the complete binary tree data structure allows tree navigation 
from the bitwise manipulations of level indices. Total nodes in a complete binary tree of height $h$ is $2^{h+1} - 1$.  }
\label{tree}
\end{figure}

% balancing
Complete binary tree serialization is simple and effective for small trees but very inefficient 
for unbalanced trees necessitating tree balancing for shapes with many constituent primitives 
to reduce the tree height.  A two stage procedure to balance input trees was developed, 
first converting to a positive form tree by recursive application of De Morgan's laws replacing 
non-commutative differences with intersections with complemented sub-trees, $A - B  = A \cap !B $. 
The end result is a tree with only union and intersection operators and some complemented leaves. Intersections with 
complemented primitives, which correspond to "inside out" solids was implemented by flipping normals
and reclassifiying "Miss" as "Exit" as is not possible to miss the unbounded "otherside" of a 
complemented primitive.  Subsequently the commutative nature of intersect and union operators is used
to break apart the tree and reconstruct it in a more balanced form. 

% translation 
Opticks provides translations in both directions between Geant4 solids and Opticks primitives.
Depending on parameter values such as inner radii or phi segments there is not always a one to one correspondence 
between the two models with some Geant4 solids being represented as Opticks node trees. The Opticks 
approach of relying more on the CSG implementation was adopted to minimize duplicated code in the primitives.
%
\subsection{Geometry Structure Model and Direct Translation}
%
A major recent advance made possible by the general CSG solid intersection 
described above is the implementation of automated direct geometry translation 
allowing the Geant4 geometry model in memory to be directly translated and uploaded to the GPU.
The former approach to geometry translation described in~\cite{chep2016} required the
geometry to be exported and imported via GDML and G4DAE files. 

The Opticks geometry model revolves around the observation that many elements of a detector 
geometry are repeated demanding the use of instancing for efficient representation.  Geometry instancing 
is a technique used in computer graphics libraries including OpenGL and NVIDIA OptiX that avoids 
duplication of information on the GPU by storing repeated elements only once together with 4x4 transform matrices 
that specify the locations and orientations of each instance.

The Geant4 geometry model comprises a heirachy of volumes with associated transforms. The first step in 
the automated translation from Geant4 to the Opticks geometry is to traverse the Geant4 volume tree converting 
materials, surfaces, solids, volumes and sensors into Opticks equivalents. Each solid is assigned 
a boundary index uniquely identifying the combination of four indices representing outer and inner materials and outer and inner surfaces.
Outer/inner surfaces handle inwards/outwards going photons allowing the Geant4 border and skin
surface functionality to be translated. 

Subsequently repeated groups of volumes and their placement transforms are identified using node digests of progeny transforms and solid identities.
All structure nodes are assigned an instance index with geometry not passing instancing criteria regarding the number of repeats 
and geometry complexity forming the global group. The converted solids contain both analytic CSG node trees and triangulated meshes.   
These groups are used then used for the creation of the NVIDIA OptiX analytic geometry instances, and OpenGL mesh geometry instances.  

GPUs contain hardware dedicated to fast texture lookup and interpolation.
This is exploited by using a single 2D {\tt float4} texture named the boundary texture
that contains interleaved material and surface properties as a function of wavelength for all
unique boundaries.
The boundary index returned from a ray traced primitive intersection together with
an orientation offset identified from the angle between the geometric normal and ray direction
enables four wavelength interpolated material or surface properties to be
obtained from a single hardware optimized texture lookup.
%
\section{Interface between Geant4 and Opticks}
\label{simulation}
%
A single class {\tt G4Opticks} is used to provide a 
minimal interface between Geant4 user code and 
the Opticks package. 
At initialization the Geant4 top volume pointer is
passed to Opticks which translates the geometry and constructs 
the OptiX GPU context.

As Geant4 has no "genstep" interface it is necessary to modify the 
classes representing scintillation and Cerenkov processes.
Instead of generating photon secondary tracks in a loop the relevant "genstep" parameters, 
such as the number of photons to generate and the line segment along which to generate them, 
are collected. Collecting and copying gensteps rather than photons avoids allocation of CPU memory
for the photons, only photons that reach sensors requiring CPU memory allocation.
The generation loops of the scintillation and Cerenkov processes are ported to CUDA 
which effectively splits the photon generation implementation between the CPU and GPU.
Reemission is implemented by a fraction of photons absorbed within scintillators being 
reborn with a wavelength generated using an appropriate inverse cumulative distribution function.  

At the end of Geant4 event processing gensteps are uploaded to the GPU and an OptiX 
kernel is launched to generate and propagate the photons.
Further details on the CUDA port of the optical photon simulation can be found in reference~\cite{chep2016}.
The CUDA Thrust~\cite{thrust} C++ template library provides a high level interface to CUDA, 
greatly simplifying GPU development. Opticks uses Thrust for several tasks:
%
\begin{itemize}
\item associating photons with their gensteps entirely on the GPU
\item downloading photons that hit sensors to the CPU using the stream compaction technique
\item sorting photons by their step sequence histories, up to 16 steps of 4-bit flags are stored in 64 bit integers for each photon
\end{itemize} 
%
\section{Random Number Aligned Comparison of Opticks and Geant4}
\label{validation}
%
Validation comparisons use a single executable that performs both
the Geant4 and hybrid Opticks simulations and writes two events in a format
which includes highly compressed positions, times, wavelengths and polarizations at up to 16 steps of the optical photon propagations.
Opticks uses the cuRAND\cite{curandURL} library for concurrent generation of millions of reproducible sequences of pseudorandom numbers.  
Copying cuRAND sequences to the CPU and configuring the Geant4 random engine to use them makes it possible to align the 
consumption of random numbers between the two simulations, resulting in near prefectly matched results with every scatter, 
absorption and reflection happening with the same positions, times, wavelengths and polarizations.
Direct comparison of the aligned simulation results allows any discrepancies to be identified immediately.
The primary cause of differences are spurious intersects resulting from coincidences between the surfaces
of CSG constituents as described in section~\ref{intersection} which require geometry fixes.
%
\section{Summary and Outlook}
%
Opticks enables Geant4 based simulations to benefit from effectively zero time 
and zero CPU memory optical photon simulation, due to the massively parallel GPU 
processing made accessible by NVIDIA OptiX.
Recent Opticks developments enable automated translation of Geant4 geometries without approximation. 
This together with the adoption of modern CMake configuration techniques~\cite{BCM} and development of a 
minimal interface between Geant4 user code and embedded Opticks make getting started with a new geometry
drastically simpler than before.
Opticks can benefit any simulation limited by optical photons, the more limited the greater the 
benefit.
%
%
\section*{Acknowledgements}
%
The Daya Bay and JUNO collaborations are acknowledged for the use of detector 
geometries and simulation software. Dr. Tao Lin is acknowledged for his assistance with 
the JUNO Offline software. 
This work is currently funded by Chinese Academy of Sciences President’s International Fellowship Initiative. 
Grant No. 2018VMB0002.
%
%
%
\begin{thebibliography}{}
%
\bibitem{opticksURL}
Opticks URL {\tt https://bitbucket.org/simoncblyth/opticks/}

\bibitem{g4A}
Agostinelli S, Allison J, Amako K, Apostolakis J, Araujo H, Arce P, et al.
2003
Geant4--a simulation toolkit
{\it Nucl Instrum Methods Phys Res} A {\bf 506} pp 250-–303

\bibitem{g4B}
Allison J, Amako K, Apostolakis J, Araujo H, Dubois P, Asai M, et al.
2006
Geant4 developments and applications
{\it IEEE Trans Nucl Sci} {\bf 53} pp 270--8

\bibitem{g4C}
Allison J, Amako K, Apostolakis J, Arce P, Asai M, Aso T, et al.
2016
Recent developments in Geant4
{\it Nucl Instrum Methods Phys Res} A {\bf 835} pp 186--225

\bibitem{optixPaper}
Parker S, Bigler J, Dietrich A, Friedrich H, Hoberock J, et al.
%Luebke D, McAllister D, McGuire M, Morley K, Robison A and Stich M 
2010
OptiX: a general purpose ray tracing engine
{\it ACM Trans. Graph. : Conf. Series} {\bf 29} p 66

\bibitem{optixSite}
NVIDIA{\textregistered} OptiX\texttrademark~ webpage {\tt https://developer.nvidia.com/optix}

\bibitem{optixDocs}
NVIDIA{\textregistered} OptiX\texttrademark~ API documentation {\tt http://raytracing-docs.nvidia.com/optix/index.html}

\bibitem{rtx}
NVIDIA RTX URL {\tt https://developer.nvidia.com/rtx}

\bibitem{chep2016}
Blyth Simon C 2017 J. Phys.: Conf. Ser. {\bf 898} 042001

\bibitem{dyb}
An F, et al.
2016
The detector system of the Daya Bay reactor neutrino experiment
{\it Nucl Instrum Methods} A {\bf 811} pp 133--161

%9
\bibitem{juno}
An F et al.
2016
Neutrino physics with JUNO
{\it J Phys G} {\bf 43} 030401


\bibitem{throughput}
M. Garland and D. B. Kirk, Understanding Throughput Oriented Architectures, 
{\it COMMUN ACM} Volume 53 Issue 11, November 2010, Pages 58-66  	


\bibitem{kensler}
A. Kensler, Ray Tracing CSG Objects Using Single Hit Intersections (2006), 
with corrections by author of XRT Raytracer {\tt http://xrt.wikidot.com/doc:csg}

\bibitem{thrust}
Bell N and Hoberock J 
2011
{\it Thrust: a Productivity-Oriented Library for CUDA}
(GPU Computing Gems Jade Edition) ed W W Hwu, Chapter 26

\bibitem{curandURL}
cuRAND URL {\tt http://docs.nvidia.com/cuda/curand/index.html}

\bibitem{BCM}
Boost CMake Modules URL {\tt https://github.com/simoncblyth/bcm}


\end{thebibliography}
\end{document}
