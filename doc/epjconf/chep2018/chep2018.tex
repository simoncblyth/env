\documentclass{webofc}
\usepackage[varg]{txfonts}
\newcommand{\comment}[1]{}

\begin{document}
%
\title{Opticks : GPU Optical Photon Simulation for\\ Particle Physics using NVIDIA\textregistered\ OptiX\texttrademark}

\author{\firstname{Simon} \lastname{Blyth}\inst{1}\fnsep\thanks{\email{simon.c.blyth@gmail.com}} }

\institute{Institute of High Energy Physics, CAS, Beijing, China.}

\abstract{%
Opticks is an open source project that integrates the NVIDIA OptiX 
GPU ray tracing engine with Geant4 toolkit based simulations.
Massive parallelism brings drastic performance improvements with  
optical photon simulation speedup expected to exceed 1000 times Geant4 
with workstation GPUs. 

Optical physics processes of scattering, absorption, reemission and 
boundary processes are implemented as CUDA OptiX programs based on the Geant4
implementations. Wavelength dependent material and surface properties as well as  
inverse cumulative distribution functions for reemission are interleaved into 
GPU textures providing fast interpolated property lookup or wavelength generation.
OptiX handles the creation and application of a choice of acceleration structures
such as boundary volume heirarchies and the transparent use of multiple GPUs. 

A major recent advance is the implementation of GPU ray tracing of  
complex constructive solid geometry shapes, enabling
automated translation of Geant4 geometries to the GPU without approximation.
Using common initial photons and random number sequences allows 
the Opticks and Geant4 simulations to be run point-by-point aligned.
Aligned running has reached near perfect equivalence with test geometries.
}
%
\maketitle
%
\section{Introduction}
\label{intro}
%
\comment{
The introduction should provide background that puts the manuscript into
context and allows readers outside the field to understand the purpose and
significance of the study. It should define the problem addressed and explain
why it is important.
}
%
Opticks\cite{opticksURL} enables Geant4\cite{g4A}\cite{g4B}\cite{g4C}
based optical photon simulations to benefit from high performance massively parallel GPU ray tracing 
made accessible by the NVIDIA\textregistered\ OptiX\texttrademark\ package\cite{optixPaper}\cite{optixSite}.
Recent Opticks developments enable automated direct translation of Geant4 geometries into a GPU optimized constructive
solid geometry (CSG) form, precisely matching the CPU geometry.

Monte Carlo simulations of optical photon propagation 
are used to develop models of diverse photon transport systems 
ranging from human tissue within medical imaging scanners 
to materials instrumented with photon detectors in neutrino 
and dark matter search experiments. 
Such simulations are the primary technique used to design, optimize 
and analyse complex detection systems. However the versatility 
of the Monte Carlo technique comes with computational and memory costs 
that become extreme when propagating large numbers of photons
using traditional sequential processing. Opticks aims to provide a 
parallel solution to this optical photon simulation problem.  Drastically 
improved optical photon simulation performance can be transformative 
to the design, operation and understanding of diverse optical systems.


\subsection{Importance of Cosmic Muons background to Neutrino Detectors}

Cosmic muon induced processes are crucial backgrounds for neutrino
detectors such as Daya Bay~\cite{dyb} and JUNO\cite{juno},
necessitating underground sites, water shields and muon veto systems.
Minimizing the dead time and dead volume that results from applying
a veto requires an understanding of the detector response to a muon.
Large simulated samples of muon events are crucial in order to
develop such an understanding.

The number of optical photons estimated to be produced by a muon of
typical energy 100 GeV crossing the JUNO scintillator is at the level of tens of millions.
Profiling the Geant4 toolkit based simulation shows that the optical photon propagation 
consumes more than 99\% of CPU time, and imposes severe memory constraints that have forced
the use of event splitting.  

As optical photons in neutrino detectors can be considered to be produced
by only the scintillation and Cerenkov processes and yield only hits
on photomultiplier tubes it is straightforward to integrate an
external optical photon simulation with a Geant4 simulation of all other particles.


\subsection{Throughput Oriented Processing}

Graphics Processing Units (GPUs) originally designed for rendering 
are now increasingly used for computing tasks with suitably parallel workloads.
Efficient use of GPUs requires problems with can be divided into 
many thousands of mostly independent parallel tasks.  Each task 
is executed within a separate thread, to maximize the number of simultaneous 
running threads the use of shared resources such as registers and shared 
memory and synchronization between threads must be minimized.

The compute characteristics of the simulation of optical photons
are well matched with the requirements for efficient GPU usage, 
with sufficient parallelism from the large numbers of photons, 
low register usage from the simplicity of the optical physics, 
and decoupled nature of the photons avoiding synchronization.

The most computationally demanding aspect of optical photon simulation 
arises from the calculation, at each step of the propagation, 
of intersection positions of rays representing photons with the geometry of the system.
This ray tracing limitation of optical photon simulation is shared 
with the synthesis of realistic images in computer graphics. Due to the many applications
of ray tracing in the advertising, design, games and film industries the computer graphics
community has continuously improved ray tracing techniques. The Turing GPU architecture 
introduced by NVIDIA in 2018 is marketed as the worlds first Ray-tracing GPU, with   
hardware "RT Cores" in every streaming multiprocessor (SM) dedicated to the 
acceleration of ray geometry intersection testing.
NVIDIA claims performance of more than 10 billion ray geometry intersections 
per second, which is a factor 10 more than possible with its Pascal architecture 
which performs the intersection in software. NVIDIA exposes this functionality 
to developers via its GPU libraries including the NVIDIA OptiX ray tracing engine
which Opticks is based upon.

GPU architectures are throughput-oriented\cite{throughput}, optimizing the total amount of work completed per unit time,
by using a structure of many simple processing units. This contrasts with CPUs which are essentially latency-oriented, 
minimizing the time elapsed between initiation and completion of a single task, using complicated techniques 
such as speculative execution, branch prediction and instruction reordering which all consume chip area.
GPUs generally do not use these complicated techniques, allowing more chip area to be dedicated to parallel compute, 
yielding more total computational throughput across all threads at the expense of slower single-thread execution.

GPU threads blocked while waiting to access memory are tolerated by using 
hardware multithreading to resume other unblocked threads, allowing latencies to be hidden assuming 
that there are sufficient parallel threads in flight. CPUs avoid latency by dedicating large amounts 
of chip area to caching systems. 

GPUs evolved to meet the needs of real-time computer graphics rendering images of millions of pixels from geometries
composed of millions of triangles, they are designed to execute literally billions of small programs per second.





    


\subsection{Structure of these proceedings}  

These proceedings highlight recent major changes to the geometry workflow of Opticks 
and to the validation approach while still providing a complete overview of how Opticks works. 
The prior proceedings\cite{chep2016} provide further details on related work, NVIDIA OptiX
and on the former export/import based geometry workflow.





\section{Through}
\label{sec-1}
For bibliography use \cite{RefJ}
\subsection{Subsection title}
\label{sec-2}
Don't forget to give each section, subsection, subsubsection, and
paragraph a unique label (see Sect.~\ref{sec-1}).

For one-column wide figures use syntax of figure~\ref{fig-1}
\begin{figure}[h]
% Use the relevant command for your figure-insertion program
% to insert the figure file.
\centering
\includegraphics[width=1cm,clip]{env/presentation/tiger}
\caption{Please write your figure caption here}
\label{fig-1}       % Give a unique label
\end{figure}

For two-column wide figures use syntax of figure~\ref{fig-2}
\begin{figure*}
\centering
% Use the relevant command for your figure-insertion program
% to insert the figure file. See example above.
% If not, use
\vspace*{5cm}       % Give the correct figure height in cm
\caption{Please write your figure caption here}
\label{fig-2}       % Give a unique label
\end{figure*}

For figure with sidecaption legend use syntax of figure
\begin{figure}
% Use the relevant command for your figure-insertion program
% to insert the figure file.
\centering
\sidecaption
\includegraphics[width=5cm,clip]{env/ok/dyb_raytrace_composite_cerenkov_half_half}
\caption{Please write your figure caption here}
\label{fig-3}       % Give a unique label
\end{figure}

For tables use syntax in table~\ref{tab-1}.
\begin{table}
\centering
\caption{Please write your table caption here}
\label{tab-1}       % Give a unique label
% For LaTeX tables you can use
\begin{tabular}{lll}
\hline
first & second & third  \\\hline
number & number & number \\
number & number & number \\\hline
\end{tabular}
% Or use
\vspace*{5cm}  % with the correct table height
\end{table}
%
%
%
%
%
\begin{thebibliography}{}


\bibitem{opticksURL}
Opticks URL {\tt https://bitbucket.org/simoncblyth/opticks/}

\bibitem{g4A}
Agostinelli S, Allison J, Amako K, Apostolakis J, Araujo H, Arce P, et al.
2003
Geant4--a simulation toolkit
{\it Nucl Instrum Methods Phys Res} A {\bf 506} pp 250-–303

\bibitem{g4B}
Allison J, Amako K, Apostolakis J, Araujo H, Dubois P, Asai M, et al.
2006
Geant4 developments and applications
{\it IEEE Trans Nucl Sci} {\bf 53} pp 270--8

\bibitem{g4C}
Allison J, Amako K, Apostolakis J, Arce P, Asai M, Aso T, et al.
2016
Recent developments in Geant4
{\it Nucl Instrum Methods Phys Res} A {\bf 835} pp 186--225

\bibitem{optixPaper}
Parker S, Bigler J, Dietrich A, Friedrich H, Hoberock J, et al.
%Luebke D, McAllister D, McGuire M, Morley K, Robison A and Stich M 
2010
OptiX: a general purpose ray tracing engine
{\it ACM Trans. Graph. : Conf. Series} {\bf 29} p 66

\bibitem{optixSite}
NVIDIA{\textregistered} OptiX\texttrademark~ webpage {\tt https://developer.nvidia.com/optix}

\bibitem{chep2016}
Blyth Simon C 2017 J. Phys.: Conf. Ser. {\bf 898} 042001

\bibitem{dyb}
An F, et al.
2016
The detector system of the Daya Bay reactor neutrino experiment
{\it Nucl Instrum Methods} A {\bf 811} pp 133--161

%9
\bibitem{juno}
An F et al.
2016
Neutrino physics with JUNO
{\it J Phys G} {\bf 43} 030401


\bibitem{throughput}
M. Garland and D. B. Kirk, Understanding Throughput Oriented Architectures, 
{\it COMMUN ACM} Volume 53 Issue 11, November 2010, Pages 58-66  	





\end{thebibliography}
\end{document}
