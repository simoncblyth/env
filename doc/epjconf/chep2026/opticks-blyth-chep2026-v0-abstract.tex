Opticks is an open source framework that accelerates Geant4 toolkit based
detector simulations by offloading the optical photon simulation to the GPU
using NVIDIA OptiX ray tracing and NVIDIA CUDA computation. Geant4 detector
geometries are auto-translated into mostly analytic Constructive Solid Geometry
forms, with only computationally demanding shapes like tori converted into
triangulated meshes.

Recent developments, driven by production experience with JUNO, focus on
enabling efficient simulation of highly energetic multi-muon events 
that can yield several billion photons per event. We introduce a reduced resource
simulation mode that cuts persistent storage per photon from 64 bytes down to
16 bytes, significantly lowering I/O overheads and data footprint.

To maximize throughput, we implemented on-device PMT hit merging using CUDA
Thrust. This process involves sort and reduction steps, where the merged hit
identity is constructed from a bitwise OR of the sensor identifier and arrival
time bucket. This novel method reduces download overheads and avoids slow
CPU-side processing, allowing the full GPU-acceleration benefits of Opticks to
be realized.

Performance measurements are presented, demonstrating the overall speedup
achieved using the various reduced resource and GPU hit merging modes. Finally,
we describe explorations of a server-client architecture for Opticks, aiming to
enable distributed deployment and resource sharing.


