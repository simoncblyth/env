\chapter {Data and Analysis}
\section {Dayabay Dry Run}

Dayabay AD Dry Run is to run ADs without GdLS/LS/MO.
The expected schedule of the Dayabay first pair of AD -- first two AD for one of the near sites --
is in October, 2009. The primary goal of the AD Dry Run will be a basic functionality and integration test of the AD and read out systems.

At the time of writing, the installation and assembly of IAV and OAV prototype are on going at Surface Assembly Building (SAB), Dayabay.
The experience to play around with the prototype will be useful for the installation of the first pair of AD.

Besides the functionality and integration test of the AD and read out systems, the dry run data can be studied offline
after the AD Dry Run. For example, studying of the Monte Carlo and dry detector optics.

The polish status of the first pair of OAV and IAV is summarized in Table \ref{tab:ADPolishStatus}.
Section \ref{sec:Polish} shows the polish status would not affect the photon yield of the ADs
very much for Wet Run. However, there is no GdLS/LS/MO for Dry Run. The index of refraction
differece between acrylics and air is large enough to reflect or scatter the optical photons
yielded within AD by acrylics. This may lead the dry run data to be unexpected.
The possible effect is still under study at the time of writing.


\begin{table}
\centering
\caption{The first pair of AD polish status}
\label{tab:ADPolishStatus}
\begin{tabular}{lcc}
\hline
 & OAV & IAV\\
\hline
\hline
AV No.1 &   Not fully polished (Aluminum oxide powder jet) &   Fully polished  \\
AV No.2 &   Not fully polished (Aluminum oxide powder jet) &   Polished to P2000   \\
\hline
\end{tabular}
\end{table}
%optical surface


%\section {Aberdeen Dry Run}


\section {Preliminary Aberdeen Calibration}

The installation of the Aberdeen neutron detector began on late February, 2009.
The calibration of the neutron detector is still on going at the time of writing.
The following shows the preliminary calibration data by Cobalt 60 placed in different
position within the neutron detector. The Cobalt 60 ADC spectrum is


\begin{equation}
\label{equ:co60Spectrum}
\mbox{Spectrum with source - Spectrum with no source}
\end{equation}


\begin{figure}
    \centering
    \includegraphics[width=0.6\textwidth]{co60_no_source.png}
    \caption
    [Calibration Aberdeen Neutron Detector. The ADC pedestals.]
    {Calibration Aberdeen Neutron Detector. The ADC pedestals.}
    \label{fig:co60_no_source.png}
    \end{figure}


\begin{figure}
    \centering
    \includegraphics[width=0.6\textwidth]{co60_ring1.png}
    \caption
    [Calibration Aberdeen Neutron Detector by Cobalt-60 in the position of ring 1]
    {Calibration Aberdeen Neutron Detector by Cobalt-60 in the position of ring 1}
    \label{fig:co60_ring1.png}
    \end{figure}


\begin{figure}
    \centering
    \includegraphics[width=0.5\textwidth]{co60_ring1_simulation.png}
    \caption
    [Aberdeen Neutron Detector Simulation by Cobalt-60 in the position of ring 1]
    {Aberdeen Neutron Detector Simulation by Cobalt-60 in the position of ring 1}
    \label{fig:co60_ring1_simulation}
    \end{figure}





\begin{figure}
    \centering
    \includegraphics[width=0.6\textwidth]{co60_ring2.png}
    \caption
    [Calibration Aberdeen Neutron Detector by Cobalt-60 in the position of ring 2]
    {Calibration Aberdeen Neutron Detector by Cobalt-60 in the position of ring 2}
    \label{fig:co60_ring2.png}
    \end{figure}


\begin{figure}
    \centering
    \includegraphics[width=0.5\textwidth]{co60_ring2_simulation.png}
    \caption
    [Aberdeen Neutron Detector Simulation by Cobalt-60 in the position of ring 2]
    {Aberdeen Neutron Detector Simulation by Cobalt-60 in the position of ring 2}
    \label{fig:co60_ring2_simulation}
    \end{figure}



\begin{figure}
    \centering
    \includegraphics[width=0.6\textwidth]{co60_center.png}
    \caption
    [Calibration Aberdeen Neutron Detector by Cobalt-60 in the detector center]
    {Calibration Aberdeen Neutron Detector by Cobalt-60 in the detector center}
    \label{fig:co60_ring1.png}
    \end{figure}

\begin{figure}
    \centering
    \includegraphics[width=0.5\textwidth]{co60_center_simulation.png}
    \caption
    [Aberdeen Neutron Detector Simulation by Cobalt-60 in the detector center]
    {Aberdeen Neutron Detector Simulation by Cobalt-60 in the detector center}
    \label{fig:co60_center_simulation}
    \end{figure}


\begin{figure}
    \centering
    \includegraphics[width=0.6\textwidth]{co60_ring3.png}
    \caption
    [Calibration Aberdeen Neutron Detector by Cobalt-60 in the position of ring 3]
    {Calibration Aberdeen Neutron Detector by Cobalt-60 in the position of ring 3}
    \label{fig:co60_ring3.png}
    \end{figure}


\begin{figure}
    \centering
    \includegraphics[width=0.5\textwidth]{co60_ring3_simulation.png}
    \caption
    [Aberdeen Neutron Detector Simulation by Cobalt-60 in the position of ring 3]
    {Aberdeen Neutron Detector Simulation by Cobalt-60 in the position of ring 3}
    \label{fig:co60_ring3_simulation}
    \end{figure}




\begin{figure}
    \centering
    \includegraphics[width=0.6\textwidth]{abtRing4_co60_calibration_exp.png}
    \caption
    [Calibration Aberdeen Neutron Detector by Cobalt-60 in the position of ring 4]
    {Calibration Aberdeen Neutron Detector by Cobalt-60 in the position of ring 4}
    \label{fig:abtRing4_co60_calibration_exp.png}
    \end{figure}

\begin{figure}
    \centering
    \includegraphics[width=0.5\textwidth]{co60_ring4_simulation.png}
    \caption
    [Aberdeen Neutron Detector Simulation by Cobalt-60 in the position of ring 4]
    {Aberdeen Neutron Detector Simulation by Cobalt-60 in the position of ring 4}
    \label{fig:co60_ring4_simulation}
    \end{figure}


The preliminary comparison of data and simulation looks consistent.
The Aberdeen neutron detector is basically z-symmetric geometry, and the calibration by
source in the certer of the ring 1 and ring 4 should be similar.
However, Figure \ref{fig:co60_ring1.png} and Figure \ref{fig:co60_ring4_simulation} show asymmetry spectrum.
At the time of writing, the asymmetric effect is still under studying.
Several possible factors to lead to such effect could be, for example, the uncertainty of the calibration source position,
or the so-called "no source" is not well-define because the calibration source is still near by the neutron detector.


%
%\begin{figure}
%    \centering
%    \includegraphics[width=0.6\textwidth]{./figure/abt/ring4_co60_calibration_sim.png}
%    \caption
%    [Simulation of the calibration Aberdeen Neutron Detector by Cobalt-60 in the position of ring 4]
%    {Simulation of the calibration Aberdeen Neutron Detector by Cobalt-60 in the position of ring 4}
%    \label{fig:ring4_co60_calibration_sim.png}
%    \end{figure}
