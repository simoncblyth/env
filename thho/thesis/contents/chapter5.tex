\chapter {Event Reconstruction}
\section {Event Reconstruction of Aberdeen Neutron Detector}
There are three kinds of reconstruction were developed, center of mass,
energy pattern, and genetic algorithm(GA).

The center of mass algorithm doesn't consider the diffuse reflectors.
The energy pattern gives up to reconstruct vertices.
The genetic algorithm is not related to the geometry property, for example,
diffuse reflectors are used or not. But GA needs careful calibration process
to build a well-behavior response matrix and takes longer time to reconstruct
the events of Aberdeen. It takes around 3 seconds each event, and of course, deponding
on the CPU speed.


(The center of mass algorithm)




(The energy pattern algorithm)




(The GA algorithm scheme)
\begin{figure}[h]
    \centering
    \includegraphics[width=0.8\textwidth]{./figure/ga_flow_chart.png}
    \caption{Genetic algorithm for the reconstruction of Aberdeen neutron detector}
    \label{ga_flow_chart.png}
    \end{figure}

The online reconstruction is not decided, but the GA is inteded for
the offline. The preliminary test of a inver-square generator of Aberdeen
detector geometry is shown below. The fitness definition of this GA is


(fitness definition)
\begin{equation}
\label{fitness}
Fitness = 100 - \chi^{2}/NDF,
\end{equation}


where


\begin{equation}
\label{fitnessNDF}
NDF(degree of freedom) = 4,
\end{equation}


and


\begin{equation}
\label{fitnessChi}
\chi^{2} = (\mathbf{P}_{obs} - \mathbf{P}_{cal})^T\mathbf{Q}^T\Sigma^{-1}\mathbf{Q}(\mathbf{P}_{obs} - \mathbf{P}_{cal}).
\end{equation}


(raw data of the inver-square, energy resolution)




(reconstruction result of the GA)


A chi-square definition to be the fitness definition is also tested.
It shows the previous fitness definition works better on the
energy resolution.


(fitness definition)




(raw data of the inver-square, energy resolution)




(reconstruction result of the GA)



\begin{figure}
    \centering
    \includegraphics[width=0.8\textwidth]{./figure/nd1_distance_diff.png}
    \caption{Difference in actual and reconstruction distance}
    \label{fig:nd1_distance_diff.png}
    \end{figure}


\begin{figure}
    \centering
    \includegraphics[width=0.8\textwidth]{./figure/nd1_distance_x_diff.png}
    \caption{Difference in actual and reconstruction x-coordinate}
    \label{fig:nd1_distance_x_diff.png}
    \end{figure}


\begin{figure}
    \centering
    \includegraphics[width=0.8\textwidth]{./figure/nd1_distance_y_diff.png}
    \caption{Difference in actual and reconstruction z-coordinate}
    \label{fig:nd1_distance_y_diff.png}
    \end{figure}


\begin{figure}
    \centering
    \includegraphics[width=0.8\textwidth]{./figure/nd1_distance_z_diff.png}
    \caption{Difference in actual and reconstruction z-coordinate}
    \label{fig:nd1_distance_z_diff.png}
    \end{figure}


\begin{figure}
    \centering
    \includegraphics[width=0.8\textwidth]{./figure/nd1_photon_no_diff.png}
    \caption{Difference in actual and reconstruction number of photons}
    \label{fig:nd1_photon_no_diff.png}
    \end{figure}


\begin{figure}
    \centering
    \includegraphics[width=0.8\textwidth]{./figure/nd1_photon_no_diff_normal.png}
    \caption{Normalized difference in actual and reconstruction number of photons}
    \label{fig:nd1_photon_no_diff_normal.png}
    \end{figure}


\begin{figure}
    \centering
    \includegraphics[width=0.8\textwidth]{./figure/nd1_fitness.png}
    \caption{Fitness distribution}
    \label{fig:nd1_fitness.png}
    \end{figure}



More detailed test is done by Antony and Jimmy by G4dyb for simulation.
The result is shown below.




