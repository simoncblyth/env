\chapter {Event Reconstruction}
\section {Event Reconstruction of Aberdeen Neutron Detector}
\subsection {Overview}
There are three kinds of reconstruction were developed, center of mass,
energy pattern matching, and genetic algorithm(GA).

The center of mass algorithm doesn't consider the diffuse reflectors.
And the single photon spectrum of PMTs should be well-known to reconstruct the energy pattern.
Nevertheless, our R1408 PMTs for Aberdeen is old and hard to see the single photon spectrum clearly.
The calibration of the single photon is on going at the time of writing.
The preliminary center of mass spatial reconstruction is done by Wan.


\begin{equation}
\label{eq:reconstructionCM}
\Delta_{s}(\mathbf{r}) = {\sum}^N_{i=1} |   \frac{q_i}{\sum^N_{j=1}q_j}  -  \frac { \mu_{eff,i}(\mathbf{r}) }{ \sum^N_{j=1}\mu_{eff,i}(\mathbf{r}) }|
\end{equation}


The energy pattern matching is done by Long.
The energy pattern gives up to reconstruct vertices and prelimnary result shows
the reconstruction of enery resolution is around 4.2\%.


(The energy pattern algorithm)







The genetic algorithm is not related to the geometry property, for example,
diffuse reflectors are used or not. But GA needs careful calibration process
to build a well-behavior response matrix and takes longer time to reconstruct
the events of Aberdeen. It takes around 3 seconds each event, and of course, deponding
on the CPU speed. Jimmy suggested the GA to be used for Aberdeen.
This chapter focuses on the GA.











(The GA algorithm scheme)
\begin{figure}[h]
    \centering
    \includegraphics[width=0.8\textwidth]{./figure/ga_flow_chart.png}
    \caption{Genetic algorithm for the reconstruction of Aberdeen neutron detector}
    \label{ga_flow_chart.png}
    \end{figure}


The most advantage to use GA is that the geometry details could be changed anytime without modifying the codes,
and could be used for very complicated geometry without coding for that. The most disadvantage is the consuming
CPU because there are so many "useless" individuals would be rejected during the algorithm.
A diffusive reflector, tyvek which details could be found at Section \ref{sec:reflector}, is used for Aberdeen neutron detector to
be a radial reflector around the target. To quantize the extent of how diffusive the tyvek reflector is, would be a problem
if the reconstruction algorithm would like to use the information of the certer of mass.
Nervertheless, GA could skip such diffusive information.


\begin{figure}[h]
    \centering
    \includegraphics[width=0.8\textwidth]{./figure/abtNDRecon.png}
    \caption
    [1000 bins to construct the reconstruction pattern.]
    {1000 bins to construct the reconstruction pattern.}
    \label{fig:abtNDRecon.png}
    \end{figure}



There are many factors to decide the performance of the GA for Aberdeen.
For example, the quality of calibration data, the covariance matrix, the fitness definition,
the population size, the evolution time and generation number.
Basically the larger the population size, the better the reconstruction result is; the longer the evolution time, the better the result is;
the more the generation number, the better the result is. But all of these: large population size, long evolution time, and many generation
number consume the CPU time very much.
At the time of writing, the calibration of aberdeen neutron detector is still on going.
This chapter focuses on the discussion of the covariance matrix and the fitness definition.














\subsection{Fitness Definition}

The preliminary test of a inver-square generator of Aberdeen
detector geometry is shown below. A suggested fitness definition of this GA is


(fitness definition)
\begin{equation}
\label{fitness}
Fitness = 100 - \chi^{2}/NDF,
\end{equation}


where


\begin{equation}
\label{eq:fitnessNDF}
NDF(degree of freedom) = 4,
\end{equation}


and


\begin{equation}
\label{eq:fitnessChi}
\chi^{2} = (\mathbf{P}_{obs} - \mathbf{P}_{cal})^T\mathbf{Q}^T\Sigma^{-1}\mathbf{Q}(\mathbf{P}_{obs} - \mathbf{P}_{cal}).
\end{equation}



\begin{equation}
\label{eq:covariance1}
\mathbf{\Sigma}_{i,j} =
\left\{
    \begin{array}{ll}
    1 & \mbox{for } i=j, \\
    0 & \mbox{for } i{\neq}j.
    \end{array} \right.
\end{equation}



%
%where
%
%
%\begin{equation}
%\label{equ:fitnessQ1_1}
%\mathbf{Q}_{i,i} = 1 / \mathbf{P}_{obs,i}, \,\, if \, i<16
%\end{equation}
%
%
%and
%\begin{equation}
%\label{equ:fitnessQ1_2}
%\mathbf{Q}_{i,i} = 1 , \,\, if \, i>15
%\end{equation}
%

The typical reconstruction patterns are shown in Figure \ref{fig:f1c1_500.png} and Figure \ref{fig:f1c1_500.png}.
The simple inverse-square law event generator generates random photon numbers randomly distributed in the acrylic vessel.
The response matrix is generated based on the event which 10000 optical photons were generated.
Events that PMTs receive photons less than 500 are rejected. This is the same as what the electronic trigger does in reality.
The comparison of enery resolution before and after the reconstruction pattern is shown in Table \ref{tab:GAISSim} and Table \ref{tab:GACoorPhoton}.



\begin{table}
\centering
\caption{Energy resolution before and after the GA reconstruction}
\label{tab:GAISSim}
\begin{tabular}{lcp{5.0cm}}
\hline
Item & Before reconstruction (\%) & After reconstruction (\%) \\
\hline
\hline
Energy resolution & 9.18 & 6.84 \\
\hline
\end{tabular}
\end{table}


\begin{table}
\centering
\caption[Summary of the actual and fitted coordinates and photon number]
{
This table shows the comparisons of fitted with actual values.
For coordinates, the fitted values minus the actual values.
For photon number, the fitted values is divided by the actual values.
}
\label{tab:GACoorPhoton}
\begin{tabular}{lccccc}
\hline
Item &  x-coordinate (mm) & y-coordinate (mm) & z-coordinate (mm) & Distance (mm) & photon number \\
\hline
\hline
Mean        &   -0.8295 &   -0.9153 &   0.5432  &   97.31   &   -0.02449    \\
RMS         &   57.35   &   58.4    &   67.66   &   42.53   &   0.06337     \\
\hline
\end{tabular}
\end{table}



(raw data of the inver-square, energy resolution)




(reconstruction result of the GA)


A chi-square definition to be the fitness definition is also tested.
It shows the previous fitness definition works better on the
energy resolution.


Figure \ref{fig:f1c1_500.png} shows the coordinates and distance of reconstruction.
The lines in x, y and z coordinate plottings shows how the target volume is divided into the bins.
In this case, a $10cm\times10cm\times10cm$ cells of cube is used for the target volume. The response matrix
is based on the way how the target is divided. The response matrix is built by simuating 10000 photons
generated in the center of each bin. Because the simulation events for the reconstruction is to generate
photons randomly and uniformly in the target. The events generated near the center of bins will be reconstructed
and converge to the reconstructed results quicker and easier than the events not generated near the centers.
The events not generated near the centers may or may not be regarded as the events near the centers by the algorithm,
but the events generated near the centers is likely to be regarded as the events near the centers.
Because it a $10cm\times10cm\times10cm$ cube, there are 10 lines in x, y and z coordinate respectively.
For the same reason, the space resolution is limited to be in the order of 10cm in each coordinate.

The reconstructed events near the center of the target volume have better space resolution.
This is because the ADC count variance generated in the bins near the center of the target is smaller
than the one in the bins not near the center of the target. Thus the ADC count of the previous case
matches the response matrix easier than the later one.

Figure \ref{fig:f1c1_500.png} shows the reconstruction energy--generated photon number-- pattern.
The more the generated photon number, the larger the variance of the reconstructed photo number.
If the fitted photon number is divided by the actual--simulation-- photon number, the ratio is
almost the same.
This is derived by the normalized term $\mathbf{Q}^T$ of the fitness definition \ref{eq:fitnessChi}.
The weighting of the fitness, the elements of the covariance matrix, is fixed. The TDC counts have upper limit
because the size of the target is finite. Nervertheless the energy, or say ADC counts, could be very large.
If there is no such normalized term $\mathbf{Q}^T$, the weighting of the covariance matrix would be negligible
if the generated photon number is very large.


%Table ------ shows the comparison of the algorithm with and without $\mathbf{Q}^T$. It's obvious the reconstruction fails
%when the photon number is large.


%and the relation of the TDC counts to
%the distance from vertices to PMTs is linear. However, The relation of the ADC counts to the distance from vertices is not linear.
%It's inverse-square.


On the other side, if the photon number is too small. There is data could not provide sufficient ADC counts.
The ADC ratio of PMT response to caculated value is relative small to the TDC term, and result in large
fitted-to-actual ratio variance of ADC. This is shown in Figure \ref{f1c1.png}.
In reality, this could be avoided by tuning electronic triggers.



\begin{figure}
    \centering
    \includegraphics[width=0.8\textwidth]{./figure/image_ND_GA/f1c1_500.png}
    \caption[Coordinate and energy reconstruction]
{
Coordinate and energy reconstruction pattern derived by the GA.
The pattern is cut by photon number of 500.
}
    \label{fig:f1c1_500.png}
    \end{figure}

\begin{figure}
    \centering
    \includegraphics[width=0.8\textwidth]{./figure/image_ND_GA/f1c1.png}
    \caption[Coordinate and enery reconstruction without threshold]
{
Coordinate and energy reconstruction pattern derived by the GA.
If the photon number is not enough, the fitting pattern is wrose.
}
    \label{fig:f1c1.png}
    \end{figure}



\begin{figure}
    \centering
    \includegraphics[width=0.8\textwidth]{./figure/image_ND_GA/f3c1_500.png}
    \caption
    [Coordinate and energy reconstruction by a non-normalized fitness]
    {Coordinate and energy reconstruction by a non-normalized fitness}
    \label{fig:f3c1_500.png}
    \end{figure}



\begin{figure}
    \centering
    \includegraphics[width=0.8\textwidth]{./figure/image_ND_GA/f3c1.png}
    \caption
    [Coordinate and energy reconstruction by a non-normalized fitness without thresold]
    {Coordinate and energy reconstruction by a non-normalized fitness without thresold}
    \label{fig:f3c1.png}
    \end{figure}






\begin{figure}
    \centering
    \includegraphics[width=0.8\textwidth]{./figure/image_ND_GA/f1c1_500proj.png}
    \caption
    [Projection of \ref{fig:f1c1_500.png}]
    {Projection of \ref{fig:f1c1_500.png}}
    \label{fig:f1c1_500proj.png}
    \end{figure}




\begin{figure}
    \centering
    \includegraphics[width=0.8\textwidth]{./figure/image_ND_GA/f3c1_500proj.png}
    \caption
    [Projection of \ref{fig:f3c1_500.png}]
    {Projection of \ref{fig:f3c1_500.png}}
    \label{fig:f3c1_500proj.png}
    \end{figure}















%
%\subsection{The Covariance Matrix}
%
%The covariance matrix plays a role to decide which information is more important in fitness definition \ref{eq:eq:fitnessChi}.
%The reconstruction of space is more directly related to the TDC term than ADC term
%because the distance between the vertices and the PMTs is easy to know by only the TDC counts.
%To reconstruct the photon number by ADC counts, the distance should be known first.
%
%Tabel------- shows the result to use different covariance matrix to reconstruct the events.
%A compromising method to build a covariance matrix is to use the bin near one of the 16 PMTs.
%Fill the bin by events generating fixed photon number randomly distributed in the bin.
%The resolution of ADC and TDC counts shows which one is more "realiable".
%The reason to choose the bin near a PMT because of the inverse-square of the light intensity
%result in the possible largest variance of ADC counts. The resolution TDC counts won't change
%very much in a bin either near a PMT or not because the time information associated to the
%distance information is linear.
%
%
\begin{equation}
\label{eq:covariance4}
\mathbf{\Sigma}_{i,j} =
\left\{
    \begin{array}{ll}
    1 & \mbox{for } i=j, \mbox{and } i\leq16, \\
    1000 & \mbox{for } i=j, \mbox{and } i>16, \\
    0 & \mbox{for } i \neq j
    \end{array} \right.
\end{equation}



\begin{equation}
\label{eq:covariance5}
\mathbf{\Sigma}_{i,j} =
\left\{
    \begin{array}{ll}
    1000 & \mbox{for } i=j, \mbox{and } i\leq16, \\
    1 & \mbox{for } i=j, \mbox{and } i>16, \\
    0 & \mbox{for } i \neq j
    \end{array} \right.
\end{equation}



\begin{figure}
    \centering
    \includegraphics[width=0.8\textwidth]{./figure/image_ND_GA/f1c4_500.png}
    \caption
    [Fitness definition as \ref{eq:fitnessChi} with covariance matrix \ref{eq:covariance4}]
    {Fitness definition as \ref{eq:fitnessChi} with covariance matrix \ref{eq:covariance4}}
    \label{fig:f1c4_500.png}
    \end{figure}


\begin{figure}
    \centering
    \includegraphics[width=0.8\textwidth]{./figure/image_ND_GA/f1c5_500.png}
    \caption
    [Fitness definition as \ref{eq:fitnessChi} with covariance matrix \ref{eq:covariance5}]
    {Fitness definition as \ref{eq:fitnessChi} with covariance matrix \ref{eq:covariance5}}
    \label{fig:f1c5_500.png}
    \end{figure}


\begin{figure}
    \centering
    \includegraphics[width=0.8\textwidth]{./figure/image_ND_GA/f3c4_500.png}
    \caption
    [Fitness definition as \ref{eq:fitnessChi} without $\mathbf{Q}$ with covariance matrix \ref{eq:covariance4}]
    {Fitness definition as \ref{eq:fitnessChi} without $\mathbf{Q}$ with covariance matrix \ref{eq:covariance4}}
    \label{fig:f3c4_500.png}
    \end{figure}

\begin{figure}
    \centering
    \includegraphics[width=0.8\textwidth]{./figure/image_ND_GA/f3c5_500.png}
    \caption
    [Fitness definition as \ref{eq:fitnessChi} without $\mathbf{Q}$ with covariance matrix \ref{eq:covariance5}]
    {Fitness definition as \ref{eq:fitnessChi} without $\mathbf{Q}$ with covariance matrix \ref{eq:covariance5}}
    \label{fig:f3c5_500.png}
    \end{figure}








However, for neutron reconstruction, because of the time delay, a time offset
for the TDC term is suggested by Jimmy.



\begin{equation}
\label{eq:fitnessChiTimeOffset}
\chi^{2} = (\mathbf{P}_{obs} - \mathbf{P}_{cal} - \mathbf{O})^T\mathbf{Q}^T\Sigma^{-1}\mathbf{Q}(\mathbf{P}_{obs} - \mathbf{P}_{cal} - \mathbf{O}).
\end{equation}

where

\begin{equation}
\label{eq:fitnessQTimeOffset}
\mathbf{Q}_{i,j}
= \left\{
    \begin{array}{lll}
        \frac{16} {  {\sum^{16}_{k=1}}  {\mathbf{P}_{obs,k}}  } & \mbox{for } i=j \mbox{ and } i\leq16, \mbox{(ADC terms)} \\
        1 & \mbox{for } i=j \mbox{ and } i>16, \mbox{(TDC terms)} \\
        0 & \mbox{for } i{\neq}j
    \end{array} \right.
\end{equation}




\begin{equation}
\label{eq:fitnessOTimeOffset}
\mathbf{O}_i =
\left\{
    \begin{array}{ll}
    0 & \mbox{for } i\leq16, \\
    min\{\mathbf{P}_{obj,j}{\mid}j\in[16,32]\} - min\{\mathbf{P}_{cal,j}{\mid}j\in[16,32]\}  & \mbox{for } i>16.
    \end{array} \right.
\end{equation}


More detailed test is done by Antony and Jimmy by G4dyb for simulation.
The preliminary result shows the reconstruction of neutron energy resolution is 4.47\%
and the ADC-sum energy resolution is 4.89\%.


The online reconstruction is not decided, but the GA is intended for
the offline.
