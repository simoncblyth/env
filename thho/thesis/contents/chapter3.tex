\chapter {Acrylic Vessel R\&D and Integration}
\label{chap:arylicVessel}

plenty of waffle...plenty of waffle...plenty of waffle...plenty of waffle...

plenty of waffle...plenty of waffle...plenty of waffle...


\section {Overview of A Acrylic Vessel of Dayabay and Aberdeen Experiment}

Both of Dayabay Experiment and Aberdeen Experiment use acrylic vessels to hold
the liquid scintillator. The requirement of the both acrylic vessels is the same.

From the point of view of material transmission, the engineering cast, and the containment, 
acrylics, PMMA, is proposed to be used for the vessels of the antineutrino detectors of Dayabay and the neutron detector of Aberdeen.

For engineering, acrylics is cheaper than glass which the transmittance is the highest attainable
especially in the blue and new UV spectral regions where the PMT maximum responsivity and light yield of GdLS/LS are.
Strengh and tensile modules of elasticity of acrylics are good.
It's safer to install acrylic vessels instead of glass vessels.
Because of the size of the vessels, their components must be easily joined, and acrylics
is tough and has almost infinite design flexibility.
Acrylics possess the requisite low levels of radioactive contamination.
The above considerations led to the selection of acrylic.


Karsten: characters of acrylic veseel DocDB
Tom: shipping etc.
Jeff: requirement of acrylic vesesel


\subsection {Acrylic Optical Property in Detectors}

The optical properties of acrylic exert a strong influence on how light is propagated
from a neutrino interaction point for Dayabay, or a neutron interaction point for Aberdeen,
within the Ls, GdLs, and mineral oil to the PMTs arranged on the barrel of the steel tank.

The signal region of both Dayabay detectors and Aberdeen detector
is around from 350nm to 450nm, namely blue region of visible light and near UV.
The signal region is decided by the PMT gain, the LS and the GdLS emitting spectrum and
the attenuation spectrum, the mineral oil attenuation spectrum, and
the acrylic panels spectrum. Fig. \ref{fig:SpectrumCoupling} shows the region.


\begin{figure}[h]
    \centering
    \includegraphics[width=0.8\textwidth]{./figure/LS_GdLS_Acrylic_PMT_coupling.png}
    \caption{LS, GdLS, acrylics ,and PMT QE sepctrum. The coupling indicates the signal spectrum is in the blue and near UV spectral regions}
    \label{fig:SpectrumCoupling}
    \end{figure}


A typical commercial acrylic panel is often doped UV absorber
to prevent aging of acrylic panel. This kind of UV absorber makes acrylic spectrum
UV-blocked and the typical cut-off trasmission is at around 350nm to 400nm,
which happens to signal region of optical photon.
Thus the acrylic panels used for the acrylic vessels should be chosen as
UV-transparent ones. Fig. \ref{UVTUVBAcrylics} shows the comparison of the transmittance of UV-transparent
and UV-blocked acrylics.


\begin{figure}[h]
    \centering
    \includegraphics[width=0.5\textwidth]{./figure/UVT_UVB_acrylics.png}
    \caption{Comparison of the transmitance of a 10mm thickness UV-transparent (blue curve) and UV-blocked (red curve) acrylics}
    \label{UVTUVBAcrylics}
    \end{figure}



optical transmission standard

Nervertheless, not only as-received acrylic material property, but also as-cast
acrylic material should be monitored. Three kinds of manufacture process happen to the acrylic panels.
They are polish, bonding and annealing, including heating the panels to shape them. Besides,
the aging of acrylic panels will be discussed too.

\subsubsection {Bonding and Aging}

IAVs are produced by Nakano Int., Taiwan. Nakano uses a kind of glue, which is basically
composed of PMMA, to bond the acrylic panels of IAV. The process to produce this kind of glue is a commercial
secret. Nakano claims the glue is pure PMMA. The glue needs a process so called UV curing. The bonding area
is compose of acrylc panels and glue, and exposured to UV lamps. To cure a 2mm thick bonding
area, a 10W UV lamp is needed to irradiate the area for around 2 days, depending on the manufacutre status of glue
and the thickness of acrylic panels. This kind of UV curing process is observed that it changes transmission of acrylic panels,
and both increasing and decreasing effect could happen, depending on different kinds of acrylic panel.


\begin{figure}[h]
    \centering
    \includegraphics[width=0.5\textwidth]{./figure/reflector_panel_UV_aging.png}
    \caption{Example of acrylic aging by UV lamps: these are the reflector samples}
    \label{ReflectorAging}
    \end{figure}



(IAV 10mm increasing)


(the spectrum of one of different UV lamps Nakano used. This is mainly used for IAV.)


Fig. IAV 10mm increasing shows the transmission will fit in with the QA in around 1 month, which is longer than typical
bonding time, 2~3 days. So the UV curing is allowed to produce IAVs.

A similar effect could be observed when acrylic panels are exposure to the sunshine like what
Fig. \ref{IAVPrototypeBarrelSunshineAging} shows. Fig shows different
acrylic panels exposured to the sunshine responds differently.


\begin{figure}[h]
    \centering
    \includegraphics[width=0.8\textwidth]{./figure/IAV_prototype_barrel_sunshine.png}
    \caption{Example of acrylic aging by the sunshine: this is the barrel sample of the IAV prototype.}
    \label{IAVPrototypeBarrelSunshineAging}
    \end{figure}

(Bryce, LJ, and my figure)


Nevertheless, the transmission measured on different date of acrylic panels without exposuing to the sunshine directly,
shows slightly variance in around a year, which is longer than typical acrylic vessel manufacture time.
Thus the acrylic vessels should be avoid to be exposured to the sunshine directly.


\subsubsection {Polish}

Besides bonding, polish affects optical behavior.
Finer polish enhance specular reflection and ensure that light
go more straightly when passing through acrylic panels, namely
reducing diffuce reflection. However, to approach very fine polish,
ingredients of some products so called scratch remover are unknown, and
ingredients of binder of finer aluminium oxide or other kinds of powder to polish
acrylic finer is unknown, too. The unknown ingredients may contaminate GdLS or LS, or
be radioactive and should not be used.

Fig----------  shows comparison of different
polish class. Acrylic transmission is obviously different in air but not in LAB.
Extent of polish only changes light path of incident beam but not affects total light intensity.
The image of a acrylic sheet with rough surface looks vague because diffuce reflection happens, not
incident beams are absorbed.



(setting of polish test, in LAB)




(result of air)
\begin{figure}[h]
    \centering
    \includegraphics[width=0.6\textwidth]{./figure/polishing_panels.png}
    \caption{Transmittance of acrylic samples}
    \label{polishing_panels.png}
    \end{figure}



\begin{figure}[h]
    \centering
    \includegraphics[width=0.6\textwidth]{./figure/polishing_panels_relative.png}
    \caption{Difference of the transmittance of acrylic samples in Fig. \ref{fig:polishing_panels.png}}
    \label{polishing_panels_relative.png}
    \end{figure}


(result of in LAB)
\begin{figure}[h]
    \centering
    \includegraphics[width=0.6\textwidth]{./figure/polishing_quartz_LAB.png}
    \caption{Quartz and LAB coupling transmittance. There is no acrylics.}
    \label{polishing_quartz_LAB.png}
    \end{figure}



(setting of different range of polish test)
\begin{figure}[h]
    \centering
    \includegraphics[width=0.6\textwidth]{./figure/polishing_2000_position1_and_2.png}
    \caption{Comparison of the class 2000 sample in position 1 and position 2}
    \label{polishing_2000_position1_and_2.png}
    \end{figure}



\begin{figure}[h]
    \centering
    \includegraphics[width=0.6\textwidth]{./figure/polishing_1000_position1.png}
    \caption{Comparision of the class 1000 sample in position 1 and the class 2000 sample in position1}
    \label{polishing_1000_position1.png}
    \end{figure}


\begin{figure}[h]
    \centering
    \includegraphics[width=0.6\textwidth]{./figure/polishing_1000_position2.png}
    \caption{Comparision of the class 1000 sample in position 2 and the class 2000 sample in position1}
    \label{polishing_1000_position2.png}
    \end{figure}

\begin{figure}[h]
    \centering
    \includegraphics[width=0.6\textwidth]{./figure/polishing_polish_position1.png}
    \caption{Comparison of the fully polish sample in position 1 and the class 2000 sample in position 1}
    \label{polishing_polish_position1.png}
    \end{figure}

\begin{figure}[h]
    \centering
    \includegraphics[width=0.6\textwidth]{./figure/polishing_polish_position2.png}
    \caption{Comparison of the fully polish sample in position 2 and the class 2000 sample in position 1}
    \label{polishing_polish_position2.png}
    \end{figure}


(result of different range shows the different)
%\begin{figure}[h]
%    \centering
%    \includegraphics[width=0.6\textwidth]{./figure/}
%    \caption{}
%    \label{}
%    \end{figure}

The 1st IAV has used the scratch remover and the abrasive with binder. The 2nd
IAV is decided not to use both of them before any further study of the ingredients
, and only using normal sand papers, which the abrasive is aluminium oxide generally. 


\subsection {Acrylic Optical Property for Simulations}

Both G4dyb and NuWa use GEANT4 as the physics simulation engine.
To simulate the optical property of photons, only two parameters are used
by GEANT4, the index of refraction and the attenuation length. These two
parameters are different from materials to materials and can be decided
by measurement only.
More will be discussed in Sec. \ref{sec:opticalModel}.

\subsection {Spectrometer and Integrating sphere}

The spectrometer, Perkin Elmer Lambda 650, with a integrating sphere is used for the QA/QC
of the acrylic optical property including
the transmission, reflection, index of refraction, and attenuation length.

Lambda 650 is a double beams spectrometer. One is reference beam and another is
sample beam. All transmission meausrement is ratio of sample beam intensity to
reference beam. Refernce beam and sample beam are frequently calibrated in air.
Fig. \ref{fig:transmittance_integrating_systematic_error_calibration.png} shows the scheme
to calibrate the integrating sphere.

This Lambda 650 has tungsten and deturian light source, and they cover UV/VIS/NIR
range. For Dayabay antineutrino detectors, the PMT/LS/GdLS coupling wavelength range
is around 350nm to 450nm. The QA/QC of the acrylic panels provides the spectrum
from 200nm to 800nm, that is, covers the sensitive wavelength range.

The incident beam of the Lambda 650 has around 30\%-70\% polarization.


(Perkin Elmer Lambda 650 optical component. copyright problem so I can't show here)


The


A integrating sphere can collect
    - too thick for acrylic
    - systematic error for transparent samples
        - modify by distance
    - light source spectrum and noise

\subsection{Basic Performance}

The detector for the UV/Vis range(up to 900nm) of Lambda 650 with a 60 mm integrating sphere
is a PMT and a PbS detector for the NUR range (860-2500nm). For the QA of Dayabay and Aberdeen acrylic vessels, the wavelength region is from
200nm to 800nm so only the PMT will be used.
Two radiation sources, a deuterium lamp (DL) and a halogen lamp (HL), cover the working wavelength range of the spectrumeter.
The 60mm integrating sphere has a diameter equal to 60mm and 19 mm diamter ports.
Due to the noise of the PMT and the stability of the
light sources, the stability of the baseline is around 0.5\%.
For the QA/QC region of acrylics, the stability of near UV and IR is wrose because
of the intensity of the light sources is lower than the other region.
Fig. \ref{fig:spectrometer_maintenace_scan_E.png} shows the energy spectrum of the light sources from 190nm to
860nm. Assume the stability of the light sources and the PMT is the same in
the region from near UV to IR, i.e. the noise of signals is the same. The higher
light intensity could decrease the ratio of the noise intensity to incident beam intensity to the noise,
so the baseline would be more stable. The baseline stability is shown in Fig. \ref{fig:spectrometer_baseline_stability.png}.
The baseline stability before and after cleaning the filter is shown in Fig. \ref{fig:spectrometer_maintenace_baseline.png}.


\begin{figure}[h]
    \centering
    \includegraphics[width=0.8\textwidth]{./figure/spectrometer_baseline_stability.png}
    \caption[The spectrometer baseline stability]
{It's obvious that the worse stability occurs in the region associated to the lower energy spectrum in Fig. \ref{fig:spectrometer_maintenace_scan_E.png}}
    \label{fig:spectrometer_baseline_stability.png}
    \end{figure}


\begin{figure}[h]
    \centering
    \includegraphics[width=0.8\textwidth]{./figure/spectrometer_maintenace_scan_E.png}
    \caption[The energy spectrum of the light source of Lambda 650]
{
After cleaning the filter, the energy intensity of the light source is higher than the one before cleaning in the wavelength from 300nm to 400nm.
}
    \label{fig:spectrometer_maintenace_scan_E.png}
    \end{figure}


\begin{figure}[h]
    \centering
    \includegraphics[width=0.8\textwidth]{./figure/spectrometer_maintenace_baseline.png}
    \caption[The spectrometer baseline stability before and after cleaing the filter]
{
The red curve shows the stability before the cleaing and the blue curve shows the stability after cleaning.
It's obvious the stability is better in the region associated to the better energy intesity in Fig. \ref{fig:spectrometer_maintenace_scan_E.png}
}
    \label{fig:spectrometer_maintenace_baseline.png}
    \end{figure}



The error may come from:

\subsection{Transmittance Measurement of Acrylics}

The transmittance measurement by a spectrometer in air is to measure the gross transmittance.
The beam will be reflected few times in a sample. A simple simulation by GEANT4 shows
an optical photon reflected around five times in a acrylic sample in Sec.-----------.

Acrylic panels used for Dayabay and Aberdeen acrylic vessels
generally thicker than around 10 cm except for the reflectors of Aberdeen
neutron detecor. However, this kind of thickness is too thick to
keep the incident and exit beam in the same path, and then
the beam probably will not collect by the detector properly or
the conditions are too different from the calibration. See Fig.------.
The thickness Perkin Elmer suggests is less than 3 mm.


(The thickness may make the beam not be collected by the detector properly)


(The thickness may vary the condition too much so the calibration fails)


Samples with different thicknesses are measured in different positions and
the result shown in Fig.------. The thicker the sample is, the variance of
measurement is. Transmittance meausred by a integrating sphere
are almost the largest one because the integrating sphere collects refraction light may
not be collected properly by the other measurement.

(scheme shows thickness effect of transmission measurement)
\begin{figure}[h]
    \centering
    \includegraphics[width=0.8\textwidth]{./figure/compartment_position_test.png}
    \caption{Different acrylic sample position in the compartment of Lambda650. From
 left to right: position A, position B, and position C.}
    \label{compartment_position_test.png}
    \end{figure}


\begin{figure}[h]
    \centering
    \includegraphics[width=0.8\textwidth]{./figure/compartment_transmittance_difference_effect.png}
    \caption{Transmittance to the transmittance measurement in position B. Green: measurement in position A, Blue:
 measurement in position C. Red: measurement by integrating sphere.}
    \label{compartment_transmittance_difference_effect.png}
    \end{figure}


So for acrylics in Dayabay and Aberdeen experiment, the transmittance measurement is
considerated to use a integrating sphere instead of the original detector.


(optical principle of a integrating sphere)
\begin{figure}[h]
    \centering
    \includegraphics[width=0.8\textwidth]{./figure/transmittance_integrating_systematic_error_calibration.png}
    \caption{Calibration of a integrating sphere by air.}
    \label{fig:transmittance_integrating_systematic_error_calibration.png}
    \end{figure}


Nevertheless, for transparent samples, the integrating sphere
may induce systematic increasing transmittance slightly when measuring tansmittance.
This is because the diffuse light reflected by the integrating
sphere, is reflected back to the integrating sphere by the sample.
Fig. \ref{fig:transmittance_integrating_systematic_error_measurement.png} demonstrates the mechanism to cause such error.


\begin{figure}[h]
    \centering
    \includegraphics[width=0.8\textwidth]{./figure/transmittance_integrating_systematic_error_measurement.png}
    \caption[Transmittance measurement of the integrating sphere error attributed by transparent sample.]
{The systematic error caused by measuring transmittance of a transparent sample with a integrating sphere.
Comparing to Fig. \ref{fig:transmittance_integrating_systematic_error_calibration.png}, the yellow beam reflected by the sample
should exit by the transmittance port as what it has done in calibration shown
in Fig. \ref{fig:transmittance_integrating_systematic_error_calibration.png}, but it is reflected by the transparent sample so
the measured transmittance should be considerated larger than true value.}
    \label{fig:transmittance_integrating_systematic_error_measurement.png}
    \end{figure}

The amount of this kind of error could be estimated. Assume the sample is acrylics. It's transparent and around 92\% transmittance
is expected. The port diamter is 19 mm and the diamter of integrating sphere is 60mm.
So the port-to-sphere area ratio is around ${19^2}\div({4\times60^2})\thickapprox2.5\%$.
The typical front-surface primary reflectance of acrylics is around 4\%, and the typical correction value of Spectralon is 99\%.
Thus the increasing is around (to second order) in the order of:


\begin{equation}
\label{eq:ISError}
92\%\times2.5\%\times4\%\times99\%\thicksim0.1%
\end{equation}


A compromising improvement to reduce such kind of systematic error are, for example,
to put the sample slightly away from the integrating sphere (Fig. \ref{fig:IS_improvement.png}), but not too much in
order not to collect the refration light improperly as the effect metioned above.The distance depends on the geometry--surface flatness and
the thickness-- of the acrylic sample.  A example of the comparison of different distance from
the integrating sphere to measure the transmittance is shown in Fig. \ref{fig:IS_distance_test.png}. The result shows for a acrylic panel
sample of thickness of around 1 cm, around 1 to 1.5 cm away from the integrating sphere is more reasonable.


\begin{figure}[h]
    \centering
    \includegraphics[width=0.8\textwidth]{./figure/IS_improvement.png}
    \caption{Improvement skill to measure transmittance by integrating sphere.}
    \label{fig:IS_improvement.png}
    \end{figure}


\begin{figure}[h]
    \centering
    \includegraphics[width=0.8\textwidth]{./figure/IS_distance_test.png}
    \caption[Example of transmittance measured in different distance from the integrating sphere.]
{
These data are subtracted by the transmittance measured by putting the acrylic samlpe adjacent to the integrating sphere.
Form top to bottom, they are the one around 0.7 cm away, the one around 1.6 cm away, the one around 2.5 cm away, the one around 4.5 cm away,
and the one larger than 10 cm--in the middle of the compartment--. The 0.7 cm one shows almost the same transmittance measurement as the baseline.
The others except the 1.6 cm one show almost the same transmittance measurement. The 1.6 cm one is in the middle of the 0.7 cm one and the others,
and is slightly lower than the baseline by around 0.3\%. This quantity is expected in \ref{eq:ISError}.
}
    \label{fig:IS_distance_test.png}
    \end{figure}


\subsection{Reflectance Measurement of Acrylics}

Uncertainty by the Spectrlon

The reflectance measured by a spectrometer is to measure the gross measurement. And the measured
reflectance is relative to the reflectance standard, the Spectralon.

Fig. \ref{fig:spectralon_0_8_degree.png} shows both the spectralon by $0^\circ$ incident angle (namely, normal. This will lead to the diffuse
reflectance measurement.) and $8^\circ$ (Specular and diffuse reflectance.).


\begin{figure}[h]
    \centering
    \includegraphics[width=0.8\textwidth]{./figure/spectralon_0_8_degree.png}
    \caption[Typical Spetralon material reflectance value to access Lambda 650 Spetralon correction]{Typical Spetralon material reflectance value to access Lambda 650 Spetralon correction. Both the two curves are similar. This shows the stability of Lambda650 and the extent of diffuse of the Spectralon calibration standard is good.}
    \label{fig:spectralon_0_8_degree.png}
    \end{figure}


The calibration is executed by the highly diffusive reflectance standard.
If the sample is highly specular, large error may occur because the uniformity of the light path.


\subsection{Sources of Error of Both Transmittance and Reflectance Measurement}

Besides the errorss metioned above for transmittance and reflectance measurement, the other possible kinds of sources of error, for example,
wall radiance uniformity error, port induced errors, stray light, and errors attributed to the reflectance standard are provided
for consideration.

The wall radiance uniformity error can occur due to seams, baffles or bright spots, and dirt on the wall surface
positioned in the field of view of the sphere detector. These variations depend on the reflectance
characteristics of each sphere detector, so the magnitude of the error cannot be quantified.
Perkin Elmer 60 mm integrating sphere accessories is claimed that the 60 mm integrating sphere is machined
directly from a block of Spectralon, cut in half and
re-assembled to insure a uniform surface. Compare 60mm with 150mm integrating sphere, the 60mm one exhibit low optical
attenuation and a high detector signal-to-noise ration at all wavelengths.

Port induced errors refer to no-uniformities produced by the locatio, size and shape
of the ports dispersed across the inner sphere surface. The perimeter along each
port is thinned and coated with Spectraflect to reduce scatter.

Stray light is defined as light detected by the sphere detector outside the programmed bandwidth of the spectrometer.
The potential sources of stray light in the accessory are from
room lighting, leaking through the accessory ild and on of the integrating sphere ports, or
through the two access holes to the baseplate thumbscrews and the dusts in the air scattering.
The impact of stray light error on measurements depends on the magnitude of the sample measurements.
The following simple technique provided by the Perkin Elmer guide can be used to locate a stray ligh source when operating with
the sphere accessories. Extinguish the laboratory room lighting and drive the instrument monochromator to 400 nm and leave the reflectance port open.
Then close the sample compartment and accessory lids. Shine a flashlight near all potential stray light sources while oberserving the live display in the
control software. Because the flashlight provides visible light, the instrument now is sensitive to 400 nm which is also visible light region. If there is stray light, the instrument will provide the information.

The actual reflectance of Spectralon varies from batches to barches, and can vary as much as 0.015 from a typical value, depending on the wavelength.
A Spectralon surface may degrade slowly over time, so regular calibration is needed on a yearly basis.


\subsection {Measurement of Refractive Index and Attenuation Length of Acrylics}
\label {sec:RTMethod}

The two parameters, index of refraction and attenuation length, are important
to describ the behavior of optical photons. Due to the simulation (see ref------------),
the calibration, and the dry run requirement. The index of refraction and attenuation length
of the GdLS, LS, mineral oil and acrylcs should be known because the signals of the detectors
are optical photons.

For acrylics,--------don't break--------------------

This method is based on a method proposed by Zwinkels $et. al.$ \cite{RTMethod} which
is to the overall performance of the Subdbury neutrino observatory (SNO),
but slightly different from the measurement facilities. In our case, we wanted to characterize the optical constants of a
transparent material--acrylics-- over a wavelength range from 200nm to 800nm. The Fresnel relationships connecting the front-surface
primary reflectance R and the internal transmittance T for normal incidence, with the optical constants $n$ and $\kappa$ at each wavelength $\lambda$
are:


\begin{equation}
\label{eq:FSR}
R = \frac{(n-1)^2 + \kappa^2}{(n+1)^2 + \kappa^2}
\end{equation}


\begin{equation}
\label{eq:IT}
T = exp(\frac{-4{\pi}{\kappa}d}{\lambda})
\end{equation}

The term $\frac{-4{\pi}{\kappa}}{\lambda}$ is often described by the absorption coefficient $\alpha$ giving the familiar Lambdert's law:

\begin{equation}
\label{eq:Lambdert}
T = exp(-{\alpha}d)
\end{equation}

For a plane-parallel sample, the relationship connecting the measured normal-incidence reflectance $R^*$ and $T^*$ by the
Perkin Elmer Lambda 650 with the 60 mm integrating sphere are:


\begin{equation}
\label{eq:TStar}
T^* = \frac{(1-R)^2T}{1-R^2T^2}
\end{equation}


\begin{equation}
\label{eq:RStar}
R^* = R(1+TT^*)
\end{equation}


This is the model for the analysis of the reflectance and transmittance measurement data.
A typical result could be found in Fig. \ref{fig:rt_10mm_result.png}.


\begin{figure}[p]
    \centering
    \includegraphics[width=0.8\textwidth]{./figure/rt_10mm_result.png}
    \caption[Result of RT method]
{A IAV No. 1 barrel acrylic sample of thickness of 10.18mm.
Top left: The indexes of refraction.
Top right: The absorption coefficient $alpha$.
Middle left: The attenuation length.
Middle right: Newton method solution status. 1 means finding a solution successfully and 0 means failing.
Bottom left: Subtract the caculated transmittance $T*$ from the RT model from the transmittance measurements, $T*$.
Bottom right: Subtract the caculated reflectance $R*$ from the RT model from the reflectance measurements, $R*$.
}
    \label{fig:rt_10mm_result.png}
    \end{figure}

The analysis algorithm is based on Newton method to find numerical solutions. It may fail
to find solutions. The possible reasons could be the uncertainty of the spectrometer.
For a transparent but thin sample, the transmittance measurement could show the attenuation
is larger than some value, and could not show  what is the exact value of the attenuation.
Most failed solutions are derived by the unreasonable attenuation lengths which are related to the transmittance measurements.
The index of refraction is mainly decided by the Eq. \ref{eq:FSR} for a tranparent sample--acrylics-- because
of the $\kappa$ is relatively small than $n$. So basically the derived $n$ is still reliable shown in Table \ref{tab:RTMethodResult}.


Schott company's  method

\begin{table}
\centering
\caption{Comparisons the index of refraction derived by RT method for IAV barrel acrylics}
\label{tab:RTMethodResult}
\begin{tabular}{lcccccc}
\hline
Wavelength  & RT Method & Unc. & SNO Acrylic & Unc. & Schott & Unc. \\
\hline
\hline
365 & 1.512 & 0.006 & 1.510 & Not Given & - & - \\
\hline
405 & 1.504 & 0.006 & 1.505 & Not Given & - & - \\
\hline
480 & 1.495 & 0.006 & - & - & 1.497 & 0.001 \\
\hline
546 & 1.491 & 0.006 & - & - & 1.492 & 0.001 \\
\hline
\end{tabular}
\end{table}

\section {Integration of Dayabay Acrylic Vessel}

\subsection {Cleaning, Transportation and Assembly}


\subsubsection {Cleaning}

IAV is the target of Dayabay antineutrino detector,
and the cleanliness is highly required to avoid any contaminant
for GdLS and possible radioactive sources like dusts in air, which
could be Th or ----------. Th and ----- could be radioactive sources potentially
by ----------decays to -------------.

IAV can not be opened anymore after manufacture, except the calibration ports.
Thus the cleaning process is suggested before the final bonding of bottom lid.
The proposed cleaning process is shown in Fig--------. The requirement to define
how clean is clean is shown below.


(how clean is clean)



A class 10000 meets the FED209a standard is adopted to be the level of the cleaning room.
The QA of the cleaning room is shown below.

(FED-STD-290E)

\begin{figure}[h]
    \centering
    \includegraphics[width=0.6\textwidth]{./figure/e209.jpg}
    \caption{Class level of FED-STD-290E definition.}
    \label{e209.jpg}
    \end{figure}





(QA of the cleaning room)




(Cleaning process scheme)
\begin{figure}[h]
    \centering
    \includegraphics[width=0.6\textwidth]{./figure/cleaning_facility_overview.png}
    \caption{Overview of the cleaing facilities.}
    \label{cleaning_facility_overview.png}
    \end{figure}

Alconox and deionized water are used to clean IAVs. Alconox is -------------------.
1\% Alconox solution.
The resistance of the deionized water should be greater than 18M$ohm$-cm.
To define the Alconox solution is removed by deionized water, conductivity of the
washdown water should be smaller than 1$mu$S/cm after cycle rinse.


(cycle rinse scheme)
\begin{figure}[h]
    \centering
    \includegraphics[width=0.6\textwidth]{./figure/cleaning_cycle_rinse_water.png}
    \caption{Cycle rinse scheme}
    \label{cleaning_cycle_rinse_water.png}
    \end{figure}

The 1$mu$S/cm conductivity is derived by measuring the conductivity of Alconox solution.
Alconox solution which is smaller than 1ppm shows conductivity smaller than around 1$mu$S.
Although the unit of conductivity meter, Hana --------, is 0.1 $mu$S, the minimum to detect
Alconox solution concentration down to around 1$mu$S because of the possible contaminant
in pipes, pumps, water tanks, and so forth. Also the carbon dioxide may dissolve in
deionized water jet. These will make the conductivity of the washdown water raise
even through the vessel surface has been clean and the cycle rinse water does not
carry any Alconox solution, dust or other dirt.


(Test of water and Alconox solution conductivity)
\begin{table}
\centering
\caption{Test of water and Alconox solution conductivity}
\label{tab:AlconoxConductivity}
\begin{tabular}{lcp{5.0cm}}
Alconox solution &   Conductivity($\mu$S-cm) \\
\hline
\hline
1.00\% &  N/A (too large?)\\
\hline
0.10\% &  N/A (too large?)\\
\hline
100ppm & 102.3\\
\hline
10ppm  & 13.1\\
\hline
1ppm   & 0.7\\
\hline
0.1ppm & 0.1\\
\hline
deionized water $\geq 18M\Omega-cm$ &0.0~0.1\\
\hline
water A &0.3\\
\hline
water B &0.8\\
\hline
water C &4.7\\
\hline
water D &~5.6\\
\hline
\end{tabular}
\end{table}


\subsubsection {Transportation}

After the production including final cleaning of IAV, the IAV will be
transported from Taiwan to Dayabay.
The IAV prototype was checked by the polarizers-------- and passed
the lifting test after tranportation to SAB, Dayabay.


(polarizer check of the IAV prototype at SAB)




(lifting test of the prototype at SAB)


Accroding to the experience of the IAV
prototype, the pack method is feasible shown in Fig -----------.


(Pack scheme)


The route for IAVNo1 and IAVNo2 is shown in Fig--------.
It will take around 1 or 2 week from Taiwan to Dayabay site.


(route)


More moniter devices were suggested to moniter the transportation
status. Table-------- summary the devices.


(moniter device for IAV transportation)


\subsubsection {Assembly}

IAV should be installed into OAV. A proposed AD installation
scheme is as Fig-------, including installation of IAV.


(AD installation scheme, Karsten slides)


From the point of installation view, the coaxiality of the top center calibration port and
the bottom pin is mostly cared. Installation could be bottom-to-top or top-to-bottom, deponding
on the coaxiality.


(bottom-to-top installation scheme and the error)




(top-to-bottom installation scheme and the error)



A proposed coaxiality QA measurement is shown in Fig--------. The uncertainty
of the QA measurement is less than 3mm.


(table of measurement uncertainty)





(QA of coaxiality measurement scheme)







\section {Manufactor of Aberdeen Reflector}
\label{sec:reflector}
