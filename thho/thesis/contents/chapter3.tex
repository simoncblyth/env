\chapter {Acrylic Vessel R\&D and Integration}

plenty of waffle...plenty of waffle...plenty of waffle...plenty of waffle...

plenty of waffle...plenty of waffle...plenty of waffle...


\section {Overview of A Acrylic Vessel of Dayabay and Aberdeen experiments}

From the point of view of material transmission, the engineering cast, and the containment, 
acrylics, PMMA, is proposed to be used for the vessels of the antineutrino detectors of Dayabay and the neutron detector of Aberdeen.

For engineering, acrylics is cheaper than glass which the transmittance is the highest attainable
especially in the blue and new UV spectral regions where the PMT maximum responsivity and light yield of GdLS/LS are.
Strengh and tensile modules of elasticity of acrylics are good.
It's safer to install acrylic vessels instead of glass vessels.
Because of the size of the vessels, their components must be easily joined, and acrylics
is tough and has almost infinite design flexibility.
Acrylics possess the requisite low levels of radioactive contamination.
The above considerations led to the selection of acrylic.


Karsten: characters of acrylic veseel DocDB
Tom: shipping etc.
Jeff: requirement of acrylic vesesel

\subsection {Acrylic Optical Property in Detectors}

The optical properties of acrylic exert a strong influence on how light is propagated
from a neutrino interaction point for Dayabay, or a neutron interaction point for Aberdeen,
within the Ls, GdLs, and mineral oil to the PMTs arranged on the barrel of the steel tank.

The signal region of both Dayabay detectors and Aberdeen detector
is around from 350nm to 450nm, namely blue region of visible light and near UV.
The signal region is decided by the PMT gain, the LS and the GdLS emitting spectrum and
the attenuation spectrum, the mineral oil attenuation spectrum, and
the acrylic panels spectrum. Fig. \ref{fig:SpectrumCoupling} shows the region.


\begin{figure}[h]
    \centering
    \includegraphics[width=0.8\textwidth]{./figure/LS_GdLS_Acrylic_PMT_coupling.png}
    \caption{LS, GdLS, acrylics ,and PMT QE sepctrum. The coupling indicates the signal spectrum is in the blue and near UV spectral regions}
    \label{fig:SpectrumCoupling}
    \end{figure}


A typical commercial acrylic panel is often doped UV absorber
to prevent aging of acrylic panel. This kind of UV absorber makes acrylic spectrum
UV-blocked and the typical cut-off trasmission is at around 350nm to 400nm,
which happens to signal region of optical photon.
Thus the acrylic panels used for the acrylic vessels should be chosen as
UV-transparent ones. Fig. \ref{UVTUVBAcrylics} shows the comparison of the transmittance of UV-transparent
and UV-blocked acrylics.


\begin{figure}[h]
    \centering
    \includegraphics[width=0.5\textwidth]{./figure/UVT_UVB_acrylics.png}
    \caption{Comparison of the transmitance of a 10mm thickness UV-transparent (blue curve) and UV-blocked (red curve) acrylics}
    \label{UVTUVBAcrylics}
    \end{figure}



optical transmission standard

Nervertheless, not only as-received acrylic material property, but also as-cast
acrylic material should be monitored. Three kinds of manufacture process happen to the acrylic panels.
They are polish, bonding and annealing, including heating the panels to shape them. Besides,
the aging of acrylic panels will be discussed too.

\subsubsection {Bonding and Aging}

IAVs are produced by Nakano Int., Taiwan. Nakano uses a kind of glue, which is basically
composed of PMMA, to bond the acrylic panels of IAV. The process to produce this kind of glue is a commercial
secret. Nakano claims the glue is pure PMMA. The glue needs a process so called UV curing. The bonding area
is compose of acrylc panels and glue, and exposured to UV lamps. To cure a 2mm thick bonding
area, a 10W UV lamp is needed to irradiate the area for around 2 days, depending on the manufacutre status of glue
and the thickness of acrylic panels. This kind of UV curing process is observed that it changes transmission of acrylic panels,
and both increasing and decreasing effect could happen, depending on different kinds of acrylic panel.


\begin{figure}[h]
    \centering
    \includegraphics[width=0.5\textwidth]{./figure/reflector_panel_UV_aging.png}
    \caption{Example of acrylic aging by UV lamps: these are the reflector samples}
    \label{ReflectorAging}
    \end{figure}



(IAV 10mm increasing)


(the spectrum of one of different UV lamps Nakano used. This is mainly used for IAV.)


Fig. IAV 10mm increasing shows the transmission will fit in with the QA in around 1 month, which is longer than typical
bonding time, 2~3 days. So the UV curing is allowed to produce IAVs.

A similar effect could be observed when acrylic panels are exposure to the sunshine like what
Fig. \ref{IAVPrototypeBarrelSunshineAging} shows. Fig shows different
acrylic panels exposured to the sunshine responds differently.


\begin{figure}[h]
    \centering
    \includegraphics[width=0.8\textwidth]{./figure/IAV_prototype_barrel_sunshine.png}
    \caption{Example of acrylic aging by the sunshine: this is the barrel sample of the IAV prototype.}
    \label{IAVPrototypeBarrelSunshineAging}
    \end{figure}

(Bryce, LJ, and my figure)


Nevertheless, the transmission measured on different date of acrylic panels without exposuing to the sunshine directly,
shows slightly variance in around a year, which is longer than typical acrylic vessel manufacture time.
Thus the acrylic vessels should be avoid to be exposured to the sunshine directly.


\subsubsection {Polish}

Besides bonding, polish affects optical behavior.
Finer polish enhance specular reflection and ensure that light
go more straightly when passing through acrylic panels, namely
reducing diffuce reflection. However, to approach very fine polish,
ingredients of some products so called scratch remover are unknown, and
ingredients of binder of finer aluminium oxide or other kinds of powder to polish
acrylic finer is unknown, too. The unknown ingredients may contaminate GdLS or LS, or
be radioactive and should not be used.

Fig----------  shows comparison of different
polish class. Acrylic transmission is obviously different in air but not in LAB.
Extent of polish only changes light path of incident beam but not affects total light intensity.
The image of a acrylic sheet with ralph surface looks vague because diffuce reflection happens, not
incident beams are absorbed.



(setting of polish test, in LAB)




(result of air)




(result of in LAB)




(setting of different range of polish test)




(result of different range shows the different)


The 1st IAV has used the scratch remover and the abrasive with binder. The 2nd
IAV is decided not to use both of them before any further study of the ingredients
, and only using normal sand papers, which the abrasive is aluminium oxide generally. 


\subsection {Acrylic Optical Property for Simulations}

Both G4dyb and NuWa use GEANT4 as the physics simulation engine.
To simulate the optical property of photons, only two parameters are used
by GEANT4, the index of refraction and the attenuation length. These two
parameters are different from materials to materials and can be decided
by measurement only.


\subsection {Spectrometer and Integrating sphere}

A spectrometer with a integrating sphere is used for the QA/QC
of the acrylic optical property including
the transmission, reflection, index of refraction, and attenuation length.
The spectrometer is Perkin Elmer Lambda 650 with the 60mm integrating sphere.

Lambda 650 is a double beams spectrometer. One is reference beam and another is
sample beam. All transmission meausrement is ratio of sample beam intensity to
reference beam. Refernce beam and sample beam are frequently calibrated in air.
Fig. \ref{transmittance_integrating_systematic_error_calibration.png} shows the scheme
to calibrate the integrating sphere.

This Lambda 650 has tungsten and deturian light source, and they cover UV/VIS/NIR
range. For Dayabay antineutrino detectors, the PMT/LS/GdLS coupling wavelength range
is around 350nm to 450nm. The QA/QC of the acrylic panels provides the spectrum
from 200nm to 800nm, that is, covers the sensitive wavelength range.

The outcoming light source of the Lambda 650 has around 30\%-70\% polarization.


(Perkin Elmer Lambda 650 optical component. copyright problem so I can't show here)


The


A integrating sphere can collect
    - too thick for acrylic
    - systematic error for transparent samples
        - modify by distance


(scheme shows thickness effect of transmission measurement)


\begin{figure}[h]
    \centering
    \includegraphics[width=0.8\textwidth]{./figure/compartment_position_test.png}
    \caption{Different acrylic sample position in the compartment of Lambda650. From
 left to right: position A, position B, and position C.}
    \label{compartment_position_test.png}
    \end{figure}


\begin{figure}[h]
    \centering
    \includegraphics[width=0.8\textwidth]{./figure/compartment_transmittance_difference_effect.png}
    \caption{Transmittance to the transmittance measurement in position B. Green: measurement in position A, Blue:
 measurement in position C. Red: measurement by integrating sphere.}
    \label{compartment_transmittance_difference_effect.png}
    \end{figure}


(optical principle of a integrating sphere)


\begin{figure}[h]
    \centering
    \includegraphics[width=0.8\textwidth]{./figure/transmittance_integrating_systematic_error_calibration.png}
    \caption{Calibration of a integrating sphere by air.}
    \label{transmittance_integrating_systematic_error_calibration.png}
    \end{figure}




\begin{figure}[h]
    \centering
    \includegraphics[width=0.8\textwidth]{./figure/transmittance_integrating_systematic_error_measurement.png}
    \caption{The systematic error caused by measuring transmittance of a transparent sample with a integrating sphere.}
    \label{transmittance_integrating_systematic_error_measurement.png}
    \end{figure}



\subsection {Measurement of Refractive Index and Attenuation Length of Acrylics}
\label {sec:RTMethod}


\section {Integration of Dayabay Acrylic Vessel}

\subsection {Cleaning, Transportation and Assembly}


\subsubsection {Cleaning}

IAV is the target of Dayabay antineutrino detector,
and the cleanliness is highly required to avoid any contaminant
for GdLS and possible radioactive sources like dusts in air, which
could be Th or ----------. Th and ----- could be radioactive sources potentially
by ----------decays to -------------.

IAV can not be opened anymore after manufacture, except the calibration ports.
Thus the cleaning process is suggested before the final bonding of bottom lid.
The proposed cleaning process is shown in Fig--------. The requirement to define
how clean is clean is shown below.


(how clean is clean)



A class 10000 meets the FED209a standard is adopted to be the level of the cleaning room.
The QA of the cleaning room is shown below.


(FED-STD-290E)

\begin{figure}[h]
    \centering
    \includegraphics[width=0.6\textwidth]{./figure/e209.jpg}
    \caption{Class level of FED-STD-290E definition.}
    \label{e209.jpg}
    \end{figure}





(QA of the cleaning room)




(Cleaning process scheme)


Alconox and deionized water are used to clean IAVs. Alconox is -------------------.
1\% Alconox solution.
The resistance of the deionized water should be greater than 18M$ohm$-cm.
To define the Alconox solution is removed by deionized water, conductivity of the
washdown water should be smaller than 1$mu$S/cm after cycle rinse.


(cycle rinse scheme)


The 1$mu$S/cm conductivity is derived by measuring the conductivity of Alconox solution.
Alconox solution which is smaller than 1ppm shows conductivity smaller than around 1$mu$S.
Although the unit of conductivity meter, Hana --------, is 0.1 $mu$S, the minimum to detect
Alconox solution concentration down to around 1$mu$S because of the possible contaminant
in pipes, pumps, water tanks, and so forth. Also the carbon dioxide may dissolve in
deionized water jet. These will make the conductivity of the washdown water raise
even through the vessel surface has been clean and the cycle rinse water does not
carry any Alconox solution, dust or other dirt.


(Test of monitor of water and Alconox solution conductivity)


\subsubsection {Transportation}

After the production including final cleaning of IAV, the IAV will be
transported from Taiwan to Dayabay.
The IAV prototype was checked by the polarizers-------- and passed
the lifting test after tranportation to SAB, Dayabay.


(polarizer check of the IAV prototype at SAB)




(lifting test of the prototype at SAB)


Accroding to the experience of the IAV
prototype, the pack method is feasible shown in Fig -----------.


(Pack scheme)


The route for IAVNo1 and IAVNo2 is shown in Fig--------.
It will take around 1 or 2 week from Taiwan to Dayabay site.


(route)


More moniter devices were suggested to moniter the transportation
status. Table-------- summary the devices.


(moniter device for IAV transportation)


\subsubsection {Assembly}

IAV should be installed into OAV. A proposed AD installation
scheme is as Fig-------, including installation of IAV.


(AD installation scheme, Karsten slides)


From the point of installation view, the coaxiality of the top center calibration port and
the bottom pin is mostly cared. Installation could be bottom-to-top or top-to-bottom, deponding
on the coaxiality.


(bottom-to-top installation scheme and the error)




(top-to-bottom installation scheme and the error)



A proposed coaxiality QA measurement is shown in Fig--------. The uncertainty
of the QA measurement is less than 3mm.


(table of measurement uncertainty)





(QA of coaxiality measurement scheme)







Aberdeen Reflector manufactor
