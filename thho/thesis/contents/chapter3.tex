\chapter {Acrylic Vessel R\&D and Integration}
\label{chap:arylicVessel}
\section {Overview of A Acrylic Vessel of Dayabay and Aberdeen Experiment}

Both of Dayabay Experiment and Aberdeen Experiment use acrylic vessels to hold
the liquid scintillator. The requirement of the both acrylic vessels is the same.

From the material transmission, the engineering cast, the radioactivity and the compatibility of LS point of view,
acrylics, PMMA, is proposed to be used for the vessels of the antineutrino detectors of Dayabay and the neutron detector of Aberdeen.

Form engineering point of view, acrylics is cheaper than glass which the optical transmittance is the highest attainable
especially in the blue and near UV spectral regions where the PMT has the maximum response for light yield of GdLS/LS.
Mechanical strengh and tensile modules of elasticity of acrylics are good.
It's safer to install acrylic vessels instead of glass vessels.
Because of the size of the vessels, their components must be easily joined or bonded, and acrylics
is tough and has almost infinite design flexibility.\cite{acrylic:handbook}
Acrylics possess the required low levels of radioactive contamination and the compatibility of LS.
The above considerations led to the selection of acrylics to be the material.

The summary of QA items of inner acrylic vessel of Dayabay AD is shown in
Table \ref{tab:IAVQA}. More details of the items will be discussed in the following and
the optical QA will be on main focus.
\begin{table}
\centering
\caption{QA items of inner acrylic vessel of Dayabay AD}
\label{tab:IAVQA}
\begin{tabular}{lp{5.0cm}p{5.0cm}}
\hline
Type & Item & Description \\
\hline
\hline
Optical & Transmittance of raw material in 200nm to 800 nm & Optical signal region in 350 nm to 500 nm \\
        & Index of refraction and attenuation of acrylics & GdLS/LS coupling optical property. Simulation parameters.\\
Mechanics & Stress concentration & Mechanics strength for installation the acrylic vessel. \\
Radioactive & Radioactive of raw material & IAV is target container. The background impact is large.\\
\hline
\end{tabular}
\end{table}




%Karsten: characters of acrylic veseel DocDB
%Tom: shipping etc.
%Jeff: requirement of acrylic vesesel

\section {Optical QA Instrument and Method}

In order to measure the basic optical property including
transmittance, reflectance, index of refraction and attenuation,
a spectrometer, Perkin Elmer Lambda 650 with 60 mm integrating sphere,
is used.


\subsection {Spectrometer and Integrating sphere}

The spectrometer, Perkin Elmer Lambda 650, with a integrating sphere is used for the QA/QC
of the acrylic optical property including
the transmission, reflection, index of refraction, and attenuation length.

Lambda 650 is a double beam spectrometer. One is reference beam and another is
sample beam. All transmission meausrement is the ratio of sample beam intensity relative to
reference beam. Refernce beam and sample beam are frequently calibrated in air.
%Fig. \ref{fig:transmittance_integrating_systematic_error_calibration.png} shows the scheme
%Figure \ref{fig:calibration1.png} shows the way to calibrate the instrument for transmittance
%and total reflectance measurement. Figure \ref{fig:calibration2.png} shows the way to calibrate
%the instrument for diffuse reflectance measurement.

This Lambda 650 has tungsten and deturian light source, and they cover UV/VIS/NIR
range. For Dayabay antineutrino detectors, the PMT/LS/GdLS coupling wavelength range
is around 350nm to 450nm. The QA/QC of the acrylic panels provides the spectrum
from 200nm to 800nm, that is, covers the sensitive wavelength range.

The incident beam of the Lambda 650 has around 30\%-70\% polarization.
To check the polarization of acrylic raw material, transmittance and reflectance of a sample
measured by two orientation which is orthogonal to each other.

%(Perkin Elmer Lambda 650 optical component. copyright problem so I can't show here)
%The
%A integrating sphere can collect
%    - too thick for acrylic
%    - systematic error for transparent samples
%        - modify by distance
%    - light source spectrum and noise


\begin{figure}
    \centering
    \includegraphics[width=0.8\textwidth]{calibration1.png}
    \caption
    [Calibration for transmittance and total reflectance measurement overview of the integrating sphere]
    {Calibration for transmittance and total reflectance measurement overview of the integrating sphere}
    \label{fig:calibration1.png}
    \end{figure}

\begin{figure}
    \centering
    \includegraphics[width=0.8\textwidth]{calibration2.png}
    \caption
    [Calibration for diffuse reflectance measurement overview of the integrating sphere]
    {Calibration for diffuse reflectance measurement overview of the integrating sphere}
    \label{fig:calibration2.png}
    \end{figure}



\subsection{Basic Performance}

The detector for the UV/Vis range(up to 900nm) of Lambda 650 with a 60 mm integrating sphere
is a PMT and a PbS detector for the NIR range (860-2500nm). For the QA of Dayabay and Aberdeen acrylic vessels, the wavelength region is from
200nm to 800nm so only the PMT will be used.
Two radiation sources, a deuterium lamp (DL) and a halogen lamp (HL), cover the working wavelength range of the spectrumeter.
The 60mm integrating sphere has a diameter equal to 60mm and 19 mm diamter ports.
Due to the noise of the PMT and the stability of the
light sources, the stability of the baseline is around 0.5\%.
For the QA/QC region of acrylics, the stability of near UV and IR is wrose because
of the intensity of the light sources is lower than the other region.
Figure \ref{fig:spectrometer_maintenace_scan_E.png} shows the energy spectrum of the light sources from 190nm to
860nm. Assume the stability of the light sources and the PMT is the same in
the region from near UV to IR, i.e. the noise of signals is the same. The higher
light intensity could decrease the ratio of the noise intensity to incident beam intensity to the noise,
so the baseline would be more stable. The baseline stability is shown in Fig. \ref{fig:spectrometer_baseline_stability.png}.
The baseline stability before and after cleaning the spectrometer filter is shown in Fig. \ref{fig:spectrometer_maintenace_baseline.png}.


\begin{figure}
    \centering
    \includegraphics[width=0.5\textwidth]{spectrometer_baseline_stability.png}
    \caption[The spectrometer baseline stability]
{It's obvious that the worse stability occurs in the region associated to the lower energy spectrum in Fig. \ref{fig:spectrometer_maintenace_scan_E.png}}
    \label{fig:spectrometer_baseline_stability.png}
    \end{figure}


\begin{figure}
    \centering
    \includegraphics[width=0.5\textwidth]{spectrometer_maintenace_scan_E.png}
    \caption[The energy spectrum of the light source of Lambda 650]
{
After cleaning the spectrometer filter,
the energy intensity of the light source is higher (the blue one) than the one before cleaning (the red one) over the wavelength from 300nm to 400nm.
}
    \label{fig:spectrometer_maintenace_scan_E.png}
    \end{figure}


\begin{figure}
    \centering
    \includegraphics[width=0.5\textwidth]{spectrometer_maintenace_baseline.png}
    \caption[The spectrometer baseline stability before and after cleaing the filter]
{
The red curve shows the stability before the cleaing and the blue curve shows the stability after cleaning.
It's obvious the stability is better in the region associated to the higher energy intesity in Fig. \ref{fig:spectrometer_maintenace_scan_E.png}
}
    \label{fig:spectrometer_maintenace_baseline.png}
    \end{figure}


%The error may come from:


\subsection{Transmittance Measurement of Acrylics}

The transmittance measurement by a spectrometer in air is to measure the gross transmittance.
The beam will be reflected few times in a sample.
%A simple simulation by GEANT4 shows
%an optical photon reflected around five times in a acrylic sample
%------------------------------------------------
%in Sec.-----------.
%-------------------------------------------------

Acrylic panels used for Dayabay and Aberdeen acrylic vessels
are generally about 10cm or thicker, except for the reflectors of Aberdeen
neutron detecor. However, this kind of thickness is too thick to
keep the incident and exit beam in the same path, and then
the beam probably will not collect by the detector properly or
the conditions are too different from the calibration, Figure \ref{fig:transmittance_integrating_systematic_error_calibration.png}.
The thickness that Perkin Elmer suggests is less than 3 mm.



\begin{figure}
    \centering
    \includegraphics[width=0.8\textwidth]{transmittance_integrating_systematic_error_calibration.png}
    \caption{Calibration of a integrating sphere by air.}
    \label{fig:transmittance_integrating_systematic_error_calibration.png}
    \end{figure}


%(The thickness may make the beam not be collected by the detector properly)
%(The thickness may vary the condition too much so the calibration fails)


Samples with different thicknesses are measured at different locations, in the sample compartment shown in Figure \ref{compartment_position_test.png},
and the result is shown in Figure \ref{compartment_transmittance_difference_effect.png}.
The result is also compared with the measurement by the 60 mm integrating sphere.
The thicker the sample is, the variance of
measurement is. Transmittance meausred by a integrating sphere
are almost the largest one because the integrating sphere collects refraction light may
not be collected properly by the other measurement.

\begin{figure}
    \centering
    \includegraphics[width=0.8\textwidth]{compartment_position_test.png}
    \caption{Different acrylic sample position in the compartment of Lambda650. From
 left to right: position A, position B, and position C.}
    \label{compartment_position_test.png}
    \end{figure}


\begin{figure}
    \centering
    \includegraphics[width=0.8\textwidth]{compartment_transmittance_difference_effect.png}
    \caption{Transmittance relative to the transmittance measurement in position B. Green: measurement in position A, Blue:
 measurement in position C. Red: measurement by integrating sphere.}
    \label{compartment_transmittance_difference_effect.png}
    \end{figure}


So for acrylics in Dayabay and Aberdeen experiment, the transmittance measurement is
considerated to use a integrating sphere instead of the original detector module with the compartment.

Nevertheless, for transparent samples, the integrating sphere
may induce systematic increasing transmittance slightly when measuring tansmittance.
This is because the diffuse light reflected by the integrating
sphere, is reflected back to the integrating sphere by the sample.
Figure \ref{fig:transmittance_integrating_systematic_error_measurement.png} and
demonstrates the mechanism to cause such error.

\begin{figure}
    \centering
    \includegraphics[width=0.8\textwidth]{transmittance_integrating_systematic_error_measurement.png}
    \caption[Transmittance measurement of the integrating sphere error attributed by transparent sample.]
{The systematic error caused by measuring transmittance of a transparent sample with a integrating sphere.
Comparing to Fig. \ref{fig:transmittance_integrating_systematic_error_calibration.png}, the yellow beam reflected by the sample
should exit by the transmittance port as what it has done in calibration shown
in Fig. \ref{fig:transmittance_integrating_systematic_error_calibration.png}, but it is reflected by the transparent sample so
the measured transmittance should be considerated larger than true value.}
    \label{fig:transmittance_integrating_systematic_error_measurement.png}
    \end{figure}

The amount of this kind of error could be estimated. Assume the sample is acrylics. It's transparent and around 92\% transmittance
is expected. The port diamter is 19 mm and the diamter of integrating sphere is 60mm.
So the port-to-sphere area ratio is around ${19^2}\div({4\times60^2})\thickapprox2.5\%$.
The typical front-surface primary reflectance of acrylics is around 4\%, and the typical correction value of Spectralon is 99\%.
Thus the increasing is around (to second order) in the order of:


\begin{equation}
\label{eq:ISError}
92\%\times2.5\%\times4\%\times99\%\thicksim0.1\%
\end{equation}


A compromising improvement to reduce such kind of systematic error are, for example,
to put the sample slightly away from the integrating sphere (Figure \ref{fig:IS_improvement.png}), but not too much in
order not to collect the refration light improperly as the effect metioned above.The distance depends on the geometry -- surface flatness and
the thickness -- of the acrylic sample.  A example of the comparison of different distance from
the integrating sphere to measure the transmittance is shown in Figure \ref{fig:IS_distance_test.png}. The result shows for a acrylic panel
sample of thickness of around 1 cm, around 1 to 1.5 cm away from the integrating sphere is more reasonable.


\begin{figure}
    \centering
    \includegraphics[width=0.8\textwidth]{IS_improvement.png}
    \caption{Improvement skill to measure transmittance by integrating sphere.}
    \label{fig:IS_improvement.png}
    \end{figure}


\begin{figure}
    \centering
    \includegraphics[width=0.6\textwidth]{IS_distance_test.png}
    \caption[Example of transmittance measured in different distance from the integrating sphere.]
{
These data are subtracted by the transmittance measured by placing the acrylic samlpe adjacent to the integrating sphere.
Form top to bottom, they are the one around 0.7 cm, 1.6 cm, 2.5 cm, 4.5 cm,
and the one larger than 10 cm -- in the middle of the compartment --. The 0.7 cm one shows almost the same transmittance measurement as the baseline.
The others except the 1.6 cm one show almost the same transmittance measurement. The 1.6 cm one is in the middle of the 0.7 cm one and the others,
and is slightly lower than the baseline by around 0.3\%. This quantity is expected in \ref{eq:ISError}.
}
    \label{fig:IS_distance_test.png}
    \end{figure}


The setup for transmittance measurement by the integrating sphere is shown in Figure \ref{fig:transmission.png}.
The distance between the sample ant the integrating sphere is not scaled. Please note the measured transmittance
is gross transmittan, which include multiple reflection and transmittion in the interface of the sample and air.
%See Figure ref{}.


\begin{figure}
    \centering
    \includegraphics[width=0.8\textwidth]{transmittance.png}
    \caption
    [Transmission measurement set up]
    {Transmission measurement set up. The distance between the sample ant the integrating sphere is not scaled.}
    \label{fig:transmission.png}
    \end{figure}


% scheme to show the so called gross transmittance and reflectance.


\subsection{Reflectance Measurement of Acrylics}

Perkin Elmer Lambda 650 with the 60 mm integrating sphere could measure
the total reflectance including specular and diffuse reflection by a near-normal ($8^\circ$) incident beam, and
the diffuse reflectance by a normal incident beam.
Figure \ref{fig:reflectance8d.png} and \ref{fig:reflectance0d.png} show the setup of the total reflectance and the diffuse only
reflectance respectively.
The reflectance measured by a spectrometer with a integrating sphere is to measure the gross reflecance including multiple reflection in samples.
And the measured reflectance is relative to the reflectance standard which is made of Spectralon, whose absolute reflectance is certified with
a relative uncertaianty less than 0.015 from a typical value. Spectralon is a kind of highly reflective and diffuse material manufactured by Labsphere Inc.
\cite{Labsphere}
% Spectrlon Int. ref
%Uncertainty by the Spectrlon
Because the reflectance measurement is relative to the standard reflectance material, Spetralon,
a correction could be approached by the absolute Spetralon reflectance data.
A typical Spetralon material reflctance correction of Perkin Elmer Lambda 650 is
shown in Figure \ref{fig:spectralon_0_8_degree.png}, derived by removing the standard
reflectance in the sample port. The reflectance variance of Spectralon material in the wavelength from 200nm to 800nm
is claimed not larger than 1.5\% from batch to batch
in the Perkin Elmer user guide \cite{SphereAccessories}.
Figure \ref{fig:spectralon_0_8_degree.png} shows both the Spectralon reflectance measured and corrected by the Perkin Elmer typical Spectralon value,
by $0^\circ$ incident angle, namely normal-incident, leading to the diffuse reflectance measurement,
and $8^\circ$, namely gross reflectance measurement including Specular and diffuse reflectance.
If the measurement was not correct by the typical Sepctralon value, the measurement is 100\% transmittance stability test
because this spectrometer is a double beam spectrometer.
After correcting the measurement by the typical Spectralon value, Figure \ref{fig:spectralon_0_8_degree.png} shows the typical Spectralon correction value.


\begin{figure}
    \centering
    \includegraphics[width=0.8\textwidth]{reflectance8d.png}
    \caption
    [Total reflectance measurement setup]
    {Total reflectance measurement setup}
    \label{fig:reflectance8d.png}
    \end{figure}


\begin{figure}
    \centering
    \includegraphics[width=0.8\textwidth]{reflectance0d.png}
    \caption
    [Diffuse reflectance measurement setup]
    {Diffuse reflectance measurement setup}
    \label{fig:reflectance0d.png}
    \end{figure}


\begin{figure}
    \centering
    \includegraphics[width=0.4\textwidth]{spectralon_0_8_degree.png}
    \caption[Typical Spetralon material reflectance value to access Lambda 650 Spetralon correction]{Typical Spetralon material reflectance value to access Lambda 650 Spetralon correction. Both the two curves are similar. This shows the stability of Lambda650 and the extent of diffuse of the Spectralon calibration standard is good.}
    \label{fig:spectralon_0_8_degree.png}
    \end{figure}


The calibration is executed by the highly diffusive reflectance standard.
If the sample is highly specular, large error may occur because the condition is too different from calibration.


\subsection{Sources of Error of Both Transmittance and Reflectance Measurement}

Besides the errors metioned above for transmittance and reflectance measurement, the other possible kinds of sources of error, for example,
wall radiance uniformity error, port induced errors, stray light, and errors attributed to the reflectance standard are provided
for consideration.

The wall radiance uniformity error can occur due to seams, baffles or bright spots, and dirt on the wall surface
positioned in the field of view of the sphere detector. These variations depend on the reflectance
characteristics of each sphere detector, so the magnitude of the error cannot be quantified.
Perkin Elmer 60 mm integrating sphere accessories is claimed that the 60 mm integrating sphere is machined
directly from a block of Spectralon, cut in half and
re-assembled to insure a uniform surface. Compare 60mm with 150mm integrating sphere, the 60mm one exhibit low optical
attenuation and a high detector signal-to-noise ration at all wavelengths.

Port induced errors refer to no-uniformities produced by the locatio, size and shape
of the ports dispersed across the inner sphere surface. The perimeter along each
port is thinned and coated with Spectraflect to reduce scatter.

Stray light is defined as light detected by the sphere detector outside the programmed bandwidth of the spectrometer.
The potential sources of stray light in the accessory are from
room lighting, leaking through the accessory ild and on of the integrating sphere ports, or
through the two access holes to the baseplate thumbscrews and the dusts in the air scattering.
The impact of stray light error on measurements depends on the magnitude of the sample measurements.
The following simple technique provided by the Perkin Elmer guide can be used to locate a stray ligh source when operating with
the sphere accessories. Extinguish the laboratory room lighting and drive the instrument monochromator to 400 nm and leave the reflectance port open.
Then close the sample compartment and accessory lids. Shine a flashlight near all potential stray light sources while oberserving the live display in the
control software. Because the flashlight provides visible light, the instrument now is sensitive to 400 nm which is also visible light region. If there is stray light, the instrument will provide the information.

The actual reflectance of Spectralon varies from batches to barches, and can vary as much as 0.015 from a typical value, depending on the wavelength.
A Spectralon surface may degrade slowly over time, so regular calibration is needed on a yearly basis.


\subsection {Measurement of Refractive Index and Attenuation Length of Acrylics}
\label {sec:RTMethod}

The two parameters, index of refraction and attenuation length, are important
to describe the behavior of optical photons in a medium.
Due to the simulation requirement ( Section \ref{sec:opticalModel}), the calibration, and the dry run requirement,
the index of refraction and attenuation length
of the GdLS, LS, mineral oil and acrylics should be known because the signals of the detectors
are optical photons.


%For acrylics,--------don't break--------------------, SNO paper


This method is based on a method proposed by Zwinkels $et. al.$ \cite{RTMethod} which
is to the overall performance of the Subdbury neutrino observatory (SNO),
but different from the measurement facilities and the algorithm to find numerical solutions.
In our case, to characterize the optical constants of a
transparent material -- acrylics -- over a wavelength range from 200nm to 800nm is needed.
The Fresnel relationships connecting the front-surface
primary reflectance R and the internal transmittance T for normal incidence, with the optical constants $n$ and $\kappa$ at each wavelength $\lambda$
are:


\begin{equation}
\label{eq:FSR}
R = \frac{(n-1)^2 + \kappa^2}{(n+1)^2 + \kappa^2}
\end{equation}


\begin{equation}
\label{eq:IT}
T = exp(\frac{-4{\pi}{\kappa}d}{\lambda})
\end{equation}

The term $\frac{-4{\pi}{\kappa}}{\lambda}$ is often described by the absorption coefficient $\alpha$ giving the familiar Lambdert's law:

\begin{equation}
\label{eq:Lambdert}
T = exp(-{\alpha}d)
\end{equation}

For a plane-parallel sample, the relationship connecting the measured normal-incidence reflectance $R^*$ and $T^*$ by the
Perkin Elmer Lambda 650 with the 60 mm integrating sphere are:


\begin{equation}
\label{eq:TStar}
T^* = \frac{(1-R)^2T}{1-R^2T^2}
\end{equation}


\begin{equation}
\label{eq:RStar}
R^* = R(1+TT^*)
\end{equation}


To model and analyze the measurement data, specular reflectance is used.
However the Lambda650 with the 60 mm integrating sphere can only access the total and diffuse reflectance.
The specular reflectance is expected to be:

\begin{equation}
\label{eq:specularR}
\mbox{Total reflectance } - \mbox{Diffuse reflectance}
\end{equation}

Equation \ref{eq:TStar} and \ref{eq:RStar} are the models for the analysis of the reflectance and transmittance measurement data.
Typical results could be found in Figure \ref{fig:rt_10mm_result.png} for the 10 mm sample of IAV No.1
and \ref{rt_15mm_result.png} for the 15 mm sample of IAV No.1.


\begin{figure}[p]
    \centering
    \includegraphics[width=0.8\textwidth]{rt_10mm_result.png}
    \caption[Result of RT method of 10 mm acrylic sample]
{A IAV No. 1 barrel acrylic sample of thickness of 10.113mm.
Top left: The indexes of refraction.
Top right: The absorption coefficient $alpha$.
Middle left: The attenuation length.
Middle right: Newton method solution status. 1 means finding a solution successfully and 0 means failing.
Bottom left: Subtract the caculated transmittance $T*$ from the RT model from the transmittance measurements, $T*$.
Bottom right: Subtract the caculated reflectance $R*$ from the RT model from the reflectance measurements, $R*$.
}
    \label{fig:rt_10mm_result.png}
    \end{figure}


\begin{figure}[p]
    \centering
    \includegraphics[width=0.8\textwidth]{rt_15mm_result.png}
    \caption[Result of RT method of 15 mm acrylic sample]
{A IAV No. 1 barrel acrylic sample of thickness of 15.150 mm.
Top left: The indexes of refraction.
Top right: The absorption coefficient $alpha$.
Middle left: The attenuation length.
Middle right: Newton method solution status. 1 means finding a solution successfully and 0 means failing.
Bottom left: Subtract the caculated transmittance $T*$ from the RT model from the transmittance measurements, $T*$.
Bottom right: Subtract the caculated reflectance $R*$ from the RT model from the reflectance measurements, $R*$.
}
    \label{fig:rt_15mm_result.png}
    \end{figure}

The analysis algorithm is based on Newton method to find numerical solutions.
The algorithm flow chart is shown in Figure \ref{fig:rt_algorithm_flow_chart.png}.
The Newton method  may fail to find solutions. One of the possible reasons is the uncertainty of the spectrometer.
For a transparent but thin sample, the transmittance measurement could show the attenuation
is larger than some value, and could not show what is the exact value of the attenuation.
The error bar of ${alpha}$ of Equation \ref{eq:Lambdert} is large enough to make
the possible ${alpha}$ equal to or smaller than zero, namely infinite attenuation length.

Most failed solutions are derived by the unreasonable large -- infinite or larger than 10 meters --
attenuation lengths which are related to the transmittance measurements.
However, the index of refraction is mainly decided by the Eq. \ref{eq:FSR} for a tranparent sample -- acrylics -- because
of the $\kappa$ is relatively small than $n$. So basically the derived $n$ is still reliable, and the comparison
of different measurement method shown in Table \ref{tab:RTMethodResult}.
Schott Int. provides the typical refraction index measurement by directly observing monochromatic beam refraction.


%-----------
%Schott company's  method
%and ref
%------------


\begin{figure}
    \centering
    \includegraphics[width=0.8\textwidth]{rt_algorithm_flow_chart.png}
    \caption
    [Flow chart to show how the algorithm to find numerical solutions of RT measurement.]
    {Flow chart to show how the algorithm to find numerical solutions of RT measurement.}
    \label{fig:rt_algorithm_flow_chart.png}
    \end{figure}


\begin{table}
\centering
\caption{Comparisons the index of refraction derived by RT method for IAV barrel acrylics}
\label{tab:RTMethodResult}
\begin{tabular}{lccccccc}
\hline
Wavelength (nm) & RT Method & Unc. & SNO Acrylic & Unc. & Schott & Unc. & Solution Status\\
\hline
\hline
365 & 1.507 & 0.006 & 1.510 & Not Given & - & - & Success\\
\hline
405 & 1.501 & 0.006 & 1.505 & Not Given & - & - & Success\\
\hline
480 & 1.494 & 0.006 & - & - & 1.497 & 0.001 & Success\\
\hline
546 & 1.490 & 0.006 & - & - & 1.492 & 0.001 & Failed by infinite attenuation length\\
\hline
\end{tabular}
\end{table}


\section {Acrylic Optical Property of AD and Neutron Detector}

The optical properties of acrylic exert a strong influence on how light is propagated
from a neutrino capture vertex for Dayabay, or a neutron capture vertex for Aberdeen,
within the LS, GdLS, and mineral oil to the PMTs arranged on the barrel of the steel tank.

The signal region of both Dayabay detectors and Aberdeen detector
is around from 350nm to 450nm, namely blue region of visible light and near UV.
The signal region is decided by the PMT gain, the LS and the GdLS emitting spectrum and
the attenuation spectrum, the mineral oil attenuation spectrum, and
the acrylic panels spectrum. Figure \ref{fig:SpectrumR5912} and \ref{fig:SpectrumLS} show the spectrum region.


\begin{figure}
    \label{fig:SpectrumR5912}
    \centering
    \includegraphics[width=0.4\textwidth]{SpectrumR5912.png}
    \caption
    [Hamamatsu R5912 8" PMT QE spectrum]
    {Hamamatsu R5912 8" PMT QE spectrum}
    \end{figure}


\begin{figure}
    \label{fig:SpectrumLS}
    \centering
    \includegraphics[width=0.4\textwidth]{SpectrumLS.png}
    \caption
    [Typical LS spectrum used fo Dayabay AD.]
    {Typical LS spectrum used fo Dayabay AD. This is not the final material.}
    \end{figure}


A typical commercial acrylic panel is often doped by UV absorber
to prevent aging of acrylic panel. This kind of UV absorber makes acrylic spectrum
UV-blocked and the typical cut-off trasmission is at wavelength from around 350nm to 400nm,
which happens to signal region of optical photon.
Thus the acrylic panels used for the acrylic vessels should be chosen as
UV-transparent ones. Fig. \ref{fig:UVTUVBAcrylics} shows the comparison of the transmittance of UV-transparent
and UV-blocked acrylics.


\begin{figure}
    \centering
    \includegraphics[width=0.4\textwidth]{UVT_UVB_acrylics.png}
    \caption{Comparison of the transmitance of a 10mm thickness UV-transparent (blue curve) and UV-blocked (red curve) acrylics}
    \label{fig:UVTUVBAcrylics}
    \end{figure}


%-----
%optical transmission standard required by Dayabay
%--------

Nervertheless, not only as-received acrylic material property, but also as-cast
acrylic material should be monitored. Three kinds of manufacture process happen to the acrylic panels.
They are polish, bonding and annealing, including heating the panels to shape them. Besides,
the aging of acrylic panels will be discussed too.


\subsection {Bonding and Aging}

IAVs are produced by Nakano Int. Co. in Taiwan. Nakano uses a kind of glue, which is basically
composed of PMMA, to bond the acrylic panels of IAV. The process to produce this kind of glue is a commercial
secret. Nakano claims the glue is pure PMMA. The glue needs a so-called UV curing process. The bonding area
is compose of acrylc panels and glue, and exposured to UV lamps at 365nm. To cure a 2 mm thick bonding
area, a 10W UV lamp is needed to irradiate the area for around 2 days, depending on the manufacutre status of glue
and the thickness of acrylic panels. This kind of UV curing process is observed that it changes transmission of acrylic panels,
and both increasing and decreasing effect could happen, depending on different kinds of acrylic panel.


\begin{figure}
    \centering
    \includegraphics[width=0.5\textwidth]{reflector_panel_UV_aging.png}
    \caption [Example of acrylic aging by UV lamps]
    {Example of acrylic aging by UV lamps: these are the reflector samples.\cite{DocDB:3691v1}}
    \label{ReflectorAging}
    \end{figure}



%table (IAV 10mm increasing)


%(the spectrum of one of different UV lamps Nakano used. This is mainly used for IAV.)


Table \ref{tab:UVLampTest}
shows the transmission will fit in with the QA in around 1 month, which is longer than typical
bonding time, 2 to 3 days. So the UV curing is allowed to produce IAVs.
The sample exposured to the UV lamp increase the transmission of few \%.
The UV lamp is the same as what Nakano uses and the wavelength is around 352 nm.

\begin{table}
\centering
\caption{352 nm UV Lamp Transmission Test. The sample is the acrylic sample 1. See \ref{tab:summaryDYBTransmittance}}
\label{tab:UVLampTest}
\begin{tabular}{lp{2.5cm}p{2.5cm}p{2.5cm}p{2.5cm}}
\hline
Wavelength &  0 hour & 3 days and 14 hours & 9 days and 18 hours & 20 days and 6 hours \\
\hline
\hline
No UV, 500nm        &92.640  &92.565  &92.734  &92.647 \\
UV – No UV, 500nm   &-0.037  &0.026   &-0.214  &0.423\\
No UV, 400nm        &92.413  &92.248  &92.457  &92.352\\
UV – No UV, 400nm   &-0.063  &0.002   &-0.335  &0.272\\
No UV, 380nm        &91.629  &91.424  &91.679  &91.734\\
UV – No UV, 380nm   &0.034   &0.507   &0.263   &0.720\\
No UV, 360nm        &90.044  &89.664  &89.979  &89.800\\
UV – No UV, 360nm   &-0.288  &1.949   &1.413   &2.002\\
No UV, 340nm        &87.26   &87.21   &87.19   &87.57\\
UV – No UV, 340nm   &0.11    &2.49    &2.64    &2.2\\

\hline
\end{tabular}
\end{table}


A similar effect could be observed when acrylic panels are exposure to the sunshine like what
Fig. \ref{IAVPrototypeBarrelSunshineAging} shows. Fig shows different
acrylic panels exposured to the sunshine responds differently.


\begin{figure}
    \centering
    \includegraphics[width=0.8\textwidth]{IAV_prototype_barrel_sunshine.png}
    \caption{Example of acrylic aging by the sunshine: this is the barrel sample of the IAV prototype.}
    \label{IAVPrototypeBarrelSunshineAging}
    \end{figure}


%(Bryce, LJ, and my figure)


The response of acrylic panels exposured to the sunshine is case by case.
%Figure \ref{%(Bryce, LJ, and my figure)} shows measurement of different panels exposure to the sunshine.
The transmission measured on different date of acrylic panels without exposuing to the sunshine directly,
shows slightly variance in around a year, which is longer than typical acrylic vessel manufacture time.
Thus the acrylic vessels should be avoid to be exposured to the sunshine directly.


\subsection {Polish}
\label{sec:Polish}

Besides bonding, polish affects optical behavior.
Finer polish enhance specular reflection and ensure that light
go more straightly when passing through acrylic panels, namely
reducing diffuce reflection. This is shown in
Figure \ref{fig:polish_distance_1000}, \ref{fig:fig:polish_distance_2000} and \ref{fig:fig:polish_distance_fully}.

%However, to approach very fine polish,
%ingredients of some products so called scratch remover are unknown, and
%ingredients of binder of finer aluminium oxide or other kinds of powder to polish
%acrylic finer is unknown, too. The unknown ingredients may contaminate GdLS or LS, or
%be radioactive and should not be used.
%
%Figure \ref{fig:polishing_panels} and \ref{fig:polishing_panels_relative} show comparison of different polish classes.
%Acrylic transmission is obviously different in air but not in LAB. See Figure \ref{fig:polishing_quartz_LAB}, \ref{fig:polishing_2000_position1_and_2},
%\ref{fig:polishing_1000_position1}, \ref{fig:polishing_1000_position2.png}, \ref{fig:polishing_polish_position1}and \ref{fig:polishing_polish_position2}.
%Extent of polish only changes light path of incident beam but not affects total light intensity.
%The image of a acrylic sheet with rough surface looks vague because diffuce reflection happens, not
%incident beams are absorbed. This is could be observed by Figure \ref{fig:polishing_panels} which shows the transmittance difference
%is small from panels to panels if the samples are close to the integrating sphere -- collecting most incident beam -- but the difference is large
%is there is around 49 mm away from the integrating sphere,
%shown in Figure \ref{fig:polish_distance_1000}, \ref{fig:polish_distance_2000} and \ref{fig:polish_distance_fully}.
%
%\begin{figure}
%    \label{fig:polishSetup_1}
%    \centering
%    \includegraphics[width=0.8\textwidth]{polishSetup_1.png}
%    \caption
%    [Setup 1 of the polish test]
%    {Setup 1 of the polish test}
%    \end{figure}
%
%
%\begin{figure}
%    \label{fig:polishSetup_2}
%    \centering
%    \includegraphics[width=0.8\textwidth]{polishSetup_2.png}
%    \caption
%    [Setup 2 of the polish test]
%    {Setup 2 of the polish test}
%    \end{figure}
%
%
%%\begin{figure}
%%    \label{fig:polishSetup_3.png}
%%    \centering
%%    \includegraphics[width=0.8\textwidth]{polishSetup_3.png}
%%    \caption
%%    [Setup 3 of the polish test]
%%    {Setup 3 of the polish test}
%%    \end{figure}
%%
%%
%%\begin{figure}
%%    \label{fig:polishSetup_4.png}
%%    \centering
%%    \includegraphics[width=0.8\textwidth]{polishSetup_4.png}
%%    \caption
%%    [Setup 4 of the polish test]
%%    {Setup 4 of the polish test}
%%    \end{figure}
%
%
%
%
%%(result of air)
%\begin{figure} 
%    \label{fig:polishing_panels}
%    \centering
%    \includegraphics[width=0.4\textwidth]{polishing_panels.png}
%    \caption
%    [Transmittance of acrylic samples in air]
%    {Transmittance of acrylic samples in air}
%    \end{figure}
%
%
%
%\begin{figure}
%    \centering
%    \includegraphics[width=0.4\textwidth]{polishing_panels_relative.png}
%    \caption{Difference of the transmittance of acrylic samples in Fig. \ref{fig:polishing_panels}}
%    \label{fig:polishing_panels_relative}
%    \end{figure}
%
%
%%(result of in LAB)
%\begin{figure}
%    \centering
%    \includegraphics[width=0.4\textwidth]{polishing_quartz_LAB.png}
%    \caption{Quartz and LAB coupling transmittance. There is no acrylics.}
%    \label{fig:polishing_quartz_LAB}
%    \end{figure}
%
%
%
%%(setting of different range of polish test)
%\begin{figure}
%    \centering
%    \includegraphics[width=0.6\textwidth]{polishing_2000_position1_and_2.png}
%    \caption{Comparison of the class 2000 sample in position 1 and position 2}
%    \label{fig:polishing_2000_position1_and_2}
%    \end{figure}
%
%
%
%\begin{figure}
%    \centering
%    \includegraphics[width=0.6\textwidth]{polishing_1000_position1.png}
%    \caption{Comparision of the class 1000 sample in position 1 and the class 2000 sample in position1}
%    \label{fig:polishing_1000_position1}
%    \end{figure}
%
%
%\begin{figure}
%    \centering
%    \includegraphics[width=0.6\textwidth]{polishing_1000_position2.png}
%    \caption{Comparision of the class 1000 sample in position 2 and the class 2000 sample in position1}
%    \label{fig:polishing_1000_position2}
%    \end{figure}
%
%\begin{figure}
%    \centering
%    \includegraphics[width=0.6\textwidth]{polishing_polish_position1.png}
%    \caption{Comparison of the fully polish sample in position 1 and the class 2000 sample in position 1}
%    \label{fig:polishing_polish_position1}
%    \end{figure}
%
%\begin{figure}
%    \centering
%    \includegraphics[width=0.6\textwidth]{polishing_polish_position2.png}
%    \caption{Comparison of the fully polish sample in position 2 and the class 2000 sample in position 1}
%    \label{fig:polishing_polish_position2}
%    \end{figure}

\begin{figure}
    \label{fig:polish_distance_1000}
    \centering
    \includegraphics[width=0.4\textwidth]{polish_distance_1000.png}
    \caption
    [Distance variance leads to different transmittance. P1000 polish.]
    {Distance variance leads to different transmittance. P1000 polish.
$Up:$ the sample is close to the integrating sphere.
$Down:$ the sample is 49 mm away from the integrating sphere.
The measurement is done in air.
}
    \end{figure}


\begin{figure}
    \label{fig:fig:polish_distance_2000}
    \centering
    \includegraphics[width=0.4\textwidth]{polish_distance_2000.png}
    \caption
    [Distance variance leads to different transmittance. P2000 polish.]
    {Distance variance leads to different transmittance. P2000 polish.
$Up:$ the sample is close to the integrating sphere.
$Down:$ the sample is 49 mm away from the integrating sphere.
}
    \end{figure}


\begin{figure}
    \label{fig:fig:polish_distance_fully}
    \centering
    \includegraphics[width=0.4\textwidth]{polish_distance_fully.png}
    \caption
    [Distance variance leads to different transmittance. Fully polished sample.]
    {Distance variance leads to different transmittance. Fully polished sample.
$Up:$ the sample is close to the integrating sphere.
$Down:$ the sample is 49 mm away from the integrating sphere.
}
    \end{figure}



%(result of different range shows the different)
%\begin{figure}
%    \centering
%    \includegraphics[width=0.6\textwidth]{./figure/}
%    \caption{}
%    \label{}
%    \end{figure}

%The 1st IAV has used the scratch remover and the abrasive with binder. The 2nd
%IAV is decided not to use both of them before any further study of the ingredients,
%and only using normal sand papers, which the abrasive is aluminium oxide generally. 


However, if the acrylics is soaked in LAB, which is the typical
solvent of GdLS/LS, the similar index of refraction of acrylics and LAB
would reduce the diffuse effect caused by the polished status.
Setup as Figure \ref{fig:setup_1_1}, \ref{fig:setup_1_2}, \ref{fig:setup_2_1.png} and \ref{fig:setup_1_2.png}
show the result as \ref{fig:setup_1_1_result}, \ref{fig:setup_1_2_result}, \ref{fig:setup_2_1_result.png} and \ref{fig:setup_1_2_result.png}.
Four kinds of polished status are tested: P1000 ( a specification of abrasive paper ),
P2000 ( a specification of abrasive paper and finer than P1000 ), very fine polishing, and raw material.
P1000 and P2000 polish status look dull obviously by naked eyes.
It shows the maximum difference of transmission is less than 3\%.
%That is the polishing uniformity of acrylic vessel is not very important.


\begin{figure}
    \label{fig:setup_1_1}
    \centering
    \includegraphics[width=0.6\textwidth]{setup_1_1.png}
    \caption
    [Polish status transmission test setup 1-1.]
    {Polish status transmission test setup 1-1.}
    \end{figure}

\begin{figure}
    \label{fig:setup_1_2}
    \centering
    \includegraphics[width=0.6\textwidth]{setup_1_2.png}
    \caption
    [Polish status transmission test setup 1-2.]
    {Polish status transmission test setup 1-2.}
    \end{figure}

\begin{figure}
    \label{fig:setup_2_1}
    \centering
    \includegraphics[width=0.6\textwidth]{setup_2_1.png}
    \caption
    [Polish status transmission test setup 2-1.]
    {Polish status transmission test setup 2-1.}
    \end{figure}

\begin{figure}
    \label{fig:setup_2_2}
    \centering
    \includegraphics[width=0.6\textwidth]{setup_2_2.png}
    \caption
    [Polish status transmission test setup 2-2.]
    {Polish status transmission test setup 2-2.}
    \end{figure}


\begin{figure}
    \label{fig:setup_1_1_result}
    \centering
    \includegraphics[width=0.6\textwidth]{setup_1_1_result.png}
    \caption
    [Result of polish status transmission test setup 1-1.]
    {Result of polish status transmission test setup 1-1.}
    \end{figure}

\begin{figure}
    \label{fig:setup_1_2_result}
    \centering
    \includegraphics[width=0.6\textwidth]{setup_1_2_result.png}
    \caption
    [Result of polish status transmission test setup 1-2.]
    {Result of polish status transmission test setup 1-2.}
    \end{figure}

\begin{figure}
    \label{fig:setup_2_1_result}
    \centering
    \includegraphics[width=0.6\textwidth]{setup_2_1_result.png}
    \caption
    [Result of polish status transmission test setup 2-1.]
    {Result of polish status transmission test setup 2-1.}
    \end{figure}

\begin{figure}
    \label{fig:setup_2_2_result}
    \centering
    \includegraphics[width=0.6\textwidth]{setup_2_2_result.png}
    \caption
    [Result of polish status transmission test setup 2-2.]
    {Result of polish status transmission test setup 2-2.}
    \end{figure}




\subsection {Acrylic Optical Property for Simulations}

Both G4dyb and NuWa use GEANT4 as the physics simulation engine.
To simulate the optical property of photons, only two parameters are used
by GEANT4, the index of refraction and the attenuation length. These two
parameters are different from materials to materials and can be decided
by measurement only.
More will be discussed in Sec. \ref{sec:opticalModel}.

\subsection {Summary of Optical Property of Dayabay and Aberdeen Acrylic Vessels}

The transmittance summary of Dayabay acrylic samples for IAV No.1 and No.2 is shown in Table \ref{tab:summaryDYBTransmittance}.
The transmittance summary of Aberdeen acrylic samples for Aberdeen is shown in Table \ref{tab:summaryABTTransmittance}.
Besides sample 5, there is another batch for the rib of Dayabay IAV No.1 and No.2 but it didn't
be recorded because the samples of the batch mixes with sample 5. However, the rest material
of the ribs were collected and the transmittance of those material is measured. They are all o.k.
to pass the transmittance requirement.
%Figure \ref{} shows the transmittance of the samples of the material.

\begin{table}
\centering
\caption{Transmittance Summary of Dayabay acrylic samples}
\label{tab:summaryDYBTransmittance}
\begin{tabular}{lp{1.5cm}p{1.5cm}p{1.5cm}p{1.5cm}p{1.5cm}}
\hline
Wavelength (nm) & DYB Sample1 & DYB Sample2 & DYB Sample3 & DYB Sample4 & DYB Sample5 \\
\hline
\hline
500 & 92.55 &  92.12 &  92.1  &  91.89 &  90.56 \\
400 & 92.25 &  91.52 &  91.64 &  91.33 &  88.38 \\
380 & 91.5  &  90.47 &  91.08 &  90.77 &  92.51 \\
360 & 89.65 &  97.46 &  89.85 &  89.12 &  62.98 \\
340 & 86.99 &  93.28 &  87.67 &  85.55 &  41.65 \\
320 & 83.89 &  77.95 &  84.2  &  79.55 &  25.67 \\
300 & 75.56 &  65.35 &  76.03 &  65.83 &  4.51  \\
\hline
\end{tabular}
\end{table}



\begin{table}
\centering
\caption{Transmittance Summary of Aberdeen Samples}
\label{tab:summaryABTTransmittance}
\begin{tabular}{lp{1.5cm}p{1.5cm}p{1.5cm}p{1.5cm}p{1.5cm}}
\hline
Wavelength (nm) & ABT Barrel  & ABT Bottom Lid & ABT Bottom Rib &  ABT Top Lid &  ABT Top Rib \\
\hline
\hline
500 & 92.55 &  92.18 &  92.22 &  92.27 &  92.17\\
400 & 92.15 &  91.55 &  91.64 &  91.74 &  91.53\\
380 & 91.32 &  90.86 &  90.28 &  91.2  &  90.18\\
360 & 89.34 &  89.03 &  86.59 &  89.7  &  86.16\\
340 & 86.04 &  84.23 &  80.73 &  86.07 &  80.12\\
320 & 81.52 &  74.91 &  73.8  &  78.7  &  73.42\\
300 & 67.05 &  44.48 &  57.18 &  58.17 &  57.03\\
\end{tabular}
\end{table}


%index of refraction and att summary



\subsection {Dayabay AD Geometry QA}


The coaxiality of IAV is directly related to the scheme to installation.
This will be discussed in Section \ref{sec:assembly}.
Figure \ref{fig:coaxiality} shows the proposal for IAV coaxiality measurement.
There is no a real and exact center because the acrylic is a flexible material and
the thickness uncertainty. For assembly and installation, a more practical information is the difference
distance of the center calibration port and the bottom block.
A proposal to measure this distance is as the following:
A template is used to define the center of the calibration port.
The calibration port, namely the flange is machined by CNC, and around 1 mm uncertainty might occur.
To define the center position of the calibration port,
1 mm uncertainty by CNC and 1 mm uncertainty by the line on the template occur.
To define the center of the bottom block, using the middle of each ribs, total 8 ribs, choose
the point where most lines concentrate to be the so-called center.
This may raise 1 mm uncertainty because of CNC and
another 1 mm uncertainty because of the lines.
The uncertainty of level is around 1 mm including the line width effect.
Two self-leveling larser levels are used to give two vertical lines.
Comparing the two vertical lines, the distance between the calibration port and the bottom
block could be known.

Thus the measurement uncertainty is estimated as:


\begin{equation}
\label{equ:coaxiality}
\sqrt{
2^2\mbox{(Port)} +
2^2\mbox{(Block)} +
2^2\mbox{(Levels)}
} < 3
\end{equation}


\begin{figure}
    \centering
    \includegraphics[width=0.5\textwidth]{coaxiality.png}
    \caption
    [Scheme to show the measurement of coaxiality]
    {Scheme to show the measurement of coaxiality}
    \label{fig:coaxiality}
    \end{figure}


The coaxiality and the information of rib-to-center length of bottom lid measurement can
provide the information for the assembly of IAV and OAV. This will be discussed in Section \ref{sec:assembly}.

Because the so-called center of the bottom block is defined by where the most lines concentrate,
the relative positions and the lengths of the bottom ribs of inner acrylic vessel is also important for installation of
inner and outer acrylic vessel.
%%%%%%%%%%%%%%%%
% more about the relative position issues
%%%%%%%%%%%%%%%%

%%%%%%%%%%%%%%%
% diameter, thickness, circu
%%%%%%%%%%%%%%%

\section {Integration of Dayabay Acrylic Vessel}
\subsection {Cleaning}

IAV is the target of Dayabay antineutrino detector,
and the cleanliness is highly required to avoid any contaminant
for GdLS and possible radioactive sources like dusts in air, which
could be some $^{238}U$, $^{232}Th$, and $^{40}K$ there.

IAV can not be opened anymore after manufacture, except the calibration ports.
Thus the cleaning process is suggested before the final bonding of bottom lid.
%The proposed cleaning process is shown in Fig--------.
%The requirement to define how clean is clean is shown below.


%(how clean is clean)
% ref
The requirements for cleanliness are no metal dust, no chemical left,
no general dirt like finger print, and dust as few as possible.
The tolerance amount of the dust is estimated as the following:
assuming the concentration of the $^{238}U$, $^{232}Th$, and $^{40}K$
in the Nakano factory is in the same order as Dayabay tunnel, 10 - 100 ppm.
The background caused by the radioactive dust should be less than 30 Hz.
Assuming the efficiency of the background caused by the radioactive dust is 100\% because
the dust is confined in the IAV. The total dust amount should be less than around 100 mg.


%design of the transportation port


A class 10000 meets the FED209a standard
( particle with diameter of $0.5 {\mu}m < 10000/ft^3$ and $5 {\mu}m < 70/ft^3$ )
is adopted to be the level of the cleaning room.\cite{DocDB:3815v1}

%The QA of the cleaning room is shown below.

%(FED-STD-290E)

%\begin{figure}
%    \centering
%    \includegraphics[width=0.6\textwidth]{e209.jpg}
%    \caption{Class level of FED-STD-290E definition.}
%    \label{e209.jpg}
%    \end{figure}


According to this, the total dust amount confined in the IAV is less
than 25 mg.
Our cleaning room as shown as \ref{fig:cleaning_facility_overview}
meets the requirement of class 10000.
The cleaing and final bonding of bottom lid would be completed in the cleaning room.

The randon concentration in the Nakano factory is also measured. The concentration
is less than 10 $Bg/m^3$ after sampling the air for 1 day.
The component of $^{7}Be$, $^{238}U$, $^{232}Th$, and $^{40}K$ of dust and dirt in the factory is also measured.
The result shows the component meets the normal dirt value in Taiwan and no extra radioactivity dust in the factory.

%(QA of the cleaning room)




%(Cleaning process scheme)
\begin{figure}
    \centering
    \includegraphics[width=0.6\textwidth]{cleaning_facility_overview.png}
    \caption{Overview of the cleaing facilities.}
    \label{fig:cleaning_facility_overview}
    \end{figure}

1\% Alconox and deionized water are used to clean IAVs.
%Alconox is -------------------.
%1\% Alconox solution.
The resistance of the deionized water should be greater than 18M$\Omega$-cm.
To define the Alconox solution is removed by deionized water, conductivity of the
washdown water should be smaller than 1$\mu$S/cm after cycle rinse.


%(cycle rinse scheme)
%\begin{figure}
%    \centering
%    \includegraphics[width=0.6\textwidth]{./figure/cleaning_cycle_rinse_water.png}
%    \caption{Cycle rinse scheme}
%    \label{cleaning_cycle_rinse_water.png}
%    \end{figure}

The 1$\mu$S/cm conductivity is derived by measuring the conductivity of Alconox solution.
Alconox solution which is smaller than 1ppm shows conductivity smaller than around 1$\mu$S/cm.
Although the unit of conductivity meter
%,Hana --------,
is 0.1 $\mu$S/cm, the minimum to detect
Alconox solution concentration down to around 1$\mu$S/cm because of the possible contaminant
in pipes, pumps, water tanks, and so forth. Also the carbon dioxide may dissolve in
deionized water jet. These will make the conductivity of the washdown water raise
even through the vessel surface has been clean and the cycle rinse water does not
carry any Alconox solution, dust or other dirt. The washing facilities should be checked
the conductivity when there is only deionized water before using Alconox.


%(Test of water and Alconox solution conductivity)
\begin{table}
\centering
\caption{Test of water and Alconox solution conductivity}
\label{tab:AlconoxConductivity}
\begin{tabular}{lcp{5.0cm}}
Alconox solution &   Conductivity($\mu$S/cm) \\
\hline
\hline
1.00\% &  N/A (too large?)\\
\hline
0.10\% &  N/A (too large?)\\
\hline
100ppm & 102.3\\
\hline
10ppm  & 13.1\\
\hline
1ppm   & 0.7\\
\hline
0.1ppm & 0.1\\
\hline
deionized water $\geq 18M\Omega-cm$ &0.0~0.1\\
%\hline
%water A &0.3\\
%\hline
%water B &0.8\\
%\hline
%water C &4.7\\
%\hline
%water D &~5.6\\
%\hline
\end{tabular}
\end{table}

%
%
%\begin{figure}
%    \centering
%    \includegraphics[width=0.4\textwidth]{IAV_cleaning_as_built.png}
%    \caption
%    [As-built cleaning room]
%    {As-built cleaning room}
%    \label{fig:IAV_cleaning_as_built.png}
%    \end{figure}
%
%\begin{figure}
%    \centering
%    \includegraphics[width=0.4\textwidth]{cleaning/IAV_cleaning_overview.png}
%    \caption
%    [Clean inner surface]
%    {Clean inner surface}
%    \label{fig:IAV_cleaning_overview.png}
%    \end{figure}
%
%\begin{figure}
%    \centering
%    \includegraphics[width=0.4\textwidth]{cleaning/IAV_cleaning_procedure2.png}
%    \caption
%    [Dry IAV and seal bottom lid]
%    {Dry IAV and seal bottom lid}
%    \label{fig:IAV_cleaning_procedure2.png}
%    \end{figure}
%
%\begin{figure}
%    \centering
%    \includegraphics[width=0.4\textwidth]{cleaning/IAV_cleaning_procedure3.png}
%    \caption
%    [Final polish]
%    {Final polish}
%    \label{fig:IAV_cleaning_procedure3.png}
%    \end{figure}
%
%\begin{figure}
%    \centering
%    \includegraphics[width=0.4\textwidth]{cleaning/IAV_cleaning_procedure4.png}
%    \caption
%    [Clean exterior and dry out]
%    {Clean exterior and dry out}
%    \label{fig:IAV_cleaning_procedure4.png}
%    \end{figure}
%

%\subsection {Transportation}
%
%After the production including final cleaning of IAV, the IAV will be
%transported from Taiwan to Dayabay.
%The IAV prototype was checked by photoelasticity with naked eye before any
%critical lifting and transportation.
%
%
%%(polarizer check of the IAV prototype at SAB)
%%(lifting test of the prototype at SAB)
%
%
%Accroding to the experience of the IAV
%prototype, the pack method is feasible shown in Fig -----------.
%
%
%(Pack scheme)
%
%
%The route for IAVNo1 and IAVNo2 is shown in Fig--------.
%It will take around 1 or 2 week from Taiwan to Dayabay site.
%
%
%(route)
%
%
%More moniter devices were suggested to moniter the transportation
%status. Table-------- summary the devices.
%
%
%(moniter device for IAV transportation)
%
%
\subsection {Assembly}
\label {sec:assembly}

%IAV should be installed into OAV. A proposed AD installation
%scheme is as Fig-------, including installation of IAV.


%(AD installation scheme, Karsten slides)


From the point of installation view, the coaxiality of the top center calibration port and
the bottom pin is mostly cared. Installation could be bottom-to-top or top-to-bottom, deponding
on the coaxiality. A top-to-bottom way, Figure \ref{fig:assembly_top_to_bottom}, focus on
the alignment of the calibration ports between inner acrylic vessel and outer acrylic vessel.
This may lead to the bottom ribs of inner acrylic vessel are not aligned with the ribs of outer acrylic vessel.
If the ribs of inner acrylic vessel are not aligned with the ribs of outer acrylic vessel,
these two vessels may not be assembled.
A bottom-to-top way, Figure \ref{fig:assembly_bottom_to_top}, focus on the assembly issues first, namely
the alignment of the bottom ribs of inner and outer acrylic vessels. This may lead to the calibration
ports could be not aligned properly.
A suitable installation way is still under studying.

The current status of the prototype installation shows the distance of the center of the stainless steel tank and
the IAV is about 7 mm. The measurement was done by a total station. It is close to our requirement, 5 mm for the installation
of the calibration pipes.

\begin{figure}
    \centering
    \includegraphics[width=0.4\textwidth]{assembly_top_to_bottom.png}
    \caption
    [Top-to-bottom assembly]
    {Top-to-bottom assembly}
    \label{fig:assembly_top_to_bottom}
    \end{figure}


\begin{figure}
    \centering
    \includegraphics[width=0.4\textwidth]{assembly_bottom_to_top.png}
    \caption
    [Bottom-to-top assembly]
    {Bottom-to-top assembly}
    \label{fig:assembly_bottom_to_top}
    \end{figure}


%(bottom-to-top installation scheme and the error)
%(top-to-bottom installation scheme and the error)



%A proposed coaxiality QA measurement is shown in Fig--------. The uncertainty
%of the QA measurement is less than 3mm.
%(table of measurement uncertainty)
%(QA of coaxiality measurement scheme)


%\section {Manufactor of Aberdeen Reflector}
%\label{sec:reflector}
