\begin{acknowledgements}

%Besides my parents, I have cosidered the dedication should be for myself at the beginning.
At the time of writing, some of experiment tasks still are kept running.
This makes that I could not focus on the writing of this thesis fully.
This thesis is completed in a hurry. I feel frustrated a little bit because I suppose
I could make the thesis better but I didn't.

I am confident of the adequate training and studying I had in the past three year
to be a master, who is going to be a particle experiment physicist.
I see the profile of future and the possibility I can be a particle experiment physicist.
However, I didn't fully show the proud traning I had. That's why I feel frustrated.


In the past three years, there are so many people that I would like to thank.
The following is the name list:
My families, for the unconditional support, including
economic and emotion support.
My girlfriend, for the company. This is really important
to fight against the frustrated days.

My advisor, for the ticket to join to the particle physics
experiment community and the instruction.

Dr. Simon C. Blyth (National United University, NUU), for the instruction (mainly software stuffs) and the
pleasure cooperation and traveling days.

Prof. Chung-hsiang Wang (NUU), for the help and resource,
and Chang Yun (NUU), for the help and mechanical knowledge, during the time of
the manufacture of inner acrylic vessel.

Prof. Min-zu Wang (National Taiwan University, NTU) for the many kinds of instruction and suggestion
and the spectrometer Perkin Elmer Lambda 650.

Our NTU high energy physics group, including the network managers and secretary team
for the support of hardware, network and paper works. I also thank all
the students for the discussion of experiment issues and the happy days.

Dr. Wen-chen Chang (Academia Sinica) and Dr. Ping-kun Teng (Academia Sinica)
for the instruction of DAQ.

Dayabay NCTU (Natinal Chiao Tung University) team, especially Yung-shun Yeh, for the support of network resource.

The Aberdeen Tunnel experiment group, including HKU (University of Hong Kong) and CUHK (Chinese University of Hong Kong),
for the pleasure cooperation and the happy days in Hong Kong. I learn many things from you.
I would like to thank Prof. Kam-bui Luk (Lawrence Berkeley National Laboratory, LBNL) for the strict and careful reminding about
how to proceed a particle experiment. I also would like to thank Jimmy Ngai (HKU) for the
pleasure cooperation and helpful discussion of the neutron reconstruction for Aberden Experiment.

Dayabay UW (University of Wisconsn at Madison) team, for the pleasure cooperation days
and many useful suggestions and discussion about the acrylic vessel issues, including assembly, RT method and QA.

Dr. Cao Jun, Dr. Xinchun Tian, Dr. Liang Zhan, and Liangjian Wen of IHEP (Institute of High Energy Physics)
for the happy days in Beijing and the many useful suggestions and instruction.

The Dayabay team, for their hard work for the experiment.

Dr. Jiunn-hsing Chao (National Tsing Hua University, NTHU), for the measurement of radioactive.

Prof. Fu-tsai Huang (NUU), for the Shimadzu UV-3101 PC spetrometer.

Jieh-wen Tsung, for the support of friendship and the encouraging from the particle experiment physics student view of point.
Chen-hua Liu and Wei-chen Lo, for the support of friendship. Without these friendship, I can not imagine how
do I fight against the tough days.



%The names above always remind me that



%There are many people I would like to thank. I might miss some people in the name list.
%Although 


%
%Dayabay:
%    taiwan:
%        advisor
%        Simon
%        prof. wang
%        prof. chang
%        prof. mz wang
%
%    nctu
%
%    hk
%
%
%    uw
%
%
%    ihep
%
%
%daq
%prof. wen z chang, acadamia sinica
%
%Spectrometer:
%Prof. Huang
%
%
%radioactivy:
%chao, nthu
%
%Friends:
%JanWen
%logit
%orange
%
%
%family:
%families
%yifang
%
\end{acknowledgements}

