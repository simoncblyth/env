\chapter {Detector Simulation}

\section {Offline Software Overview}

\subsection {G4dyb}

G4dyb is a application program composed of GEANT4.
G4dyb provides many preliminary simulation tasks and could be
the validation of NuWa, the later offline software.
The version control of G4dyb is CVS in the early days and
SVN later. CMT is used to manage the configuration of G4dyb.
The latest version of G4dyb is release ------------, and
G4dyb has been stopped to develop around in the middle of last year(2008).


\subsection {NuWa}

NuWa is the official offline software of Dayabay experiment.
It contains not only the physics simulation engine, but also
other offline programs, including electronic simulation,
trigger simulation, analysis programs, and so forth.

NuWa is a framework, which is different from G4dyb.
A framework-----------------------, ------------------.
It could be realized NuWa is a assembly of many kinds of application program,
e.g. Root, the analysis framework developed by CERN, GEANT4, and so forth.
The framework is based on Gaudi. Gaudi is ----------------------.

NuWa communicates with GEANT4 by GiGa, Gaudi interface to GEANT application-------.
A typical script to control GiGa is Python. The simulation structure scheme is shown below.


(NuWa simulation structure scheme)


The version control of NuWa is SVN and CMT is used to manage
the configuration of NuWa. At the time of writing, the latest
version of NuWa is 1.3.0--------------------------.

The algorithm of NuWa simulation scheme is shown below.


(simulation algorithm scheme of NuWa)





NuWa overview

why NuWa

DataModel AES TES

Gaudi

Giga


\subsubsection {Comparison of NuWa and G4dyb}





\section {Mock Data Challenge}

http://dayabay.ihep.ac.cn/cgi-bin/DocDB/ShowDocument?docid=2602


\section {Basic Distributions of Dayabay and Aberdeen Detectors}

There are some basic distribution has been done to study the general
property of both Dayabay and Aberdeen detectors by G4dyb in the early
days and NuWa.


(The uniformity of Dayabay AD--R)




(The uniformity of Dayabay AD--Z)




(Energy resolution of Dayabay AD --- at 1 mev gamma)




(The uniformity of Aberdeen Neutron detector--R)




(The uniformity of Aberdeen Neutron detector--Z)




(Energy resolution of Aberdeen Detector-----at 1mev gamma)


Because both Dayabay and Aberdeen detectors uses inverse
beta decay as signals, the prompt and delayed signals are
1-------mev and 8------mev gamma separately. Thus
the energy cut of gamma at 6mev is adopted.


\subsection {Signal box}

Dan, collaboration meeting June, 2009


\section {The Effect of Reflector Position}
\label {Reflector}

Reflectors are used to enhance the p.e. number collected by the PMTs of Dayabay
detectors. Accroding to the location to position reflectors in AD, the cost and
manufacure process would be different. Simulation of the p.e. number difference
between different reflector location in AD is shown below by G4dyb release--------.
The condiction to simulate..................


(reflector position 1, uniformity R)

\begin{figure}
    \centering
    \includegraphics[width=0.8\textwidth]{./figure/reflector_position_pe_e_vs_r_basic_distri.png}
    \caption{Total p.e. distribution generated along r-axis by positrons distributed uniformally in the acrylic tank.}
    \label{fig:reflector_position_pe_e_vs_r_basic_distri.png}
    \end{figure}


\begin{figure}
    \centering
    \includegraphics[width=0.8\textwidth]{./figure/reflector_position_pe_e_vs_z_basic_distri.png}
    \caption{Total p.e. distribution generated along z-axis by positrons distributed uniformally in the acrylic tank.}
    \label{fig:reflector_position_pe_e_vs_r_basic_distri.png}
    \end{figure}



(reflector position 1, uniformity Z)



\begin{figure}
    \centering
    \includegraphics[width=0.8\textwidth]{./figure/reflector_position_pe_basic_distri.png}
    \caption{Total p.e. distribution generated by positrons distributed uniformally in the acrylic tank.}
    \label{fig:reflector_position_pe_basic_distri.png}
    \end{figure}



(reflector position 1, total p.e.)




(reflector position 2, uniformity R)




(reflector position 2, uniformity Z)




(reflector position 2, total p.e.)




(reflector position 3, uniformity R)




(reflector position 3, uniformity Z)




(reflector position 3, total p.e.)


The simulation shows the difference of total p.e. is---------------
The cost--------------------
So--------------------------
Other R \& D details could be found in section--------------------.


\section {The Effect of Optical Property Measurement}
\subsection {The Optical Model}

The optical model of GEANT4 requires the index of refraction and the attenuation length
of matirails. The measurement method of these two parameters of acrylic pannels for Dayabay IAV
is mentioned in section \ref{sec:RTMethod}.
The measurement may fail at few wavelength for some reasons, also discussed
in section --------------.
To model the index of refraction by a smooth curve to fit in GEANT4 requirement, Cauchy equation


(Chaucy equation)


is used. To model attenuation length, a Fermi-Dirac-like distribution



