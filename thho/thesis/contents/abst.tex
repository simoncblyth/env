\begin{abstract}

Dayabay neutrino experiment is a high-precision measurement experiment
of the reactor anti-electron neutrino flux at Dayabay to search for $\theta_{13}$.
The goal of the Dayabay reactor antineutrino experiment is to determine the unkown
neutrino mixing angle $\theta_{13}$ with a sensitivity of 0.01 or better in
$sin^{2}{2\theta_{13}}$, an order of magnitude better than the current limit.

The detection mechanism is to detect neutron generated from inverse $\beta$ decay.
Thus the neutron background in Dayabay neutrino experiment is significant.
Aberdeen muon-induced netron detector monitors the neutron backgound from the
cosmic rays most of which are mouns, nearby Dayabay.

This thesis would focus on the acrylic vessel of antineutrino detector for Dayabay experiment, including
the fabrication and optical property.
For the neutron detector for Aberdeen experiment, the test of generic algorithm for reconstruction.
It is expected this kind of algorithm will be one of the reconstruction algorithm for Dayabay antineutrino
detector. This thesis would also briefly introduce the recent milestones of both the Dayabay experiment and
Abedeen experiment.

\end{abstract}
