\begin{abstract}

Dayabay neutrino experiment in Dayabay, China is a high-precision measurement experiment
of the reactor anti-electron neutrino flux at Dayabay to search for neutrino mixing angle $\theta_{13}$.
The goal of the Dayabay reactor antineutrino experiment is to determine the unknown
mixing angle $\theta_{13}$ with a sensitivity of 0.01 or better in
$sin^{2}{2\theta_{13}}$, an order of magnitude better than the current experimental limit.

The detection mechanism is to detect both positron (prompt) and neutron (delayed) generated
from inverse $\beta$ decay of anti-neutrino.
Thus the understanding of neutron background in Dayabay neutrino experiment is important.
Muon-induced neutron background can be detected in the Aberdeen neutron detecotr in Aberdeen, Hong Kong.
Aberdeen neutron detector monitors the induced neutron backgound from the
cosmic rays most of which are mouns, nearby or passing through the detector.
Aberdeen is nearby Dayabay site and with similar geological composition as Dayabay site.
The neutron flux observed by Aberdeen Experiment would be a important reference of Dayabay Experiment.

This thesis would focus on the R\&D, construction, and optical property of acrylic vessel of antineutrino detector for Dayabay Experiment.
For the neutron detector of Aberdeen Experiment, the test of generic algorithm for reconstruction is mainly discussed.
It is expected this kind of algorithm could be one of the reconstruction algorithm for Dayabay antineutrino
detector. This thesis would also briefly introduce the recent milestones of both the Dayabay experiment and
Abedeen experiment.

\end{abstract}
