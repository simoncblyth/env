\begin{abstract}

Dayabay neutrino experiment is a high-precision measurement experiement
of the reactor anti-electron neutrino flux at Dayabay to search for $\theta_{13}$.
The goal of the Dayabay reactor antineutrino experiment is to determine the unkown
neutrino mixing angle $\theta_{13}$ with a sensitivity of 0.01 or better in
$sin^{2}{2\theta_{13}}$, an order of magnitude better than the current limit.

The detection mechanism is to detect neutron generated from inverse $\beta$ decay.
Thus the neutron background in Dayabay neutrino experiment is significant.
Aberdeen muon-induced netron detector monitors the neutron backgound from the
cosmic rays which most of them are mouns, nearby Dayabay.

This thesis would focus on the antineutrino detector for Dayabay experiment and the neutron
detector for Aberdeen experiment.

\end{abstract}
