\chapter{Aberdeen Neutron Detector}
\section{A experiment at Aberdeen}

The Dayabay antineutrino signal is composed of a prompt positron signal and a delayed
neutron signal. Although the time and energy tagged signals could be powerful tool to suppress
the background, there are still accidentally conincidence signals. Any signal happens during
the typical time window after the prompt position signals happen for 30 micro-second to 120 micro-second,
which is typical thermal neutron drifting time, 
could be regarded as a antineutrino candidate. These accidnetally conincidence signals could be
muon-induced neutrons, radioactivity of PMT glasses, natural radioactivity of rock
( usually $^238_U$, $^232Th$, and $^40K$) etc. $(\alpha,n)$ processes may also produce
neutrons from natural radio-nuclides. Cosmogenic $^8He$ and $^9Li$ can undergo beta-neutron decay,
and this may also mimick neutrons from inverse-beta decay.

%The radioactivity of PMT could be known by studying the ingredient of the PMTs and decreased
%by the mineral oil buffer. The other natural radioactivity, like rock or dusts in the 
%air could be studied or monitored directly. Muon-induced neutron is a complicated
%process and needed measured the flux directly.

A muon-induced neutron detector is proposed to measure muon-indeced neutron flux Aberdenn, Hong Kong,
around 90 km away from Dayabay. See \ref{fig:googleMap.png}.
The detector is positioned at a vertical overburden of ~600 m.w.e. in Aberdeen tunnel, and the rock of the tunnel
providing veto of many cosmic rays. Muon, which has low attenuation length in rock, could
survives after passing through the rock of the tunnel.

Aberdeen is nearby Dayabay and with similar geological composition as Dayabay. Figure \ref{fig:abtRock.png}
shows the comparison of the rock composition.
The Aberdeen Tunnel Experiment is aimed at studying those backgrounds encountered by the Dayabay
Experiment. The main focus of the Aberdeen Experiment is to study cosimic-ray muons in terms
of measuring their angular distribution, flux and contribution to neutron background in an
underground environment like the Dayabay ones. The similar geological composition
is expected to lead into the similar muon-induced neutron background at Dayabay and Aberdeen.


\begin{figure}
    \label{fig:abtRock.png}
    \centering
    \includegraphics[width=0.8\textwidth]{abtRock.png}
    \caption
    [Rock composition of Dayabay and Aberdeen]
    {Rock composition of Dayabay and Aberdeen. This shows both the compositions are similar.}
    \end{figure}


The setup of the Aberdeen Tunnel Experiment is similar to the Dayabay one and consists of
two parts, the muon tracker and the neutron detecotr. The neutron detector is placed in the
middle of the muon tracker.
More details of the neutron detector will be disscussed in section \ref{sec:ND}.

The muon tracker, which consists of 63 scintillator hodoscopes and 72 proportional
tubes arranged in six layers.
Three layers are above the neutron detector and three layers are below the neutron detector seperately.
The top layer consists of 30 scintillator hodoscope of 1 m long and 10 cm wide with
one end readout by 2-inch PMTs. The second layer consists of 24 proportional tubes of diameters of
7.6 cm and lengths of 2 m placed orthogonal to the top layer. the third layer consists of 18 1.5 m-long
and 10 cm-wide scintillator hodoscope with one end readout by a 2" PMT. The forth layer is a row
of 24 proportional tubes of diameters of 7.6cm and lengths of 2 m placed orthogonal to the third layer.
The fifth layer is make up of 15 2m-long and 9.3 cm-wide scintillator hodoscope with both end
readout by Hamamatsu 7826 Small PMTs (SPMT). The last layer is an array of 24 proportional tubes
of diameters of 7.6 cm and lengths of 2 m placed  orthogonal to the fif layer.


%\begin{figure}
%    \centering
%    \subfloat[Aberdeen Experiment Setup]{
%        \label{fig:abtExp.png}
%        \includegraphics[width=0.8\textwidth]{./figure/abt/abtExp.png}
%    }
%    \hspace{1in}
%    \subfloat[Aberdeen Tunnel Experiment DAQ procedure.]{
%        \label{fig:abtDataFlow.png}
%        \includegraphics[width=0.8\textwidth]{./figure/abt/abtDataFlow.png}
%    }
%    \caption{Overview of Aberdeen Tunnel Experiment}
%    \label{fig:abtOverview}
%    \end{figure}


\begin{figure}
    \label{fig:abtExp.png}
    \centering
    \includegraphics[width=0.8\textwidth]{abtExp.png}
    \caption
    [Aberdeen Experiment Setup]
    {Aberdeen Experiment Setup}
    \end{figure}



\begin{figure}
    \centering
    \includegraphics[width=0.8\textwidth]{abtNeutronDetector.png}
    \caption
    [Exterior looking of Aberdeen neutron detector]
    {Exterior looking of Aberdeen neutron detector}
    \label{fig:abtNeutronDetector}
    \end{figure}

\begin{figure}
    \centering
    \includegraphics[width=0.8\textwidth]{abtND.png}
    \caption
    [Cross-sectional model of Aberdeen neutron detector.]
    {
Cross-sectional model of Aberdeen neutron detector.
Aberdeen neutron detector is a 2-zone detector. The inner one is the target, which is
0.1\% Gd doped liquid scintillator hold by a acrylic vessel.
The outer one is the oil buffer between 16 PMTs and the acrylic vessel.}
    \label{fig:abtND}
    \end{figure}


\begin{figure}
    \label{fig:abtDataFlow}
    \centering
    \includegraphics[width=0.8\textwidth]{abtDataFlow.png}
    \caption
    [Overview of Aberdeen Tunnel Experiment]
    {Overview of Aberdeen Tunnel Experiment}
    \end{figure}




The conincidence of the top and bottom muon trackers can tag the muon candidates
associated to the neutron signals in the neutron detector. Neutron signals happened
in the time window after muon candidates are tagged for around 30 micro-second to
120 micro-second, would be regarded as muon-induced neutron signal candidates.
The DAQ process scheme is shown in Figure \ref{fig:abtDataFlow}.




\section{The Neutron Detector at Aberdeen}
\label{sec:ND}
\subsection{Design and Signals}

The neutron detector of Aberdeen Tunnel Experiment is similar to the
anti-neutrino detectors of Dayabay Experiment. A Dayabay anti-neutrino
detector is a neutron detector in essence. A Dayabay antineutrino detector is a
3-zone detector and a Aberdeen neutron detector is a 2-zone detector. See
Figure \ref{fig:abtND}.

Because of limitation of the tunnel room, a 2-zone neutron detector is used.
The inner zone is composed of a cylindrical acrylic tank with a diamter of 110 cm and a height of 80 cm,
filled with liquid scintillator doped 0.1\% Gd to be the neutron target and
the outer zone is composed of mineral oil. There are 16 8" Hamamatzu R1408 PMTs in the mineral oil and viewing the target.
All of these are positioned inside a square steel tank with dimension 160 cm $*$ 160 cm $*$ 113 cm. The 16 PMTs
are arranged in four columns, mounted in the four coners of the steel tank. On top and beneath the acrylic tank is
a sheet of Miro-Silver with a diameter of 152 cm and a thickness of 0.5 mm, for reflecting photons and
enhancing the photon number received by PMTs.
%This kind of Miro-Silver is so called the reflector and will be discussed later in section \ref{sec:reflector}.
The Miro-Silver sheet is attached on a acrylic panel of thickness of 3 mm by 3M DP810 glue.

Besides the top and bottom specular reflectors, there are diffuse reflector on the inner surface of the stainless steel tank.
The diffuse reflector is a tyvek sheet.
% The tyvek sheet measurement data.

%calibration of muon tracker
%http://theta13.phy.cuhk.edu.hk/elog/Aberdeen-Hardware/542
%
%MO calibration
%http://theta13.phy.cuhk.edu.hk/elog/Aberdeen-Design/259




