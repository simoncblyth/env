\chapter{Dayabay Anti-neutrino Detectors and Aberdeen Neutron Detector}

\section{The Dayabay Anti-neutrino Detectors}
\label{sec:AD}

The inverse beta-decay is adopted as the signals of Dayabay antineutrino detectors.
The measurement of sin2theta13 to less than 0.01 corresponds to a small
difference in a lot of antineutrino events observed between the far site and
the near site.

The delayed signals are generated by the neutrons of inverse beta-decay process.
Neutrons are neutral so the cross section with matters is small.
A large mass antineutrino detector should be in the order of tons.
Typical requirement of an antineutrino detector is summarized in Table \ref{tab:ADRequirement}


\begin{table}
\centering
\caption{Physical requirements of the antineutrio detector\cite{TDR}}
\label{tab:ADRequirement}
\begin{tabular}{lcp{5.0cm}}
\hline
Item & Requirement & Justification\\
\hline
\hline
Target mass at far site &
$\geq$ 80T &
Achieve sensitivity goal in three years over allowed ${\Delta}m^{2}_31 range$ \\
\hline
Precision on target mass &
$\leq$ 0.3\% &
Meet detector systematic uncertainty baseline per module \\
\hline
Energy resolution &
$\leq$ $15\%/\sqrt{E}$ &
Assure accurate calibration to achieve required uncertainty in energy-threshold cuts dominated by energy threshold cut \\
\hline
Detector efficiency error &
$\leq 0.2\%$ &
Should be small compared to target mass uncertainty \\
\hline
Positron energy threshold &
$\leq 1MeV$ &
Fully efficient for positrons of all energies \\
Radioactivity signals rate &
$\leq 50 Hz$ &
Limit accidental background to less than other backgrounds and keep data rate manageable \\
\hline
\end{tabular}
\end{table}


It's simpler to simulate and analysis the antineutrino detectors if they are
symmetric, so a sphere or a cylinder shape could be considered. The cylinder
shape is easier to produce from the point of engineering view, and also lower
price. Thus the antineutrino detectors are designed based on the cylinder shape.

The delayed signal are the 2.2MeV and 8MeV neutrons . There are external backgrounds like
muon-induced neutrons, radioactivity of PMT glasses, natural radioactivity of rock,
buffer water etc. Mineral oil is used for the antineutrino detectors to be a buffer.

Accroding to the neutron capture processes, by protons and by Gd, and the buffer to
suppress the background, 2-zone and 3-zone detectors are considered. The innermost
zone is the antineutrino target, composed of the liquid scintillator doped 1\% Gd.
The second zone is the gamma catcher, composed of the liquid scintillator, for a
3-zone detector or the mineral oil for a 2-zone detector. The outermost zone
is the mineral oil for a 3-zone.

(overview of the 2-zone and 3-zone antineutrino detecotrs)


The comparison for 2-zone and 3-zone detectors shows 
uncertainty of the neutron energy threshold, caused by uncertainty in the energy scale.
3-zone detectors could have better energy resolution where the energy scale
uncertainty is taken to be 1\% and 1.2\% at 6MeV and 4 MeV, respectively.

(table of the comparison)

Thus 3-zone detectors are used. The innermost GdLS is 20 tons and the outer layer is another
20 tons of the unloaded  liquid scintillator. The outermost is 40 tons of mineral oil. 192 eight-inch
Hamamatsu photomutiplier tubes (PMTs) are mounted on the inner surface of the stainless steel tank
in eight horizontal rings, viewing the target volume and the gamma catcher. The crooss-sectional model of an AD
is shown in Fig. \ref{fig:AD_module.png}.

\begin{figure}[h]
    \centering
    \includegraphics[width=0.6\textwidth]{./figure/AD_module.png}
    \caption{Cross-sectional model of the antineutrino detectors.
The model show complete AD including calibration boxes and overflow tanks on the top of the AD.}
    \label{fig:AD_module.png}
    \end{figure}





TDR v2




\section{The Neutron Detector at Aberdeen}
\label{sec:ND}
\subsection{Design and Signals}


\begin{figure}[h]
    \centering
    \includegraphics[width=0.8\textwidth]{./figure/abtND.png}
    \caption
    [Cross-sectional model of Aberdeen neutron detector.]
    {Cross-sectional model of Aberdeen neutron detector.}
    \label{fig:abtND.png}
    \end{figure}

\begin{figure}[h]
    \centering
    \includegraphics[width=0.8\textwidth]{./figure/abtNeutronDetector.png}
    \caption
    [Exterior looking of Aberdeen neutron detector]
    {Exterior looking of Aberdeen neutron detector}
    \label{fig:abtNeutronDetector.png}
    \end{figure}








The neutron detector of Aberdeen Tunnel Experiment is similar to the
anti-neutrino detectors of Dayabay Experiment. A Dayabay anti-neutrino
detector is a neutron detector in essence. The detection mechanism is the same.
Both of them use inverse-beta decay.

Because of limitation of the tunnel room, a 2-zone neutron detector is used. The inner zone
is composed of a cylindrical acrylic tank with a diamter of 110 cm and a height of 80 cm, filled with liquid scintillator doped 1\% Gd to be the neutron target and
the outer zone is conposed of mineral oil. There are 16 8" Hamamatzu R1408 PMTs in the mineral oil and viewing the target.
All of these are positioned inside a square steel tank with dimension 160 cm $*$ 160 cm $*$ 113 cm. The 16 PMTs
are arranged in four columns, mounted in the four coners of the steel tank. On top and beneath the acrylic tank is
a sheet of Miro-Silver with a diameter of 152 cm and a thickness of 1 mm, for reflecting photons and
enhancing the photon number received by PMTs. This kind of Miro-Silver is so called the reflector and will
be discussed later in section \ref{sec:reflector}


calibration of muon tracker
http://theta13.phy.cuhk.edu.hk/elog/Aberdeen-Hardware/542

MO calibration
http://theta13.phy.cuhk.edu.hk/elog/Aberdeen-Design/259

