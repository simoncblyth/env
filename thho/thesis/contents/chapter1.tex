
\chapter{Introduction}

\section{Neutrinos}

In the early 1900s, William Bragg demonstrated that mono-energetic $\alpha$-rays
were emitted in $\alpha$ decays\cite{Franklin:ch1}, which are two-body decays. Therefore $\beta$-rays
were supposed to be mono-energetic also because of 2-body decay of protons and
electrons. Nevertheless, J. Chadwich showed that the $\beta$-ray
spectrum was continuous\cite{Franklin:ch1}. Fig. \ref{fig:inverse_beta_decay_spectrum.png} shows the typical beta-rays energy distribution curve.


Pauli postulated a tiny neutral particle which we call neutrino today. The neutrino
of $\beta$ decay carrys away some energy so it's a 3-body decay.\cite{Franklin:ch1}






The neutrinos are not observed directly. Only the products from their interactions
with matter are observed. A particular neutrino flavor is associated to the obseved lepton type.
Our reference to neutrino mass is the effective mass
associated with the flavor state.
There are three flavors of neutrinos: electron neutrino $\nu_e$, muon neutrino $\nu_{\mu}$, and
tau neutrino $\nu_{\tau}$.\cite{Franklin:ch1}\cite{pdg}


%----------------------
%(overview of the flavor types, observed flavors and exps)
%----------------------



\begin{figure}
    \centering
    \includegraphics[width=0.6\textwidth]{./figure/inverse_beta_decay_spectrum.png}
    \caption{Energy distribution curve of the beta-rays \cite{Scott:1935}}
    \label{fig:inverse_beta_decay_spectrum.png}
    \end{figure}




\subsection{Neutrino Oscillation, $\theta_{13}$ and Standard Model}


%-----------------------------------------------------------------
%The standard model of quarks and leptons. (Alan poon, OCPA slides)
%standard model: m= 0
%-----------------------------------------------------------

Neutrinos in the standard model of particle physics are massless.
However, in the past decade, experiments in solar have shown indirectly through neutrino oscillation 
that neutrinos are massive. Since neutrinos can have mass with very small mass splitting, different flavors of neutrinos
can mix quantumechanically and show oscillation behavior when propagating through space.\cite{Perkins:ch1}\cite{Vahle:ch1}\cite{Fkuda:1998} \cite{SNO:2001} 

%-----------------------------------------------------------------
%MNSP matrix
%-----------------------------------------------------------------

%----------
%exp ref
%----------


%\section{Underground Experiments}
%
%(Alan poon, OCPA slides)
%
%No. of observed events:
%
%Because of small cross section, neutrinos do not interact readily, thus
%a powerful neutrino sources are needed for study neutrinos. A source could be
%accelerators, nuclear reactors, the sun, or supernovae.
%
%(comparison of diff sources power)
%
%And also, large mass targets are needed. A typical target is around in the order
%of few tons, at least.
%





\section{The Dayabay Neutrino Oscillation Experiment}
\subsection{Overview}

A deep underground neutrino observatory at Dayabay, China, has
been proposed.\cite{DYBProposal:2007} \cite{TDR}
Dayabay is nearby Hong Kong. See Figure \ref{fig:googleMap.png}


\begin{figure}
    \label{fig:googleMap.png}
    \centering
    \includegraphics[width=0.8\textwidth]{googleMap.png}
    \caption
    [Dayabay and Aberdeen Tunnel location]
    {Dayabay and Aberdeen Tunnel location shown by Google Map.
$The\,dot\,at\,the\,upper-right\,of\,the\,map:$ Dayabay.
$The\,dot\,at\,the\,lower-left\,of\,the\,map:$ Aberdeen Tunnel.
The distance between Dayabay and Aberdeen Tunnel is around 90 km.}
    \end{figure}


The idea to measure $\theta_{13}$ is to use two or more antineutrino detectors
to measure the survival or disappearance probability of reactor antineutrino, which is electron-type antineutrino.
The observed survival probability associated to different distance could be derived 
%------------------------
%from the MNS matrix :
%-------------------------


\begin{equation}
\label{equ:survivalProb}
P_{sur} = P_{ee} = 1 - \sin^{2}{2\theta_{13}}\sin^{2}{\frac{{\Delta}m^{2}_{31}L}{4E_{\nu}}} - \cos^{4}{\theta_{13}}\sin^{2}{2\theta_{12}}\sin^{2}{\frac{{\Delta}m^{2}_{21}L}{4E_{\nu}}}
\end{equation}


Thus compare the ratio of interaction rates in multiple detectors as shown in Equation \ref{equ:compareDetectorsRatio}, and
the survival probability term shows the $\sin^{2}{2\theta_{13}}$.


\begin{equation}
\label{equ:compareDetectorsRatio}
\frac{N_{f}}{N_{n}} = (\frac{N_{p,f}}{N_{p,n}})(\frac{L_n}{L_f})^{2}(\frac{\epsilon_f}{\epsilon_n})[\frac{P_{sur}(E,L_f)}{P_{sur}(E,L_n)}]
\end{equation}

$N_{f}$ and $N_{n}$ are measured ratio of rates of far site and near site respectively.
$N_{p,f}$ and $N_{p,n}$ are detector mass ratio.
${L_n}$ and ${L_f}$ are distance of near site and far site respectively.
${\epsilon_f}$ and ${\epsilon_n}$ are detector efficiency ratio of far site and near site respectively.
$E$ and $L_f$/$L_n$ of $P_{sur}$ associated to the $E_{\nu}$ and $L$ in Equation \ref{equ:survivalProb} respectively.
Dayabay anti-neutrino detectors (AD), which will discussed more in Section \ref{sec:AD},
are be placed at Dayabay underground laboratory. They are placed at varying distance from the Dayabay reactor and
the Ling Ao reatcors as shown in \ref{fig:location_dayabay.png}. There are two sets of near detectors
at near sites, close to the Dayabay reactor and the Ling Ao reactors respectively, and two modules of each set.
There is one set of far detectors, with four modules. The Dayabay near site with an overburden of
255 with meter of water equivalent (m.w.e.) is located at a distance of 363m from the Dayabay cores. The Ling Ao near sitewith an overburden of
294 m.w.e. is 481m and 526m from the Ling Ao I and Ling Ao II cores respectively. The far site is at the distance
of 1985m and 1615m from the Dayabay cores and the mid-point of the Ling Ao-Ling Ao II cores, and has an overburden
of 910 m.w.e.

At the time of writing, the Ling Ao II cores is still under construction and are going to be built and run in around
2011. This effect should be study and has been studied by Dayabay collaborators.


\begin{figure}
    \centering
    \includegraphics[width=0.8\textwidth]{./figure/location_dayabay.png}
    \caption{AD module positions at Dayabay.}
    \label{fig:location_dayabay.png}
    \end{figure}


The baseline between the near site AD and far site AD associated to the neutrino oscillation
location is shown in Figure \ref{fig:oscillation}


\begin{figure}
    \label{fig:oscillation}
    \centering
    \includegraphics[width=0.8\textwidth]{oscillation.png}
    \caption
    [The Dayabay detector location associated to the survival probability due to neutrino oscillation]
    {The Dayabay detector location associated to the survival probability due to neutrino oscillation.
$Blue:$ Large but slow oscillation due to the $\theta_{12}$. $Green:$ Small oscillation due to the $\theta_{13}$.}
    \end{figure}




The main backgrounds to the Dayabay Experiement are induced by cosmic-ray muons.
The site is located adjacent to mountainous terrain, ideal for sitting underground detectors that
are well shield from external backgrounds like cosmogeic backgrounds including the cosmic-ray mouons.
However, background due to the muon spallation products at the depths of the experiment halls and ambient
$\gamma$ background due to the radioactivity of the rock surrounding the experiment halls is reduced by shielding
the ADs with 2.5 meter of water. This water shield also attenuates the flux of neutrons produced outside the water
pool.

Events with a muons that passes through the water in the water pool less than 200 $\mu$s, which is the typical time window
a prompt signal due to positron annihilation and a delayed signal due to capture of neutron of an inverse-beta decay process, have a small but finite
probability to create a fake signal event.
There is a layer of  Resistive Plate Chambers (RPC) above a water pool
and 8" PMTs surrounding the water pool for the Cherenkov detection of muons to tag such muon signals.
The setup is shown in Fig. \ref{fig:water_pool.png}.


\begin{figure}
    \centering
    \includegraphics[width=0.8\textwidth]{./figure/water_pool.png}
    \caption{Elevation view of an experiment hall,
showing the water pool to shield backgrounds from muons, muon-induced neutrons and radioactivity of rock.}
    \label{fig:water_pool.png}
    \end{figure}


%\subsection{Progject Organization}

%http://dayabay.ihep.ac.cn/cgi-bin/DocDB/ShowDocument?docid=2298

%WBS, TDR



\section{The Dayabay Anti-neutrino Detectors}
\label{sec:AD}

Dayabay antineutrino detector is a 3-zone detector as shown in Figure \ref{fig:3zone.png}.
The inner most zone is the antineutrino target, which
contains 0.1\% gadolinium(Gd)-doped liquid scintillator. The second zone
contains the gamma catcher, which is liquid scintillator only.
The outer most zone is the mineral oil buffer surrounded by 192 8" PMTs.


\begin{figure}
    \centering
    \includegraphics[width=0.8\textwidth]{./figure/introduction/3zone.png}
    \caption
    [3-zone structure of Dayabay antineutrino detector]
    {3-zone structure of Dayabay antineutrino detector. $Left:$ Transparent cross-sectionof the antineutrino detector.
$Right:$ Scheatic top view of the three different zones o the antineutrino detector.}
    \label{fig:3zone.png}
    \end{figure}


The detection mechanism uses the inverse $\beta$ decay.


\begin{equation}
\overline{\nu_e} + p \rightarrow e^+ + n
\label{eq:InverseBetaDecay}
\end{equation}


The positrons will annihilate soon after they are generated by the inverse beta-decay in Equation \ref{eq:InverseBetaDecay},
and generate gammas which could be observed by PMTs.
This is the so-called prompt signal as shown in Figure \ref{fig:promptSignal}.
The neutrons will go through
thermal scattering process and finally captured by nucleus, namely protons or Gd.
If a neutron captured by a proton, this process will generate a 2.2 $MeV$ gamma-ray each event, and it is a delayed signal comparing
to the positon annihilation.


\begin{equation}
n + p \rightarrow D + \gamma (2.2 MeV)
\label{NeutronPCapture}
\end{equation}


The cross section of the neutron captured by a proton is only 0.3b.
To enhance the neutron capture probability, a 0.1\%Gd-doped
is proposed. The cross section of neutrons captured by Gd is 49000b. The Gd will be excited
after capturing neutrons and go back to ground state by emitting 8 MeV gamma rays. This
process is also a delayed signal as shown in Figure \ref{fig:delayedSignal} comparing to the positron annihilation (prompt).


\begin{equation}
n + Gd \rightarrow Gd^*
\label{NeutronGdCapture}
\end{equation}


\begin{equation}
Gd^* \rightarrow Gd + \Sigma \gamma (8 MeV)
\label{GdStarToGd}
\end{equation}


The scheme of the detection mechanism based on inverse $\beta$ decay is shown as Figure \ref{fig:ibd}.
The prompt and delayed signals are both time-tagged signals.
Together with the energy-tagged signals, the inverse beta-decay
in 0.1\% Gd-doped liquid scintillator is a good tool to suppress
background events.


\begin{figure}
    \label{fig:promptSignal}
    \centering
    \includegraphics[width=0.4\textwidth]{promptSignal.png}
    \caption
    [Reconstruction prompt signal and the energy cut]
    {Reconstruction prompt signal ( positron ) and the energy cut. This is Mote Carlo, not real data.}
    \end{figure}

\begin{figure}
    \label{fig:delayedSignal}
    \centering
    \includegraphics[width=0.4\textwidth]{delayedSignal.png}
    \caption
    [Reconstruction delayed signal and the energy cut]
    {Reconstruction delayed signal ( neutron captured by proton and Gd ) and the energy cut. This is Mote Carlo, not real data.}
    \end{figure}



%(inverse beta decay and signal scheme)
\begin{figure}
    \label{fig:ibd}
    \centering
    \includegraphics[width=0.6\textwidth]{ibd.png}
    \caption
    [Dayabay AD detection mechanism based on inverse $\beta$ decay]
    {Dayabay AD detection mechanism based on inverse $\beta$ decay}
    \end{figure}


The inverse beta-decay is adopted as the signals for the Dayabay antineutrino detectors.
The measurement of $\sin^{2}{2\theta_{13}}$ to less than 0.01 corresponds to a small
difference in a number of antineutrino events observed between the far site and
the near site.

The delayed signals are generated by the neutrons of inverse beta-decay process.
Neutrons are neutral particles so the direct detection is difficult.
A large mass antineutrino detector should be in the order of tons.
Typical requirement of an antineutrino detector is summarized in Table \ref{tab:ADRequirement}


\begin{table}
\centering
\caption{Physical requirements of the antineutrio detector\cite{TDR}}
\label{tab:ADRequirement}
\begin{tabular}{lp{2.5cm}p{5.0cm}}
\hline
Item & Requirement & Justification\\
\hline
\hline
Target mass at far site &
$\geq$ 80T &
Achieve sensitivity goal in three years over allowed ${\Delta}m^{2}_31 range$ \\
\hline
Precision on target mass &
$\leq$ 0.3\% &
Meet detector systematic uncertainty baseline per module \\
\hline
Energy resolution &
$\leq$ $15\%/\sqrt{E}$ &
Assure accurate calibration to achieve required uncertainty in energy-threshold cuts dominated by energy threshold cut \\
\hline
Detector efficiency error &
$\leq 0.2\%$ &
Should be small compared to target mass uncertainty \\
\hline
Positron energy threshold &
$\leq 1MeV$ &
Fully efficient for positrons of all energies \\
Radioactivity signals rate &
$\leq 50 Hz$ &
Limit accidental background to less than other backgrounds and keep data rate manageable \\
\hline
\end{tabular}
\end{table}


It's simpler to simulate and analyze the antineutrino detectors if they are
symmetric, so a spherical or a cylindrical shape could be considered. The cylindrical
shape is easier to produce from the engineering point of view, and also cheaper
price. Thus the antineutrino detectors are designed based on the cylindrical shape.

The delayed signal are the 2.2 $MeV$ and 8 $MeV$ gamma rays from neutron capture. There are external backgrounds like
muon-induced neutrons, radioactivity of PMT glasses, natural radioactivity of rock, etc.
Mineral oil is used for the antineutrino detectors to be a buffer for the detection of scintillation light.

Accroding to the neutron capture processes ( by protons and by Gd ) and the buffer to
suppress the background, 2-zone and 3-zone detectors are considered. The innermost
zone is the antineutrino target, composed of liquid scintillator doped 0.1\% Gd.
The second zone is gamma catcher, composed of pure liquid scintillator, for a
3-zone detector or mineral oil for a 2-zone detector. The outermost zone
is mineral oil for a 3-zone.

%(overview of the 2-zone and 3-zone antineutrino detecotrs)


The comparison for 2-zone and 3-zone detectors shows 
uncertainty of the neutron energy threshold, caused by uncertainty in the energy scale.
3-zone detectors could have better energy resolution where the energy scale
uncertainty is taken to be 1\% and 1.2\% at 6MeV and 4 MeV, respectively.\cite{TDR}

%(table of the comparison)

Thus 3-zone detectors are chosen. The innermost GdLS is 20 tons and the outer layer is another
20 tons of the undopped liquid scintillator. The outermost is 40 tons mineral oil. 192 eight-inch R5912
Hamamatsu photomutiplier tubes (PMTs) \cite{R5912} are mounted on the inner surface of the stainless steel tank
in eight horizontal rings, viewing the target volume and the gamma catcher. The crooss-sectional model of an AD
is shown in Fig. \ref{fig:AD_module.png}.

\begin{figure}
    \centering
    \includegraphics[width=0.6\textwidth]{AD_module.png}
    \caption{Cross-sectional model of the antineutrino detectors.
The model show complete AD including calibration boxes and overflow tanks on the top of the AD.}
    \label{fig:AD_module.png}
    \end{figure}




%-------------------
%TDR v2
%-------------------------

On the top of and beneath the gamma catcher, there is a reflector in each AD module respectively.
The reflector could enhance the total optical photon number received by PMTs.
The reflector is a acrylic-ESR-acrylic sandwich structure. The thickness of the acylic panel is 1.5 cm each.
The ESR, Enhanced Spectular Reflector Film, is a product of 3M.\cite{3M}
The typical reflectance of ESR is larger than 98\% over the most signal wavelength around 400 nm.
The typical thickness of ESR is 65 $\mu$m.
%\ref of ESR

%Table \ref{} shows the G4dyb simulation of the reflector distance associated to the total p.e. generated by the target.
%G4dyb is the Monte Carlo simulation software based on GEANT4. The details of G4dyb will be shown in Section{}.
%---------
%reflector
%----------
