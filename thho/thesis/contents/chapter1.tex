
\chapter{Introduction}

\section{Neutrinos}

The history of neutrinos.

In the early 1900s, William Bragg demonstrated that mono-energetic $\alpha$-rays
were emitted in $\alpha$ decays, which are two-body decays. Therefore $\beta$-rays
were supposed to be mono-energetic also because of 2-body decay of protons and
electrons. Nevertheless, C.D. Ellis and J. Chadwich showed that the $\beta$-ray
spectrum was continuous. Fig. \ref{fig:tmp} shows the typical beta-rays energy distribution curve.


\begin{figure}
    \centering
    \includegraphics[width=0.8\textwidth]{./figure/tmp.png}
    \caption{Energy distribution curve of the beta-rays\cite{Scott:1935}}
    \label{fig:tmp}
    \end{figure}


Pauli postulated a tiny neutron particle which we call neutrino today. The neutrino
of $\beta$ decay carrys away some energy so it's a 3-body decay.

The neutrions are not observed directly. Only the products from their interactions
with matter are seen. A particular neutrino flavor is associated to the obseved lepton type.


(overview of the flavor types, observed flavors and exps)




\subsection{Theta 13 and Standard Model}

The standard model of quarks and leptons. (Alan poon, OCPA slides)


standard model: m= 0

in the past decade, exp have shown indirectly through neutrino oscillation 
that neutrinos are massive.

For the time being, our reference to neutrino mass is the effective mass
associated with the flavor state.

\section{Underground Experiments}

(Alan poon, OCPA slides)

No. of observed events:

Because of small cross section, neutrinos do not interact readily, thus
a powerful neutrino sources are needed for study neutrinos. A source could be
accelerators, nuclear reactors, the sun, or supernovae.

(comparison of diff sources power)

And also, large mass targets are needed. A typical target is around in the order
of few tons, at least.





\subsection{Inverse Beta-decay}


The signal process, inverse beta-decay, emits neutrons and positrons.


\begin{equation}
\overline{\nu_e} + p \rightarrow e^+ + n
\label{InverseBetaDecay}
\end{equation}


The positrons will anniheas soon after they are generated by the inverse beta-decay,
and generate gammas which could be observed by PMTs. The neutrons will go through
thermal scattering process and finally captured by nuclies, namely protons. This
process will generate a 2.2Mev gamma each event, and it is a delayed signal comparing
to the positon annilesas.


\begin{equation}
n + p \rightarrow D + \gamma (2.2 MeV)
\label{NeutronPCapture}
\end{equation}


The cross section of the neutron captured by protons process is 0.3b, so a 0.1\%Gd-doped
is proposed. The cross section of neutrons captured by Gd is 50000b. The Gd will be excited
after capturing neutrons and go back to ground state by emitting gammas, totally 8 MeV. This
process is also a delayed comparing to the positron annealing.


\begin{equation}
n + Gd \rightarrow Gd^*
\label{NeutronGdCapture}
\end{equation}


\begin{equation}
Gd^* \rightarrow Gd + \gamma (8 MeV)
\label{GdStarToGd}
\end{equation}


The prompt and delayed signals are time-tagged signals.
Together with the energy-tagged signals, the inverse beta-decay
in 0.1\% Gd-doped liquid scintillator is a good tool to suppress
background events.


(inverse beta decay and signal scheme)


\section{The Dayabay Neutrino Oscillation Experiment}



\subsection{Progject Organization}

http://dayabay.ihep.ac.cn/cgi-bin/DocDB/ShowDocument?docid=2298

WBS, TDR


\subsection{Antineutrino Detectors}

The inverse beta-decay is adopted as the signals of Dayabay antineutrino detectors.
The measurement of sin2theta13 to less than 0.01 corresponds to a small
difference in a lot of antineutrino events observed between the far site and
the near site.

The delayed signals are generated by the neutrons of inverse beta-decay process.
Neutrons are neutral so the cross section with matters is small.
A large mass antineutrino detector should be in the order of tons.
Typical requirement of an antineutrino detector is summarized in Table \ref{tab:ADRequirement}


\begin{table}
\centering
\caption{Physical requirements of the antineutrio detector\cite{TDR}}
\label{tab:ADRequirement}
\begin{tabular}{lcp{5.0cm}}
\hline
Item & Requirement & Justification\\
\hline
\hline
Target mass at far site &
$\geq$ 80T &
Achieve sensitivity goal in three years over allowed ${\Delta}m^{2}_31 range$ \\
\hline
Precision on target mass &
$\leq$ 0.3\% &
Meet detector systematic uncertainty baseline per module \\
\hline
Energy resolution &
$\leq$ $15\%/\sqrt{E}$ &
Assure accurate calibration to achieve required uncertainty in energy-threshold cuts dominated by energy threshold cut \\
\hline
Detector efficiency error &
$\leq 0.2\%$ &
Should be small compared to target mass uncertainty \\
\hline
Positron energy threshold &
$\leq 1MeV$ &
Fully efficient for positrons of all energies \\
Radioactivity signals rate &
$\leq 50 Hz$ &
Limit accidental background to less than other backgrounds and keep data rate manageable \\
\hline
\end{tabular}
\end{table}


It's simpler to simulate and analysis the antineutrino detectors if they are
symmetric, so a sphere or a cylinder shape could be considered. The cylinder
shape is easier to produce from the point of engineering view, and also lower
price. Thus the antineutrino detectors are designed based on the cylinder shape.

The delayed signal are the 2.2MeV and 8MeV neutrons . There are external backgrounds like
muon-induced neutrons, radioactivity of PMT glasses, natural radioactivity of rock,
buffer water etc. Mineral oil is used for the antineutrino detectors to be a buffer.

Accroding to the neutron capture processes, by protons and by Gd, and the buffer to
suppress the background, 2-zone and 3-zone detectors are considered. The innermost
zone is the antineutrino target, composed of the liquid scintillator doped 1\% Gd.
The second zone is the gamma catcher, composed of the liquid scintillator, for a
3-zone detector or the mineral oil for a 2-zone detector. The outermost zone
is the mineral oil for a 3-zone.

(overview of the 2-zone and 3-zone antineutrino detecotrs)


The comparison for 2-zone and 3-zone detectors shows 
uncertainty of the neutron energy threshold, caused by uncertainty in the energy scale.
3-zone detectors could have better energy resolution where the energy scale
uncertainty is taken to be 1\% and 1.2\% at 6MeV and 4 MeV, respectively.

(table of the comparison)

Thus 3-zone detectors are used.





TDR v2

\section{A experiment at Aberdeen}

The Dayabay antineutrino signal is composed of a prompt positron signal and a delayed
neutron signal. Although the time and energy tagged signals could be powerful tool to suppress
the background, there are still accidentally conincidence signals. Any signal happens during
the typical time window after the prompt position signals happen for 30 micro-second to 120 micro-second,
which is typical thermal neutron drifting time, 
could be regarded as a antineutrino candidate. These accidnetally conincidence signals could be
muon-induced neutrons, radioactivity of PMT glasses, natural radioactivity of rock etc.
The radioactivity of PMT could be known by studying the ingredient of the PMTs and decreased
by the mineral oil buffer. The other natural radioactivity, like rock or dusts in the 
air could be studied or monitored directly. Muon-indeuced neutron is a complicated
process and needed measured the flux directly.

A neutron detector is proposed to measure muon-indeced neutron flux at Aberdeen, Hong Kong,
around 90 km away from Dayabay. The detector is positioned in Aberdeen tunnel, and the rock of the tunnel
providing veto of many cosmic rays. Muon, which has low attenuation length in rock, could
survives after passing through the rock of the tunnel.

Because of limitation of the tunnel room, a 2-zone detector is used. The inner zone
is composed of liquid scintillator doped 1\% Gd to be the neutron target and 
the outer zone is conposed of mineral oil. There are 16 8" Hamamatzu PMTs around the target.


Besides the neutron detector, there are total 6 layers muon trackers. Three layers
are above the neutron detector and three layers are below the neutron detector seperately.
The conincidence of the top and bottom muon trackers can tag the muon candidates
associated to the neutron signals in the neutron detector. Neutron signals happened
in the time window after muon candidates are tagged for around 30 micro-second to
120 micro-second, would be regarded as muon-induced neutron signal candidates.


(overview of the aberdeen neutron detector)


(overview the settings of Aberdeen exp)

(the process scheme to tag aberdeen exp)
