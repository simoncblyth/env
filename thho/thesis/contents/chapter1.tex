
\chapter{Introduction}

\section{Neutrinos}

The history of neutrinos.

In the early 1900s, William Bragg demonstrated that mono-energetic $\alpha$-rays
were emitted in $\alpha$ decays, which are two-body decays. Therefore $\beta$-rays
were supposed to be mono-energetic also because of 2-body decay of protons and
electrons. Nevertheless, C.D. Ellis and J. Chadwich showed that the $\beta$-ray
spectrum was continuous. Fig. \ref{fig:inverse_beta_decay_spectrum.png} shows the typical beta-rays energy distribution curve.


\begin{figure}
    \centering
    \includegraphics[width=0.8\textwidth]{./figure/inverse_beta_decay_spectrum.png}
    \caption{Energy distribution curve of the beta-rays\cite{Scott:1935}}
    \label{fig:inverse_beta_decay_spectrum.png}
    \end{figure}


Pauli postulated a tiny neutron particle which we call neutrino today. The neutrino
of $\beta$ decay carrys away some energy so it's a 3-body decay.

The neutrions are not observed directly. Only the products from their interactions
with matter are seen. A particular neutrino flavor is associated to the obseved lepton type.


(overview of the flavor types, observed flavors and exps)




\subsection{Neutrino Oscillation, Theta 13 and Standard Model}

The standard model of quarks and leptons. (Alan poon, OCPA slides)


standard model: m= 0

in the past decade, exp have shown indirectly through neutrino oscillation 
that neutrinos are massive.

For the time being, our reference to neutrino mass is the effective mass
associated with the flavor state.

\section{Underground Experiments}

(Alan poon, OCPA slides)

No. of observed events:

Because of small cross section, neutrinos do not interact readily, thus
a powerful neutrino sources are needed for study neutrinos. A source could be
accelerators, nuclear reactors, the sun, or supernovae.

(comparison of diff sources power)

And also, large mass targets are needed. A typical target is around in the order
of few tons, at least.





\subsection{Inverse Beta-decay}


The signal process, inverse beta-decay, emits neutrons and positrons.


\begin{equation}
\overline{\nu_e} + p \rightarrow e^+ + n
\label{InverseBetaDecay}
\end{equation}


The positrons will anniheas soon after they are generated by the inverse beta-decay,
and generate gammas which could be observed by PMTs. The neutrons will go through
thermal scattering process and finally captured by nuclies, namely protons. This
process will generate a 2.2Mev gamma each event, and it is a delayed signal comparing
to the positon annilesas.


\begin{equation}
n + p \rightarrow D + \gamma (2.2 MeV)
\label{NeutronPCapture}
\end{equation}


The cross section of the neutron captured by protons process is 0.3b, so a 0.1\%Gd-doped
is proposed. The cross section of neutrons captured by Gd is 50000b. The Gd will be excited
after capturing neutrons and go back to ground state by emitting gammas, totally 8 MeV. This
process is also a delayed comparing to the positron annealing.


\begin{equation}
n + Gd \rightarrow Gd^*
\label{NeutronGdCapture}
\end{equation}


\begin{equation}
Gd^* \rightarrow Gd + \Sigma \gamma (8 MeV)
\label{GdStarToGd}
\end{equation}


The prompt and delayed signals are time-tagged signals.
Together with the energy-tagged signals, the inverse beta-decay
in 0.1\% Gd-doped liquid scintillator is a good tool to suppress
background events.


(inverse beta decay and signal scheme)


\section{The Dayabay Neutrino Oscillation Experiment}

A deep underground neutrino observatory at Dayabay, China, has
been proposed. \cite{TDR}

\subsection{Overview}

Dayabay anti-neutrino detectors (AD), which will discussed more in section \ref{sec:AD},
will be placed at Dayabay. They will be placed at varying distance from the Dayabay reactor and
the Ling Ao reatcors as shown in \ref{fig:location_dayabay.png}. There are two sets of near detectors
at near sites, close to the Dayabay reactor and the Ling Ao reactors respectively, and two modules each set.
There is one set of far detectors, and four modules are there. The Dayabay near site with an overburden of
255 m.w.e.is located at a distance of 363m from the Dayabay cores. The Ling Ao near sitewith an overburden of
294 m.w.e. is 481m and 526m from the Ling Ao I and Ling Ao II cores respectively. The far site is at the distance
of 1985m and 1615m from the Dayabay cores and the mid-pointof the Ling Ao-Ling Ao II cores, and has an overburden
of 910 m.w.e.

At the time of writing, the third Ling Ao cores were proposed and are going to be built and run in around
2011. This effect should be study and has been studied by Dayabay collaborators.


\begin{figure}[h]
    \centering
    \includegraphics[width=0.8\textwidth]{./figure/location_dayabay.png}
    \caption{AD module positions at Dayabay.}
    \label{fig:location_dayabay.png}
    \end{figure}


The main backgrounds to the Dayabay Experiement are induced by cosmic-ray muons.
The site is located adjacent to mountainous terrain, ideal for sitting underground detectors that
are well shield from external backgrounds like cosmogeic backgrounds including the cosmic-ray mouons.
However, background due to the muon spallation products at the depths of the experiment halls and ambient
$\gamma$ background due to the radioactivity of the rock surrounding the experiment halls is reduced by shielding
the ADs with 2.5 meterof water. This water shield also attenuates the flux of neutrons produced outside the water
pool.

Events with a muons that passes through the water in the water pool less than 200 $\mu$s, which is the typical time window bewtween
a prompt signal and a delayed signal of an inverse-beta decay process, have a small but finite
probability to create a fake signal event.
There is a layer of  Resistive Plate Chambers (RPC) above a water pool
and 8" PMTs arround the water pool for the Cherenkov detection of muons to tag such muon signals.
The setup is shown in Fig. \ref{fig:water_pool.png}.


\begin{figure}[h]
    \centering
    \includegraphics[width=0.8\textwidth]{./figure/water_pool.png}
    \caption{Elevation view of an experiment hall,
showing the water pool to shield backgrounds from muons, muon-induced neutrons and radioactivity of rock.}
    \label{fig:water_pool.png}
    \end{figure}


\subsection{Progject Organization}

http://dayabay.ihep.ac.cn/cgi-bin/DocDB/ShowDocument?docid=2298

WBS, TDR



\section{A experiment at Aberdeen}

The Dayabay antineutrino signal is composed of a prompt positron signal and a delayed
neutron signal. Although the time and energy tagged signals could be powerful tool to suppress
the background, there are still accidentally conincidence signals. Any signal happens during
the typical time window after the prompt position signals happen for 30 micro-second to 120 micro-second,
which is typical thermal neutron drifting time, 
could be regarded as a antineutrino candidate. These accidnetally conincidence signals could be
muon-induced neutrons, radioactivity of PMT glasses, natural radioactivity of rock
( usually $^238_U$, $^232Th$, and $^40K$) etc. $(\alpha,n)$ processes may also produce
neutrons from natural radio-nuclides. Cosmogenic $^8He$ and $^9Li$ can undergo beta-neutron decay,
and this may also mimick neutrons from inverse-beta decay.

%The radioactivity of PMT could be known by studying the ingredient of the PMTs and decreased
%by the mineral oil buffer. The other natural radioactivity, like rock or dusts in the 
%air could be studied or monitored directly. Muon-induced neutron is a complicated
%process and needed measured the flux directly.

A muon-induced neutron detector is proposed to measure muon-indeced neutron flux Aberdenn, Hong Kong,
around 90 km away from Dayabay.
The detector is positioned at a vertical overburden of ~600 m.w.e. in Aberdeen tunnel, and the rock of the tunnel
providing veto of many cosmic rays. Muon, which has low attenuation length in rock, could
survives after passing through the rock of the tunnel.

Aberdeen is nearby Dayabay and with similar geological composition as Dayabay.
The Aberdeen Tunnel Experiment is aimed at studying those backgrounds encountered by the Dayabay
Experiment. The main focus of the Aberdeen Experiment is to study cosimic-ray muons in terms
of measuring their angular distribution, flux and contribution to neutron background in an
underground environment like the Dayabay ones.

The setup of the Aberdeen Tunnel Experiment is similar to the Dayabay one and consists of
two parts, the muon tracker and the neutron detecotr. The neutron detector is placed in the
middle of the muon tracker.
More details of the neutron detector will be disscussed in section \ref{sec:ND}.

The muon tracker, which consists of 63 scintillator hodoscopes and 72 proportional
tubes arranged in six layers.
Three layers are above the neutron detector and three layers are below the neutron detector seperately.
The top layer consists of 30 scintillator hodoscope of 1 m long and 10 cm wide with
one end readout by 2-inch PMTs. The second layer consists of 24 proportional tubes of diameters of
7.6 cm and lengths of 2 m placed orthogonal to the top layer. the third layer consists of 18 1.5 m-long
and 10 cm-wide scintillator hodoscope with one end readout by a 2" PMT. The forth layer is a row
of 24 proportional tubes of diameters of 7.6cm and lengths of 2 m placed orthogonal to the third layer.
The fifth layer is make up of 15 2m-long and 9.3 cm-wide scintillator hodoscope with both end
readout by Hamamatsu 7826 Small PMTs (SPMT). The last layer is an array of 24 proportional tubes
of diameters of 7.6 cm and lengths of 2 m placed  orthogonal to the fif layer.


(overview the settings of Aberdeen exp)


\begin{figure}[h]
    \centering
    \includegraphics[width=0.8\textwidth]{./figure/abtExp.png}
    \caption
    [Aberdeen Experiment Setup]
    {Aberdeen Experiment Setup}
    \label{fig:abtExp.png}
    \end{figure}


(overview of the aberdeen neutron detector)


The conincidence of the top and bottom muon trackers can tag the muon candidates
associated to the neutron signals in the neutron detector. Neutron signals happened
in the time window after muon candidates are tagged for around 30 micro-second to
120 micro-second, would be regarded as muon-induced neutron signal candidates.


(the process scheme to tag aberdeen exp)
