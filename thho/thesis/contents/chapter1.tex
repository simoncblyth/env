
\chapter{Introduction}

\section{Neutrinos}

The history of neutrinos.

\subsection{Theta 13 and Standard Model}

\subsection{Inverse Beta-decay}

The signal process, inverse beta-decay, emits neutrons and positrons.
The positrons will anniheas soon after they are generated by the inverse beta-decay,
and generate gammas which could be observed by PMTs. The neutrons will go through
thermal scattering process and finally captured by nuclies, namely protons. This
process will generate a 2.2Mev gamma each event, and it is a delayed signal comparing
to the positon annilesas.

The cross section of the neutron captured by protons process is 0.3b, so a 0.1\%Gd-doped
is proposed. The cross section of neutrons captured by Gd is 50000b. The Gd will be excited
after capturing neutrons and go back to ground state by emitting gammas, totally 8 MeV. This
process is also a delayed comparing to the positron annealing.


The prompt and delayed signals are time-tagged signals.
Together with the energy-tagged signals, the inverse beta-decay
in 0.1\% Gd-doped liquid scintillator is a good tool to suppress
background events.

\section{Underground Experiments}

\section{The Dayabay Neutrino Oscillation Experiment}



\subsection{Progject Organization}

http://dayabay.ihep.ac.cn/cgi-bin/DocDB/ShowDocument?docid=2298

\subsection{Antineutrino Detectors}

The inverse beta-decay is adopted as the signals of Dayabay antineutrino detectors.
The measurement of sin2theta13 to less than 0.01 corresponds to a small
difference in a lot of antineutrino events observed between the far site and
the near site.

The delayed signals are generated by the neutrons of inverse beta-decay process.
Neutrons are neutral so the cross section with matters is small.
A large mass antineutrino detector should be in the order of tons.



(overveiw of the phsics requirement of the antineutrino detectos, TDR2 p1)

It's simpler to simulate and analysis the antineutrino detectors if they are
symmetric, so a sphere or a cylinder shape could be considered. The cylinder
shape is easier to produce from the point of engineering view, and also lower
price. Thus the antineutrino detectors are designed based on the cylinder shape.
Accroding to the neutron capture processes, by protons and by Gd, there are






TDR v2

\section{A experiment at Aberdeen}
