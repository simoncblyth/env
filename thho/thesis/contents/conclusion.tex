\chapter {Summary and Conclusion}

The fabrication experience of inner acrylic vessel for Dayabay AD
is now mature.
The QA items are also established, including optical property, geometry, and
cleanliness.
The current study for the original and more realistic optical model of acrylics
shows the efficiency of n-Gd capture would not change much but the p.e.
number received by PMTs decreases by a few percent.
The manufacture time of the inner acrylic vessels by Nakano
could be shorter for the remaining 6 inner acrylic vessels.
It's expected the 6 inner acrylic vessels can be completed in one year.

The Aberdeen neutron detector is now running and at the stage of calibration.
The Aberdeen neutron detector reconstruction algorithm is completed and
now used for the calibration data.
This is a first test of the GA for Aberdeen neutron detector and real data.


\section {Outlook}

The Dayabay first pair of AD will begin running on March 2009.
The other ADs will be completed in the following 1 year.
Namely all the ADs for the far site and near sites
will run by the end of 2010.
Accroding to the sensitivity study,
it needs around 3 years to approach the goal of 0.01 or better.
%%%%
%sensitivity study ref
%%%%%%%%%%
The preliminary measurement of $\theta_{13}$ is expected
to be reached before 2014.


The integration of online and offline system of the Aberdeen experiment is at the final stage.
Towards the end of this year, the Aberdeen experiment is expected to take data.
It is hoped that the Aberdeen experiment could provide the neutron background for
the Dayabay Reactor Neutrino Experiment before the running of the Dayabay first pair next year.


